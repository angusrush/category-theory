\documentclass[notes.tex]{subfiles}

\begin{document}
\chapter{Higher categories}
\label{ch:higher_categories}

\section{2-categories}
\label{sec:2_categories}

\begin{definition}[2-category]
  \label{def:2_category}
  A \defn{2-category} is a category enriched over categories. More explicitly, a 2-category $\C$ consists of the following.
  \begin{itemize}
    \item A set $\Obj(\C)$ of objects.

    \item For every two objects $x$, $y \in \Obj(\C)$, a category $\C(x, y)$ of morphisms.

    \item For every object $x \in \Obj(\C)$, an object $\id_{x} \in \Obj(\C(x, y))$.

    \item For every three objects $x$, $y$, $z$, a functor
      \begin{equation*}
        \mu\colon \C(y, z) \times \C(x, y) \to \C(x, z)
      \end{equation*}
      implementing composition.

      These must make the following diagrams commute.
      \begin{equation*}
        \text{Associativity:}\qquad
        \begin{tikzcd}
          \C(z, w) \times \C(y, z) \times \C(x, y)
          \arrow[r]
          \arrow[d]
          & \C(z, w) \times \C(x, z)
          \arrow[d]
          \\
          \C(y, w) \times \C(x, y)
          \arrow[r]
          & \C(x, w)
        \end{tikzcd}
      \end{equation*}
      \begin{equation*}
        \text{Unitality:}\qquad
        \begin{tikzcd}
          \C(x, y)
          \arrow[r]
          \arrow[d]
          \arrow[dr, "\id"]
          & \C(y, y) \times \C(x, y)
          \arrow[d]
          \\
          \C(x, y) \times \C(x, x)
          \arrow[r]
          & \C(x, y)
        \end{tikzcd}
      \end{equation*}
  \end{itemize}
\end{definition}

\begin{example}
  The category $\mathsf{Cat}$ is a 2-category.

  \begin{itemize}
    \item The objects $\Obj(\mathsf{Cat})$ are given by small categories.

    \item The morphisms $\mathsf{Cat}(\mathsf{C}, \mathsf{D})$ is the category $[\mathsf{C}, \mathsf{D}]$.

    \item The identity morphism is the identity functor. We have already seen that this is a functor.

    \item The composition is given by composition of functors
  \end{itemize}
  These satisfy the conditions:
  \begin{enumerate}
    \item Indeed, the identity functor functions as a left and right identity, and whiskering by the identity is the identity.

    \item Indeed, composition of functors is associative, and vertical composition is associative, although I don't know a nice way to prove this.
  \end{enumerate}
\end{example}

\end{document}
