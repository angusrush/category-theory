\documentclass[a4paper,12pt]{scrbook}

\usepackage{mystyle}
\usepackage{subfiles}

\begin{document}

\begin{titlepage}
   \vspace*{\stretch{1.0}}
   \begin{center}
     \Huge\textbf{A Gentle Introduction to \\
     Category Theory}\\
     \Large(Work in progress) \\
     \vspace*{1em}
     \Large\textit{Angus Rush}
   \end{center}
   \vspace*{\stretch{1.0}}
   \begin{equation*}
     \begin{tikzcd}[row sep=huge, column sep=huge]
       \Hom(A, A)
       \arrow[r]
       \arrow[d]
       & \Hom(B, A)
       \arrow[d]
       \\
       \mathcal{F}(A)
       \arrow[r]
       & \mathcal{F}(B)
     \end{tikzcd}
   \end{equation*}
   \vspace*{\stretch{0.3}}
\end{titlepage}

\tableofcontents

\subfile{intro.tex}
\subfile{basics.tex}
\subfile{universal.tex}
\subfile{ccc.tex}
\subfile{yoneda.tex}
\subfile{limits.tex}
\subfile{adjunctions.tex}
\subfile{kan.tex}
\subfile{monoidal.tex}
\subfile{abelian.tex}
\subfile{tensor.tex}
\subfile{internalization.tex}
\subfile{enrichment.tex}
\subfile{higher.tex}
\begin{appendix}
  \subfile{universes.tex}
\end{appendix}

\begin{thebibliography}{9}
  \bibitem{michelson-lawson} M.-L. Michelson and H.B. Lawson.
    \textit{Spin Geometry}.
    Princeton University Press, Princeton, NJ, 1989.

  \bibitem{vakil-rising-sea} R. Vakil,
    \textit{The Rising Sea}.
    \url{http://www.math216.wordpress.com/}

  \bibitem{sontz-principal-bundles-classical} S. B. Sontz.
    \textit{Principal Bundles: The Classical Case}.
    Springer International Publishing, Switzerland, 2015.

  \bibitem{hungerford-algebra} T. Hungerford.
    \textit{Algebra}.
    Springer-Verlag New York, NY, 1974.

  \bibitem{o'farril-spin-geometry} J. Figuroa-O'Farril.
    \textit{PG Course on Spin Geometry}.
    \url{https://empg.maths.ed.ac.uk/Activities/Spin/}

  \bibitem{quantum-fields-and-strings} P. Deligne et al.
    \textit{Quantum Fields and Strings: a Course for Mathematicians}.
    American Mathematical Society, 1999.

  \bibitem{susy-for-mathematicians} V. S. Varadarajan.
    \textit{Supersymmetry for Mathematicans: An Introduction}.
    American Mathematical Society, 2004.

  \bibitem{catsters} The Catsters.
    \url{https://www.youtube.com/channel/UC5Y9H2KDRHZZTWZJtlH4VbA}

  \bibitem{connes-marcolli-ncg} A. Connes and M. Marcolli.
    \textit{Noncommutative Geometry, Quantum Fields and Motives}.
    \url{http://www.alainconnes.org/docs/bookwebfinal.pdf}

  \bibitem{aluffi-algebra-chapter-0} P. Aluffi.
    \textit{Algebra: Chapter 0}.
    American Mathematical Society, 2009.

  \bibitem{nlab-deligne-theorem}
    \textit{Nlab---Deligne's Theorem on Tensor Categories}.
    \url{https://ncatlab.org/nlab/show/Deligne's+theorem+on+tensor+categories}

  \bibitem{baez-this-weeks-finds-137} J. Baez.
    \textit{This Week's Finds in Mathematical Physics (Week 137)}.
    \url{http://math.ucr.edu/home/baez/week137.html}

  \bibitem{baez-definitions-everyone-should-know} J. Baez.
    \textit{Some Definitions Everyone Should Know}.
    \url{http://math.ucr.edu/home/baez/qg-fall2004/definitions.pdf}

  \bibitem{unapolagetic-mathematician-mac-lanes-theorem}
    \textit{The Unapologetic Mathematician: Mac Lane's Coherence Theorem}.
    \url{https://unapologetic.wordpress.com/2007/06/29/mac-lanes-coherence-theorem/}

  \bibitem{gleason-montgomery-zippin} D. Montgomery and L. Zippin.
    \textit{Topological Transformation Groups}.
    University of Chicago Press, Chicago, 1955.

  \bibitem{DMOS} P. Deligne, J.S. Milne, A. Ogus, and K. Shih.
    \textit{Hodge Cycles, Motives, and Shimura Varieties}.
    Springer Verlag, 1982

  \bibitem{maclane-categories} S. Mac Lane.
    \textit{Categories for the Working Mathematician}.
    Springer-Verlag New York, New York, 1998.

  \bibitem{baez-higher-categories} J. Baez.
    \textit{An introduction to $n$-Categories}.
    \url{https://arxiv.org/pdf/q-alg/9705009.pdf}

  \bibitem{baez-categories-usenet}
    \url{https://groups.google.com/forum/#!topic/sci.math/7LqPFfmWGOA}

  \bibitem{nlab}
    \url{https://www.ncatlab.org/}

  \bibitem{wikipedia-product}
    \textit{Wikipedia - Product (Category Theory)}.
    \url{https://en.wikipedia.org/wiki/Product\_(category\_theory)}

  \bibitem{EGNO-tensor-categories} P. Etingof, S. Gelaki, D. Nikshych, and V. Ostrik.
    \textit{Tensor Categories}.
    American Mathematical Society, 2015.

  \bibitem{awodey-intro-to-categories} S. Awodey and A. Bauer.
    \textit{Lecture Notes: Introduction to Categorical Logic}

  \bibitem{awodey-category-theory-foundations-videos} S. Awodey.
    \textit{Category Theory Foundations}.
    \url{https://www.youtube.com/watch?v=BF6kHD1DAeU&index=1&list=PLGCr8P\_YncjVjwAxrifKgcQYtbZ3zuPlb}

  \bibitem{awodey-category-theory} S. Awodey.
    \textit{Category Theory}.
    Oxford University Press, 2006.

  \bibitem{haag-local-quantum-physics} R. Haag.
    \textit{Local Quantum Physics: Fields, Particles, Algebras}.
    Springer-Verlag Berlin Heidelberg New York, 1996

  \bibitem{sexl-urbantke-relativity-groups-particles} R. Sexl and H. Urbantke.
    \textit{Relativity, Groups, Particles: Special Relativity and Relativistic Symmetry in Field and Particle Physics}.
    Springer-Verlag Wein, 1992

  \bibitem{wess-bagger-susy-sugra} J. Wess and J. Bagger.
    \textit{Supersymmetry and Supergravity}.
    Princeton University Press, Princeton NJ, 1992

  \bibitem{muller-kirsen-wiedmann-intro-susy} H J W M{\"u}ller-Kirsen and Armin Wiedemann
    \textit{Introduction to Supersymmetry (Second edition)}.
    World Scientific, 2010.

  \bibitem{deligne-categories-tensorielle} Pierre Deligne.
    \textit{Cat{\'e}gories Tensorielle},
    Moscow Math. Journal 2 (2002) no. 2, 227-228
    \url{https://www.math.ias.edu/files/deligne/Tensorielles.pdf}

  \bibitem{KMS-natural-operations-differential-geometry} I. Kol\'{a}\v{r}, P. Michor, and J. Slov\'{a}k.
    \textit{Natural Operations in Differential Geometry}.
    Springer-Verlag, Berlin Heidelberg, 1993.

  \bibitem{hartshorne-algebraic-geometry} R. Hartshorne.
    \textit{Algebraic Geometry}.
    Springer-Verlag, New York, 1977.

  \bibitem{annoying-precision-meditation} Q. Yuan.
    \textit{Annoying Precision: A neditation on semiadditive categories}.
    \url{https://qchu.wordpress.com/2012/09/14/a-meditation-on-semiadditive-categories/}

  \bibitem{nestruev-smooth-manifolds-observables} J. Nestruev.
    \textit{Smooth Monifolds and Observables}.
    Springer-Verlag, New York, 2002.

  \bibitem{baez-lauda-prehistory} J. Baez and A. Lauda.
    \textit{A prehistory of $n$-categorical physics}.
    \url{https://arxiv.org/pdf/0908.2469.pdf}

  \bibitem{neumaier} A. Neumaier.
    \textit{Elementary particles as irreducible representations}.
    \url{https://www.physicsoverflow.org/21960}.

  \bibitem{physicsforums-why-deligne} U. Schreiber.
    \textit{Why Supersymmetry? Because of Deligne's theorem}
    \url{https://www.physicsforums.com/insights/supersymmetry-delignes-theorem/}

  \bibitem{mackey-induced-representations} G. W. Mackey.
    \textit{Induced representations}.
    W.A. Benjamin, Inc., and Editore Boringhieri, New York, 1968.

  \bibitem{nlab-additive-category} nLab: Additive category.
    \url{https://ncatlab.org/nlab/show/additive+category#ProductsAreBiproducts}

  \bibitem{freyd-abelian-categories} P. Freyd.
    \textit{Abelian Categories}
    Harper and Row, New York, 1964.

  \bibitem{milne-affine-group-schemes} J. Milne,
    \textit{Basic Theory of Affine Group Schemes},
    2012.
    Available at \url{www.jmilne.org/math/}

  \bibitem{wigner-little-group} E. Wigner.
    \textit{On Unitary Representations of the Inhomogeneous Lorentz Group.}
    Annals of Mathematics. Second Series, Vol. 40, No. 1 (Jan., 1939), pp. 149-204

  \bibitem{jpmds-representation-theory-symmetric-groups} J. P. M. dos Santos.
    \textit{Representation theory of symmetric groups.}
    \url{https://www.math.tecnico.ulisboa.pt/~ggranja/joaopedro.pdf}

  \bibitem{manin-gauge-fields} U. Manin.
    \textit{Gauge Fields and Complex Geometry},
    Springer-Verlag Berlin Heidelberg New York, 1988.

  \bibitem{dissipation-in-lagrangian-mechanics} Valter Moretti (\url{https://physics.stackexchange.com/users/35354/valter-moretti}).
    \textit{Euler-Lagrange equations and friction forces},
    URL (version: 2014-02-02): \url{https://physics.stackexchange.com/q/96470}

  \bibitem{milneliealgebrasalgebraicgroupsliegroups} J.S. Milne.
    \textit{Lie Algebras, Algebraic Groups, and Lie Groups.}
    Available at \url{www.jmilne.org/math/}

  \bibitem{braidstatisticsinlocalqft} J. Fr\"{o}hlich and F. Gabbiani.
    \textit{Braid statistics in Local Quantum Theory}.
    Rev. Math. Phys. 02, 251 (1990).

  \bibitem{binneyphysicsofqm} J. Binney and D. Skinner.
    \textit{The Physics of Quantum Mechanics.}
    Available at \url{https://www-thphys.physics.ox.ac.uk/people/JamesBinney/qb.pdf}

  \bibitem{feynmanpathintegral} R. Feynman.
    \textit{Spacetime Approach to Non-Relativistic Quantum Mechanics.}
    Rev. Mod. Phys. 20, 367 – Published 1 April 1948

  \bibitem{berezinsecondquantization} F. A. Berezin.
    \textit{The Method of Second Quantization.}
    Nauka, Moscow, 1965. Tranlation: Academic Press, New York, 1966. (Second edition, expanded: M. K. Polivanov, ed., Nauka, Moscow, 1986.)

  \bibitem{coleman-mandula} S. Coleman and J. Mandula.
    \textit{All Possible Symmetries of the S Matrix}.
    Physical Review, 159(5), 1967, pp. 1251–1256.

  \bibitem{haag-lopuszanski-sohnius} R. Haag, M. Sohnius, and J. T. {\L}opusza\'{n}ski.
    \textit{All possible generators of supersymmetries of the S-matrix}.
    Nuclear Physics B, 88: 257–274 (1975)
\end{thebibliography}
\end{document}
