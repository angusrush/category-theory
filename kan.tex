\documentclass[notes.tex]{subfiles}

\begin{document}

\chapter{Kan extensions}
\label{sec:kan_extensions}

\section{Motivation}
\label{sec:kan_extensions_motivation}

Let $\mathsf{C}$ be a category, and $\mathcal{I}\colon \mathsf{S} \to \mathsf{C}$ a subcategory inclusion. Given a functor $\mathcal{F}$ from $\mathsf{C}$ to any category $\mathsf{D}$, we can restrict $\mathcal{F}$ to $\mathsf{S}$ by pulling it back with $\mathcal{I}$. We denote this restriction by
\begin{equation*}
  \mathcal{I}^{*}(\mathcal{F}) = \mathcal{F} \circ \mathcal{I}.
\end{equation*}

In fact, $\mathcal{I}$ induces a functor from the category of functors $\mathsf{C} \to \mathsf{D}$ (\hyperref[def:functorcategory]{Definition~\ref*{def:functorcategory}}) to the category of functors $\mathsf{S} \to \mathsf{D}$. Recall that we denoted these categories by $[\mathsf{C}, \mathsf{D}]$ and $[\mathsf{S}, \mathsf{D}]$ respectively. What we have created is a functor
\begin{equation*}
  \mathcal{I}^{*}\colon [\mathsf{C}, \mathsf{D}] \to [\mathsf{S}, \mathsf{D}]
\end{equation*}
which acts on functors $\mathsf{C} \to \mathsf{D}$ by restricting their domain to the subcategory $\mathsf{S}$.

Admittedly, restricting the domain of a functor is not a very good party trick, even if done in a functorial way. The more impressive thing would be to \emph{extend} the domain of a functor along some inclusion. The problem of extending the domain of a functor is one example of the sort of problem that Kan extensions are designed to do.

We have come up against the problem of finding a weak inverse to a lossy process before. For example, it is easy to turn a group into a set via the lossy process of forgetting the group structure; this is done by a functor $\mathcal{U}\colon \mathsf{Grp} \to \mathsf{Set}$. The inverse problem, i.e. turning a set into a group, is solved (to the extent possible) by taking a left adjoint to the forgetful functor. Together, these two functors form an adjunction, an example of a more general phenomenon known as \emph{free-forgetful adjunction} (\hyperref[eg:freeforgetfuladjunctions]{Note~\ref*{eg:freeforgetfuladjunctions}}). If we believe the mantra that adjunctions are weak inverses, we should try to solve the problem of extending the domain of a functor by searching for a functor which is adjoint to the restriction functor $\mathcal{I}^{*}$.

Of course, it is also interesting to consider the case where $\mathcal{I}$ is not an inclusion. In fact, we will see that by choosing $\mathcal{I}$ in different ways leads to a generalization of many different categorical concepts, such as limits.

\section{Global Kan extensions}
\label{sec:global_kan_extensions}

The most high-level notion (although, as we will see, not the most useful) of a Kan extension is called a \emph{global Kan extension.} This is the concept discussed in \hyperref[sec:kan_extensions_motivation]{Section~\ref*{sec:kan_extensions_motivation}}.

\begin{definition}[pullback]
  \label{def:pullback}
  Let $\mathcal{F}\colon \mathsf{C} \to \mathsf{C}'$ be a functor. For any other category $\mathsf{D}$, $\mathcal{F}$ induces a functor, called the \defn{pullback by $\mathcal{F}$},
  \begin{equation*}
    \mathcal{F}^{*}\colon [\mathsf{C}', \mathsf{D}] \to [\mathsf{C}, \mathsf{D}]
  \end{equation*}
  which sends a functor $\mathcal{G}\colon \mathsf{C}' \to \mathsf{D}$ to its precomposition $\mathcal{G} \circ \mathcal{F}$
  \begin{equation*}
    \begin{tikzcd}
      \mathsf{C}'
      \arrow[r, "\mathcal{G}"]
      & \mathsf{D}
    \end{tikzcd}
    \qquad
    \mapsto
    \qquad
    \begin{tikzcd}[column sep=small]
      \mathsf{C}
      \arrow[dr, swap, "\mathcal{F}"]
      \arrow[rr, "\mathcal{F}^{*}\mathcal{G}"]
      && \mathsf{D}
      \\
      & \mathsf{C'}
      \arrow[ur, swap, "\mathcal{G}"]
    \end{tikzcd}
  \end{equation*}
  and a natural transformation $\eta\colon \mathcal{G} \to \mathcal{G}'$ to its left whiskering (\hyperref[eg:whiskering]{Example~\ref*{eg:whiskering}}) $\eta\mathcal{F}$.
  \begin{equation*}
    \begin{tikzcd}[column sep=huge]
      \mathsf{C}
      \arrow[r, "\mathcal{F}"]
      &\mathsf{C}'
      \arrow[r, bend left, "\mathcal{G}"{name=U}]
      \arrow[r, bend right, swap, "\mathcal{G}'"{name=D}]
      & \mathsf{D}
      \arrow[from=U, to=D, Rightarrow, "\eta"]
    \end{tikzcd}
    \qquad
    \mapsto
    \qquad
    \begin{tikzcd}[column sep=huge]
      \mathsf{C}
      \arrow[r, bend left, "\mathcal{G} \circ \mathcal{F}"{name=U}]
      \arrow[r, bend right, swap, "\mathcal{G} \circ \mathcal{F}'"{name=D}]
      & \mathsf{D}
      \arrow[from=U, to=D, Rightarrow, "\mathcal{F}^{*}(\eta)"]
    \end{tikzcd}
  \end{equation*}
\end{definition}

\begin{definition}[Kan extension functor]
  \label{def:kan_extension}
  Let $\mathcal{F}\colon \mathsf{C} \to \mathsf{C}'$ be a functor, and let $\mathsf{D}$ be any other category. Consider the pullback functor $\mathcal{F}^{*}$.

  \begin{itemize}
    \item If $\mathcal{F}^{*}$ has right adjoint
      \begin{equation*}
        \mathcal{F}_{*}\colon [\mathsf{C}, \mathsf{D}] \to [\mathsf{C}', \mathsf{D}].
      \end{equation*}
      We call $\mathcal{F}_{*}$ the \defn{right Kan extension functor} along $\mathcal{F}$.
      %, and for any $\mathcal{G} \in [\mathsf{C}, \mathsf{D}]$, we call $\mathcal{F}_{*}(\mathcal{G})$ the \defn{right Kan extension of $\mathcal{G}$ along $\mathcal{F}$}.

    \item If $\mathcal{F}^{*}$ has left adjoint
      \begin{equation*}
        \mathcal{F}_{!}\colon [\mathsf{C}, \mathsf{D}] \to [\mathsf{C}', \mathsf{D}].
      \end{equation*}
      We call $\mathcal{F}_{!}$ the \defn{left Kan extension functor} along $\mathcal{F}$.
      %For $\mathcal{\mathcal{G}} \in [\mathsf{C}, \mathsf{D}]$, we call $\mathcal{F}_{!}(\mathcal{G})$ the \defn{left Kan extension of $\mathcal{G}$ along $\mathcal{F}$}.
  \end{itemize}
\end{definition}

\section{Kan extensions capture limits and colimits}
\label{sec:kan_extensions_capture_limits_and_colimits}

When working with Kan extensions of a functor $\mathcal{D}$ along another functor $\mathcal{F}$, it is always helpful to have a drawing of the following triangle handy.

\begin{equation*}
  \begin{tikzcd}
    \mathsf{C}
    \arrow[rr, "\mathcal{D}"]
    \arrow[dr, swap, "\mathcal{F}"]
    && \mathsf{D}
    \\
    &\mathsf{C}'
  \end{tikzcd}
\end{equation*}

Consider the case in which $\mathsf{C}'$ is the terminal category $\mathsf{1}$ (\hyperref[eg:categorywithoneobject]{Example~\ref*{eg:categorywithoneobject}}). There is a unique functor from any category $\mathsf{C} \to \mathsf{1}$, which sends every object to the unique object $* \in \Obj(\mathsf{1})$ and every morphism to the identity morphism. Furthermore, any functor $\mathcal{F}\colon 1 \to \mathsf{D}$ is completely determined by where it sends the unique object $*$.

\begin{equation*}
  \begin{tikzcd}
    \mathsf{C}
    \arrow[rr, "\mathcal{D}"]
    \arrow[dr, swap, "\mathcal{C}"]
    && \mathsf{D}
    \\
    & \mathsf{1}
  \end{tikzcd}
\end{equation*}

The restriction of any $\mathcal{F}\colon \mathsf{1} \to \mathsf{C}$ along $\mathcal{C}$ is simply the constant functor $\Delta_{\mathcal{F}(*)}$. Therefore, the functor $\mathcal{C}^{*}\colon [\mathsf{1},\mathsf{D}] \to [\mathsf{C}, \mathsf{D}]$ carries exactly the same data as the diagonal functor $\Delta\colon \mathsf{D} \to [\mathsf{C},\mathsf{D}]$. Thus, $\mathcal{C}_{!}$ fits into the following adjunction.
\begin{equation*}
  \mathcal{C}_{!} : [\mathsf{C}, \mathsf{D}] \leftrightarrow \mathsf{D} : \Delta
\end{equation*}

But we have seen that the colimit functor is right adjoint to the diagonal functor, so we have a natural isomorphism
\begin{equation*}
  \mathcal{C}_{!} \simeq \lim_{\rightarrow}.
\end{equation*}
That is, the left Kan extension of any functor $\mathcal{D}$ along the terminal functor $\mathcal{C}$ is\footnote{Strictly speaking, we should write ``uniquely isomorphic to,'' but we don't because limits are only defined up to unique isomorphism anyway.} the colimit of $\mathcal{D}$:
\begin{equation*}
  \mathcal{C}_{!}\mathcal{D} = \lim_{\rightarrow} \mathcal{D}.
\end{equation*}

Exactly analogous reasoning tells us that the right Kan extension of any functor $\mathcal{D}$ along the terminal functor $\mathcal{C}$ is the limit of $\mathcal{D}$:
\begin{equation*}
  \mathcal{C}_{*}\mathcal{D} = \lim_{\leftarrow} \mathcal{D}.
\end{equation*}


\section{Local Kan extensions}
\label{sec:local_kan_extensions}

By \hyperref[sec:unit_counit_adjunctions_are_hom_set_adjunctions]{Section~\ref*{sec:unit_counit_adjunctions_are_hom_set_adjunctions}}, we can also look at the right-adjointness of the pullback functor $\mathcal{F}^{*}$'s as guaranteeing that for every $\mathcal{D} \in \Obj([\mathsf{C}', \mathsf{D}])$, the category $(\mathcal{F}^{*} \downarrow \mathcal{D})$ has a terminal object. Similarly, we can look at $\mathcal{F}_{!}$'s left-adjointness as demonstrating that any category $(\mathcal{D} \downarrow \mathcal{F}^{*})$ has an initial object. This mirrors the definition of limits and colimits as being terminal (resp. initial) objects in the category $(\Delta \downarrow \mathcal{D})$ (resp. $(\mathcal{D} \downarrow \Delta)$). As we saw in \hyperref[sec:kan_extensions_capture_limits_and_colimits]{section~\ref*{sec:kan_extensions_capture_limits_and_colimits}}, this is no accident; Kan extensions form a vast generalization of limits and colimits.

For now, we will specialize our discussion to the case of right Kan extensions and limits, so as not to have to keep writing down 'resp.'

Recall that we defined limits as universal cones. We have seen that there is a functor which assigns limits, and that this is the right adjoint to the diagonal functor. This gives us another way to define limits: we could first have defined the limit functor as right adjoint to the diagonal functor, and then defined limits as being the images of the limit functor. However, this would have been an unnecessarily restrictive definition, since in general some, but not all, limits exist; if we had chosen to define limits in this functorial way, in order to define limits for any diagram $\mathcal{D}\colon \mathsf{J} \to \mathsf{C}$ we would have had to demand that limits exist for \emph{every} diagram in $\mathsf{D}$. Clearly this is not what we want; for example, we want to be able to take products of two objects in a category without worrying about whether all limits exist in that category.

Fortunately, this is not what we did: we first defined the categories $(\Delta \downarrow \mathcal{D})$, then defined limits as their terminal objects. Only much later did we show that under certain conditions, limits organized themselves into a functor whose left adjoint was the diagonal functor.

By comparison, we first defined right Kan extensions as right adjoints to a restriction functor, then showed that their existence was equivalent to the existence of terminal morphisms in appropriate comma categories. The analogy with limits suggests that we can define Kan extension of one functor along another even when not all extensions exist.

\begin{definition}[local Kan extension]
  \label{def:local_kan_extension}
  Consider categories and functors as below.
  \begin{equation*}
    \begin{tikzcd}
      \mathsf{C}
      \arrow[dr, swap, "\mathcal{F}"]
      \arrow[rr, "\mathcal{D}"]
      && \mathsf{D}
      \\
      & \mathsf{C'}
    \end{tikzcd}
  \end{equation*}
  \begin{itemize}
    \item A \defn{right Kan extension of $\mathcal{D}$ along $\mathcal{F}$} is a terminal morphism in the category $(\mathcal{F}^{*} \downarrow \mathcal{D})$.

    \item A \defn{left Kan extension of $\mathcal{D}$ along $\mathcal{F}$} is a terminal morphism in the category $(\mathcal{D}\downarrow \mathcal{F}^{*})$.
  \end{itemize}
\end{definition}

This local definition will turn out to be much more useful than \hyperref[def:kan_extension]{Definition~\ref*{def:kan_extension}}. It will even to allow us to compute Kan extensions via an explicit formula in certain situations.

\begin{example}
  In the case where $\mathsf{C}' = \mathsf{1}$, we saw that global left and right Kan extensions recovered the limit and colimit functors. Essentially by definition, this works for local Kan extensions as well; we have
  \begin{equation*}
    \mathcal{C}_{*}\mathcal{D} = \lim_{\leftarrow} \mathcal{D},\qquad \text{and}\qquad \mathcal{C}_{!}\mathcal{D} = \lim_{\rightarrow} \mathcal{D}.
  \end{equation*}
\end{example}

We defined limits in terms of cones. The close analogy between limits and Kan extensions suggests that we should define a Kan extensionish analog of cones.

The appropriate notion is that of an \emph{extension.} Extensions generalize (co)cones in the sense that they reduce to them in the case of Kan extensions along $\mathcal{C}$.

\begin{definition}[extension]
  \label{def:extension}
  Consider categories and functors as below.
  \begin{equation*}
    \begin{tikzcd}
      \mathsf{C}
      \arrow[dr, swap, "\mathcal{F}"]
      \arrow[rr, "\mathcal{D}"]
      && \mathsf{D}
      \\
      & \mathsf{C'}
    \end{tikzcd}
  \end{equation*}
  \begin{itemize}
    \item A \defn{right extension of $\mathcal{G}$ along $\mathcal{F}$} is an object in the comma category $(\mathcal{F}^{*} \downarrow \mathcal{D})$. That is, it is a pair $(\mathcal{G}, \eta)$, where $\mathcal{G}\colon \mathsf{C}' \to \mathsf{D}$ and
      \begin{equation*}
        \eta\colon \mathcal{D} \Rightarrow \mathcal{F}^{*}\mathcal{G}
      \end{equation*}

    \item A \defn{left extension of $\mathcal{G}$ along $\mathcal{F}$} is an object in the comma category $(\mathcal{D} \downarrow \mathcal{F}^{*})$.
  \end{itemize}
\end{definition}

\begin{example}
  \label{eg:kan_extension_along_functor_to_discrete_groupoid}
  We have seen that Kan extensions along the functor $\mathsf{C} \to \mathsf{1}$ reproduce limits and colimits. It will be instructive to study Kan extensions along functors $\mathsf{C} \to \mathsf{n}$, where $\mathsf{n}$ is the category with $n$ objects and only identity morphisms, i.e.\ the discrete groupoid with $n$ objects.

  Consider such a functor $\mathsf{C} \to \mathsf{n}$. Denote by $i$ the functor $\mathsf{1} \to \mathsf{n}$ which takes $* \in \Obj(\mathsf{1})$ to the $i$th object in $\mathsf{n}$.

  A functor $\mathcal{F}\colon \mathsf{C} \to \mathsf{n}$ is the same thing as a decomposition
  \begin{equation*}
    \mathsf{C} = \coprod_{c = 1}^{n} \mathsf{C}_{i},
  \end{equation*}
  where each $\mathsf{C}_{i}$ is the fiber over $i \in \Obj(\mathsf{n})$,
  together with a family of functors
  \begin{equation*}
    \mathcal{D}|_{\mathsf{C}_{i}}\colon \mathsf{C}_{i} \to \mathsf{1}.
  \end{equation*}

  Consider right extensions of $\mathcal{D}$ along $\mathcal{F}$, i.e.\ objects in $(\mathcal{F}^{*} \downarrow \mathcal{D})$. These consist of pairs $(\eta, \mathcal{G})$, where $\eta\colon \mathcal{F}^{*}\mathcal{G} \to \mathcal{D}$.
  \begin{equation*}
    \begin{tikzcd}
      \mathsf{C}
      \arrow[dr, swap, "\mathcal{F}"]
      \arrow[rr, ""'{name=U}, "\mathcal{D}"]
      && \mathsf{D}
      \\
      & \mathsf{n}
      \arrow[ur, swap, "\mathcal{G}"]
      \arrow[to=U, Rightarrow, swap, "\eta"]
    \end{tikzcd}
  \end{equation*}

  The functor $\mathcal{F}^{*}\mathcal{G}$ is really the composition $\mathcal{G} \circ \mathcal{F}$, which (by the universal property for coproducts) consists of $n$ constant functors $\Delta^{i}_{d}\colon \mathsf{C}_{i} \to \mathsf{D}$, where $\Delta^{i}_{d}$ is the constant functor mapping everything in $\mathsf{C}_{i}$ to $d \in \Obj(\mathsf{D})$. The natural transformation $\eta$ therefore consists of $n$ cones
  \begin{equation*}
    \eta^{i}\colon \Delta^{i}_{d} \Rightarrow \mathcal{D}|_{\mathsf{C}_{i}}.
  \end{equation*}

  Such a right extension is universal if any other right extension factors through it. That is, given another right extension $(\epsilon, \mathcal{H})$, there is a unique natural transformation $\xi\colon \mathcal{H} \circ \mathcal{F} \Rightarrow \mathcal{G} \circ \mathcal{F}$ making the following diagram commute.
  \begin{equation*}
    \begin{tikzcd}[column sep=small]
      \mathcal{H} \circ \mathcal{F}
      \arrow[rr, Rightarrow, "\xi"]
      \arrow[dr, Rightarrow, swap, "\epsilon"]
      && \mathcal{G} \circ \mathcal{F}
      \arrow[dl, Rightarrow, "\eta"]
      \\
      & \mathcal{D}
    \end{tikzcd}
  \end{equation*}
  This says precisely that each cone must be universal. That is, we have
  \begin{equation*}
    (\mathcal{F}_{*}\mathcal{D})(i) = \lim_{\leftarrow} \mathcal{D}|_{\mathsf{C}_{i}},
  \end{equation*}
  where $\mathsf{D}|_{\mathsf{C}_{i}}$ is simply $\mathcal{D}$ restricted to the fiber over $i$.

  The same is true for left Kan extensions; the value
  \begin{equation*}
    (\mathcal{F}_{!}\mathcal{D})(i) = \lim_{\rightarrow} \mathcal{D}|_{\mathsf{C}_{i}}.
  \end{equation*}
\end{example}

We have more or less proved the following. I still need to work out size conditions, e.g.\ $\mathsf{C}'$ should almost certainly have to be small.
\begin{proposition}
  \label{prop:formula_for_kan_extension_through_groupoid}
  Let $\mathsf{C}$ and $\mathsf{D}$ be categories and $\mathsf{C}'$ a discrete groupoid, and let $\mathcal{F}\colon \mathsf{C} \to \mathsf{C}'$ and $\mathcal{D}\colon \mathsf{C} \to \mathsf{D}$ be functors.
  \begin{itemize}
    \item Suppose the limit $\lim_{\leftarrow}\mathcal{D}|_{\mathsf{C}_{i}}$ exists for all $i \in \Obj(\mathsf{C}')$. Then a right Kan extension $\mathcal{F}_{*}\mathcal{D}$ exists, and is given by the formula
      \begin{equation*}
        (\mathcal{F}_{*}\mathcal{D})(i) = \lim_{\leftarrow} \mathcal{D}|_{\mathsf{C}_{i}},
      \end{equation*}

    \item Suppose the colimit $\lim_{\rightarrow}\mathcal{D}|_{\mathsf{C}_{i}}$ exists for all $i \in \Obj(\mathsf{C}')$. Then a left Kan extension $\mathcal{F}_{!}\mathcal{D}$ exists, and is given by the formula
      \begin{equation*}
        (\mathcal{F}_{!}\mathcal{D})(i) = \lim_{\rightarrow} \mathcal{D}|_{\mathsf{C}_{i}}.
      \end{equation*}
  \end{itemize}
\end{proposition}

This says that a Kan extension along a functor to a discrete groupoid $\mathsf{C}'$ consists of a collection of (co)limits parametrized by the objects of $\mathsf{C}'$.

\begin{note}
  In our simple case, the converse also holds: any Kan extension through a discrete groupoid is given by the above formulae. This is \emph{not} a general feature of the formula we derive! There are Kan extensions whose
\end{note}

\section{A pointwise formula for Kan extensions}
\label{sec:a_pointwise_formula_for_kan_extensions}

One would hope that we would be able to generalize the above formulae easily. However, their derivation rested crucially on the fact that for any category $\mathsf{C}'$ with no morphisms, any functor $\mathsf{C} \to \mathsf{C}'$ splits $\mathsf{C}$ into a disjoint union of its fibers, and we can easily construct (co)cones over the images of these fibers. Adding any morphisms to $\mathsf{C}'$ ruins this splitting, and we lose our (co)cones.

The solution to this problem is to use `smarter' categories than the fibers of $\mathcal{F}$, over which we can still construct (co)cones in an obvious way: comma categories.

The intuition is as follows. Any morphism $f\colon c' \to \tilde{c}'$ we add to $\mathsf{C}'$ allows the fiber $\mathsf{C}_{c'}$ to map into to the fiber $\mathsf{C}_{\tilde{c}'}$. We should include in the base of our cone not just the parts of $\mathsf{C}$ which are mapped to the identity $\id_{c'}$ (i.e.\ the fiber over $c'$), but also any fibers to which $\mathsf{C}_{c'}$ has access via some morphism. We have such a category: $(c' \downarrow \mathcal{F})$.

Let $\mathcal{D}\colon \mathsf{C} \to \mathsf{D}$ and $\mathcal{F}\colon \mathsf{C} \to \mathsf{C'}$ be functors. Let $(\mathcal{G}, \eta)$ be a right extension of $\mathcal{D}$ along $\mathcal{F}$.
\begin{equation*}
  \begin{tikzcd}
    \mathsf{C}
    \arrow[dr, swap, "\mathcal{F}"]
    \arrow[rr, "\mathcal{D}", ""'{name=U}]
    && \mathsf{D}
    \\
    & \mathsf{C'}
    \arrow[ur, swap, "\mathcal{G}"]
    \arrow[d, to=U, Rightarrow, swap, "\eta"]
  \end{tikzcd}
\end{equation*}
For any $c' \in \Obj(\mathsf{C}')$, denote by $(c' \downarrow \mathcal{F})$ the category of morphisms from $c'$ down to $\mathcal{F}$ (see \hyperref[def:categoryofmorphismsfromanobjecttoafunctor]{Definition~\ref*{def:categoryofmorphismsfromanobjecttoafunctor}}), and denote by $\mathcal{U}$ the forgetful functor $(c' \downarrow \mathcal{F}) \to \mathsf{C}$. We can define a functor $\evl{\mathcal{D}}_{(c' \downarrow \mathcal{F})}\colon (c' \downarrow \mathcal{F}) \to \mathsf{D}$ as the following composition.
\begin{equation*}
  \begin{tikzcd}
    (c' \downarrow \mathcal{F})
    \arrow[r, "\mathcal{U}"]
    & \mathsf{C}
    \arrow[r, "\mathcal{D}"]
    & \mathsf{D}
  \end{tikzcd}
\end{equation*}

To each such $c' \in \Obj(\mathsf{C}')$, there is an associated cone $\xi$ over $\evl{\mathcal{D}}_{(c' \downarrow \mathcal{F})}$, constructed as follows.

\begin{itemize}
  \item The tip of the cone is $\mathcal{G}(c')$.

  \item For each $(\alpha, f) \in \Obj(c' \downarrow \mathcal{F})$, the morphism $\xi_{(\alpha, f)}\colon\mathcal{D}(c') \to \evl{\mathcal{D}}_{(c' \downarrow \mathcal{F})}(\alpha, f) = \mathcal{D}(\alpha)$ is given by the composition
    \begin{equation*}
      \begin{tikzcd}
        \mathcal{G}(c')
        \arrow[r, "\mathcal{G}(f)"]
        & \mathcal{G}(\mathcal{F}(\alpha))
        \arrow[r, "\eta_{\alpha}"]
        & \mathcal{D}(\alpha)
      \end{tikzcd}.
    \end{equation*}
\end{itemize}

Dually, considering now a \emph{left} extension $(\mathcal{G}, \eta)$ of $\mathcal{D}$ along $\mathcal{F}$, we get a cone as follows.
\begin{itemize}
  \item The tip is $\mathcal{G}(c')$

  \item For each $(\alpha, f) \in \Obj(\mathcal{F} \downarrow c')$, the morphism $\xi_{\alpha, f}\colon \mathcal{D}(\alpha) \to \mathcal{D}(c')$ is given by the composition
    \begin{equation*}
      \begin{tikzcd}
        \mathcal{D}(\alpha)
        \arrow[r, "\eta_{\alpha}"]
        & (\mathcal{G} \circ \mathcal{F})(\alpha)
        \arrow[r, "\mathcal{G}(f)"]
        & \mathcal{G}(c')
      \end{tikzcd}
    \end{equation*}
\end{itemize}

\begin{lemma}
  \label{lemma:right_extension_induces_cone}
  The cone described above really is a cone. That is, for each morphism $g\colon (\alpha, f) \to (\alpha', f')$ as follows
  \begin{equation*}
    \begin{tikzcd}[column sep=tiny]
      & c'
      \arrow[ld, swap, "f"]
      \arrow[rd, "f'"]
      \\
      \mathcal{F}(\alpha)
      \arrow[rr, swap, "\mathcal{F}(g)"]
      && \mathcal{F}(\alpha')
    \end{tikzcd}
  \end{equation*}
  the diagram
  \begin{equation*}
    \begin{tikzcd}[column sep=tiny]
      & \mathcal{G}(c')
      \arrow[dl, swap, "\xi_{(\alpha, f)}"]
      \arrow[dr, "\xi_{(\alpha', f')}"]
      \\
      \mathcal{D}(\alpha)
      \arrow[rr, swap, "\mathcal{D}(g)"]
      && \mathcal{D}(\alpha')
    \end{tikzcd}
  \end{equation*}
  commutes.
\end{lemma}

\begin{proof}
  Consider the following diagram.
  \begin{equation*}
    \begin{tikzcd}[column sep=tiny]
      & \mathcal{G}(c')
      \arrow[ld, swap, "\mathcal{G}(f)"]
      \arrow[rd, "\mathcal{G}(f')"]
      \\
      (\mathcal{G} \circ \mathcal{F})(\alpha)
      \arrow[rr, swap, "\mathcal{G} \circ \mathcal{F}(\alpha)"]
      \arrow[d, swap, "\eta_{\alpha}"]
      && \mathcal{G} \circ \mathcal{F}(\alpha')
      \arrow[d, "\eta_{\alpha'}"]
      \\
      \mathcal{D}(\alpha)
      \arrow[rr, swap, "\mathcal{D}(f)"]
      &&\mathcal{D}(\alpha')
    \end{tikzcd}
  \end{equation*}
  The upper triangle is the functor $\mathcal{D}$ applied to the definition of commutativity in a category of morphisms, and the lower square is the naturality square for $\eta$.
\end{proof}

\begin{example}
  In the case that $\mathsf{C}'$ is a discrete groupoid, we have the following.

  \begin{itemize}
    \item The category $\mathsf{C}_{c'}$ is equivalent (in fact isomorphic) to both of the categories $(\mathcal{F} \downarrow c')$ and $(c' \downarrow \mathcal{F})$.

    \item The cone which constructed above over $\evl{\mathcal{D}}_{(c' \downarrow \mathcal{F})}$ agrees with the cones $\eta^{i}$ from \hyperref[eg:kan_extension_along_functor_to_discrete_groupoid]{Example~\ref*{eg:kan_extension_along_functor_to_discrete_groupoid}}.

    \item The cocone under $\evl{\mathcal{D}}_{(\mathcal{F} \downarrow c')}$ agrees with the cocones that we would have constructed if we had more time.
  \end{itemize}

  That is, the (co)cones we have constructed are a more general case of those we got from considering discrete groupoids.
\end{example}

\begin{lemma}
  \label{lemma:morphism_induces_functor_between_comma_categories}
  Let $\mathcal{F}\colon \mathsf{I} \to \mathsf{J}$ be a functor. Let $f\colon j \to j'$ be a morphism in $\mathsf{J}$. Then $\mathcal{F}$ induces the following functors.
  \begin{itemize}
    \item $(\mathcal{F} \downarrow j) \to (\mathcal{F} \downarrow j')$.

    \item $(j' \downarrow \mathcal{F}) \to (j \downarrow \mathcal{F})$.
  \end{itemize}
\end{lemma}
\begin{proof}
  Obvious.
\end{proof}

We saw above that given any extension $(\mathcal{G}, \eta)$ and any object $c' \in \Obj(\mathsf{C}')$, we could construct a cone. We hoped that this cone would hope turn out to be the analog of the cones over $\mathsf{C}_{i}$ from \hyperref[eg:kan_extension_along_functor_to_discrete_groupoid]{Example~\ref*{eg:kan_extension_along_functor_to_discrete_groupoid}}. This turns out to be the case.

\begin{theorem}
  Consider categories and functors as follows,
  \begin{equation*}
    \begin{tikzcd}
      \mathsf{C}
      \arrow[dr, swap, "\mathcal{F}"]
      \arrow[rr, "\mathcal{D}"]
      && \mathsf{D}
      \\
      & \mathsf{C}'
    \end{tikzcd}
  \end{equation*}
  where $\mathsf{C}$ is small and $\mathsf{D}$ has all small colimits. Then the Kan extension $\mathcal{F}_{!}\mathcal{D}$ exists and has values
  \begin{equation*}
    (\mathcal{F}_{!}\mathcal{D})(c') = \lim_{\rightarrow} \left( \begin{tikzcd} (\mathcal{F} \downarrow c') \arrow[r] & \mathsf{C} \arrow[r, "\mathcal{D}"] & \mathsf{D} \end{tikzcd} \right)
  \end{equation*}
\end{theorem}
\begin{proof}
  We first have to define how $\mathcal{F}_{!}\mathcal{D}$ behaves on morphisms. To this end, let $f\colon c' \to c''$. Then by \hyperref[lemma:morphism_induces_functor_between_comma_categories]{Lemma~\ref*{lemma:morphism_induces_functor_between_comma_categories}}, $f$ induces a
\end{proof}

\begin{lemma}
  \label{lemma:extension_whose_cones_are_limits_is_kan}
  Let $(\mathcal{G}, \eta)$ be a right extension as below.
  \begin{equation*}
    \begin{tikzcd}
      \mathsf{C}
      \arrow[dr, swap, "\mathcal{F}"]
      \arrow[rr, "\mathcal{D}", ""'{name=U}]
      && \mathsf{D}
      \\
      & \mathsf{C'}
      \arrow[ur, swap, "\mathcal{G}"]
      \arrow[d, to=U, Rightarrow, swap, "\eta"]
    \end{tikzcd}
  \end{equation*}
  If for every $c' \in \Obj(\mathsf{C'})$ the cone with tip $\mathcal{G}(c')$ over the functor given by the composition
  \begin{equation*}
    (c' \downarrow \mathcal{F}) \to \mathsf{C} \to \mathsf{D}
  \end{equation*}
  with sides
  \begin{equation*}
    \xi_{(\alpha, f)} = \left( \begin{tikzcd} \mathcal{G}(c') \arrow[r, "\mathcal{G}(f)"] & (\mathcal{G} \circ \mathcal{F})(\alpha) \arrow[r, "\eta_{\alpha}"] & \mathcal{D}(\alpha) \end{tikzcd} \right)
  \end{equation*}

  is a limit cone, then $(\mathcal{G}, \eta)$ is a right Kan extension.
\end{lemma}
\begin{proof}
  Let $(\mathcal{G}', \eta')$ be any other extension. We need to show that there is a unique natural transformation $\chi$ as follows.
  \begin{equation*}
    \begin{tikzcd}
      \mathcal{G}' \circ \mathcal{F}
      && \mathcal{G} \circ \mathcal{F}
    \end{tikzcd}
  \end{equation*}
\end{proof}

This is a


\end{document}
