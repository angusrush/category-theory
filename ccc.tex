\documentclass[main.tex]{subfiles}

\begin{document}

\chapter{Cartesian closed categories}

\section{Cartesian closed categories}\label{sec:cartesianclosedcategories}

This section loosely follows~\cite{awodey-category-theory-foundations-videos}.

One of the limiting aspects of category theory is that it is very difficult to talk about specifics: one is not allowed to talk about the elements of sets, or the 

\section{Products}

We saw in \hyperref[eg:universalpropertyofproducts]{Example~\ref*{eg:universalpropertyofproducts}} the definition for a categorical product.

\begin{definition}[categorical product]
  \label{def:categoricalproduct}
  Let $\mathsf{C}$ be a category, and $A$ and $B$ in $\Obj(\mathsf{C})$. A product of $A$ and $B$ is an object $X$ (often denoted $A\times B$) together with a pair of morphisms $\pi_{1}\colon X \to A$ and $\pi_{2}\colon X \to B$ such that for every object $Y$ and pair of morphisms $f_{1}\colon Y \to A$, $f_{2}\colon Y \to B$, there exists a unique morphism $f\colon Y \to A \times B$ such that the following diagram commutes.
  \begin{equation*}
    \begin{tikzcd}
      & Y
      \arrow[dl, swap, "f_{1}"]
      \arrow[d, "f"]
      \arrow[dr, "f_{2}"]
      &
      \\
      A
      & X
      \arrow[l, "\pi_{1}"]
      \arrow[r, swap, "\pi_{2}"]
      & B
    \end{tikzcd}
  \end{equation*}
\end{definition}

\begin{example}
  In the category $\mathsf{Set}$, the categorical product is the Cartesian product. More specifically:
  \begin{itemize}
    \item For any two sets $A$ and $B$, the product object $X$ is $A \times B$
    \item The projectios $\pi_{1}$ and $\pi_{2}$ are defined by
      \begin{equation*}
        \pi_{1}(a, b) = a,\qquad \pi_{2}(a, b) = b.
      \end{equation*}
  \end{itemize}
  To check this, we must show that $X$, together with $\pi_{1}$ and $\pi_{2}$ satisfy the universal property for products.

  Let $Y$ be any set and choose functions
  \begin{equation*}
    f_{1}\colon Y \to A,\qquad \text{and}\qquad f_{2}\colon Y \to B.
  \end{equation*}
\end{example}

In some categories, we can take the product of any two objects; one generally says that such a category \emph{has products}.

\begin{definition}[category with products]
  \label{def:categorywithproducts}
  Let $\mathsf{C}$ be a category such that for every two objects $A$, $B \in \Obj(\mathsf{C})$, there exists an object $A \times B$ which satisfies the universal property (\hyperref[eg:universalpropertyofproducts]{Example~\ref*{eg:universalpropertyofproducts}}). Then we say that $\mathsf{C}$ \defn{has products}.
\end{definition}
\begin{note}
  Sometimes people call a category with products a \emph{Cartesian category}, but others use this terminology to mean a category with all small limits.\footnote{We will see later that any category with both products and equalizers has all finite limits.} We will avoid it altogether.
\end{note}

Recall that in \hyperref[eg:cartesian_product_of_sets_is_bifunctor]{Example~\ref*{eg:cartesian_product_of_sets_is_bifunctor}} we showed that the Cartesian product of two sets is functorial. The same turns out to be true for the categorical product.
\begin{theorem}
  \label{thm:productisafunctor}
  Let $\mathsf{C}$ be a category with products. Then the product can be extended to a bifunctor (\hyperref[def:bifunctor]{Definition~\ref*{def:bifunctor}}) $\mathsf{C} \times \mathsf{C} \to \mathsf{C}$.
\end{theorem}
\begin{proof}
  Let $X$, $Y \in \Obj(C)$. We need to check that the assignment $(X,Y) \mapsto \times(X,Y) \equiv X \times Y$ is functorial, i.e.\ that $\times$ assigns
  \begin{itemize}
    \item to each pair $(X,Y) \in \Obj(\mathsf{C}\times\mathsf{C})$ an object $X\times Y \in \Obj(\mathsf{C})$ and
    \item to each pair of morphisms $(f,g)\colon (X,Y) \to (X',Y')$ a morphism
      \begin{equation*}
        \times(f,g) = f \times g\colon X\times Y \to X' \times Y'
      \end{equation*}
      such that
      \begin{itemize}
        \item $\id_{X} \times \id_{Y} = \id_{X \times Y}$, and
        \item $\times$ respects composition as follows.
          \begin{equation*}
            \begin{tikzcd}[column sep=large, row sep=small]
              {(X,Y)} \arrow[r, "{(f,g)}"] \arrow[rr, bend left, "{(f',g') \circ (f,g) = (f'\circ f, g'\circ g)}"] & {(X',Y')} \arrow[r, "{(f',g')}"] &  {(X'',Y'')} \\
              & \,\arrow[dd, rightarrow, "\times"] & \\
              & \, & \\
              & \, & \\
              X\times Y \arrow[r, "f\times g"] \arrow[rr, bend right, swap, "f'\times g' \circ f \times g = (f'\circ f)\times(g'\circ g)"] & X' \times Y' \arrow[r, "f'\times g'"]& X''\times Y''
            \end{tikzcd}
          \end{equation*}
      \end{itemize}
  \end{itemize}
  We know how $\times$ assigns objects in $\mathsf{C} \times \mathsf{C}$ to objects in $\mathsf{C}$. We need to figure out how $\times$ should assign to a morphism $(f,g)$ in $\mathsf{C} \times \mathsf{C}$ a morphism $f\times g$ in $\mathsf{C}$. We do this by diagram chasing.

  Suppose we are given two maps $f\colon X_{1} \to X_{2}$ and $g\colon Y_{1} \to Y_{2}$. We can view this as a morphism $(f,g)$ in $\mathsf{C} \times \mathsf{C}$.

  Recall the universal property for products: a product $X_{1}\times Y_{1}$ is a final object in the category $(\Delta \downarrow (X_{1}, Y_{1}))$. Objects in this category can be thought of as diagrams in $\mathsf{C} \times \mathsf{C}$.
  \begin{equation*}
    \begin{tikzcd}[column sep=huge]
      {(X_{1}\times Y_{1}, X_{1}\times Y_{1})} \arrow[r, "{\pi_{1}, \pi_{2}}"] & {(X_{1}, Y_{1})}
    \end{tikzcd}.
  \end{equation*}
  By assumption, we can take the product of both $X_{1}$ and $Y_{1}$, and $X_{2}$ and $Y_{2}$. This gives us two diagrams living in $\mathsf{C} \times \mathsf{C}$, which we can put next to each other.
  \begin{equation*}
    \begin{tikzcd}[column sep=huge]
      {(X_{1} \times Y_{1}, X_{1} \times Y_{1})} \arrow[r, "{(\pi_{1}, \pi_{2})}"] & (X_{1}, Y_{1}) \\
      {(X_{2} \times Y_{2}, X_{2} \times Y_{2})} \arrow[r, "{(\rho_{1}, \rho_{2})}"] & (X_{2}, Y_{2}) \\
    \end{tikzcd}
  \end{equation*}
  We can draw in our morphism $(f,g)$, and take its composition with $(\pi_{1}, \pi_{2})$.
  \begin{equation*}
    \begin{tikzcd}[row sep=huge, column sep=huge]
      {(X_{1} \times Y_{1}, X_{1} \times Y_{1})} \arrow[r, "{(\pi_{1}, \pi_{2})}"] \arrow[dr, swap, "{(f\circ \pi_{1}, g \circ \pi_{2})}"] & (X_{1}, Y_{1}) \arrow[d, "{(f,g)}"] \\
      {(X_{2} \times Y_{2}, X_{2} \times Y_{2})} \arrow[r, swap, "{(\rho_{1}, \rho_{2})}"] & (X_{2}, Y_{2})
    \end{tikzcd}
  \end{equation*}
  Now forget about the top right of the diagram. I'll erase it to make this easier.
  \begin{equation*}
    \begin{tikzcd}[row sep=huge, column sep=huge]
      {(X_{1} \times Y_{1}, X_{1} \times Y_{1})}  \arrow[dr, "{(f\circ \pi_{1}, g \circ \pi_{2})}"] & \\
      {(X_{2} \times Y_{2}, X_{2} \times Y_{2})} \arrow[r, swap, "{(\rho_{1}, \rho_{2})}"] & (X_{2}, Y_{2})
    \end{tikzcd}
  \end{equation*}
  The universal property for products says that there exists a unique map $h\colon X_{1}\times Y_{1} \to X_{2}\times Y_{2}$ such that the diagram below commutes.
  \begin{equation*}
    \begin{tikzcd}[row sep=huge, column sep=huge]
      (X_{1} \times Y_{1}, X_{1} \times Y_{1}) \arrow[d, swap, "{(h, h)}"] \arrow[rd, "{(f \circ \pi_{1}, g \circ \pi_{2})}"] & & X_{1}\times Y_{1} \arrow[d, "\exists!h"]\\
      (X_{2}\times Y_{2}, X_{2}\times Y_{2}) \arrow[r, swap, "{(\rho_{1}, \rho_{2})}"] &  (X_{1}, X_{2})& X_{2}\times Y_{2}
    \end{tikzcd}
  \end{equation*}
  And $h$ is what we will use for the product $f \times g$.

  Of course, we must also check that $h$ behaves appropriately. Draw two copies of the diagram for the terminal object in $(\Delta\downarrow (X,Y))$ and identity arrows between them.
  \begin{equation*}
    \begin{tikzcd}[row sep=huge, column sep=huge]
      {(X\times Y, X \times Y)} \arrow[r, "{(\pi_{1}, \pi_{2})}"] \arrow[d, swap, "{(\id_{X \times Y}, \id_{X \times Y})}"] & (X, Y) \arrow[d, "{(\id_{X}, \id_{Y})}"] \\
      {(X\times Y, X \times Y)} \arrow[r, swap, "{(\pi_{1}, \pi_{2})}"] & (X, Y)
    \end{tikzcd}
  \end{equation*}
  Just as before, we compose the morphisms to and from the top right to draw a diagonal arrow, and then erase the top right object and the arrows to and from it.
  \begin{equation*}
    \begin{tikzcd}[row sep=huge, column sep=huge]
      {(X\times Y, X \times Y)}  \arrow[d, swap, "{(\id_{X \times Y},\, \id_{X \times Y}) = (\id_{X}\times \id_{Y},\, \id_{X}\times \id_{Y})}"] \arrow[dr, "{(\id_{X} \circ \pi_{1},\, \id_{Y} \circ \pi_{2})}"] & & X\times Y \arrow[d, "\id_{X}\times\id_{Y} = \id_{X}\times\id_{Y}"] \\
      {(X\times Y, X \times Y)} \arrow[r, swap, "{(\pi_{1}, \pi_{2})}"] & (X, Y) & X\times Y
    \end{tikzcd}
  \end{equation*}
  By definition, the arrow on the right is $\id_{X}\times\id_{Y}$. It is also $\id_{X\times Y}$, and by the universal property it is unique. Therefore $\id_{X\times Y} = \id_{X}\times \id_{Y}$.

  Next, put three of these objects together. The proof of the last part is immediate.
  \begin{equation*}
    \begin{tikzcd}[column sep=huge, row sep=huge]
      {(X_{1} \times Y_{1}, X_{1} \times Y_{1})}  \arrow[d, "{(f\times g, f\times g)}"] \arrow[dd, bend right=70, swap, "{(f'\times g') \circ (f\times g)} = (g' \circ g) \times (f' \circ f)"]  & (X_{1}, Y_{1}) \arrow[d, swap, "{(f,g)}"] \arrow[dd, bend left=70, "{(g'\circ g, f' \circ f)}"] \\
      {(X_{2} \times Y_{2}, X_{2} \times Y_{2})}\arrow[d, "{(f'\times g', f'\times g')}"]  & (X_{2}, Y_{2}) \arrow[d, swap, "{(f',g')}"] \\
      {(X_{3} \times Y_{3}, X_{3} \times Y_{3})}  & (X_{3}, Y_{3})
    \end{tikzcd}
  \end{equation*}
\end{proof}
The meaning of this theorem is that in any category where you can take products of the objects, you can also take products of the morphisms.

\begin{example}[diagonal functor]
  \label{eg:diagonalfunctor}
  Let $\mathsf{C}$ be a category, $\mathsf{C} \times \mathsf{C}$ the product category of $\mathsf{C}$ with itself. The \defn{diagonal functor} $\Delta\colon \mathsf{C} \rightarrow \mathsf{C} \times \mathsf{C}$ is the functor which sends
  \begin{itemize}
    \item each object $A \in \Obj(\mathsf{C})$ to the pair $(A,A) \in \Obj(\mathsf{C} \times \mathsf{C})$, and
    \item each morphism $f\colon A \to B$ to the ordered pair
      \begin{equation*}
        (f,f) \in \Hom_{\mathsf{C}\times\mathsf{C}}(A\times B,A \times B).
      \end{equation*}
  \end{itemize}
\end{example}

\begin{example}
  The category $\mathsf{Vect}_{k}$ of vector spaces over a field $k$ has the direct sum $\oplus$ as a product.
\end{example}

The product is, in an appropriate way, commutative.
\begin{lemma}
  There is a natural isomorphism between the following functors $\mathsf{C} \times \mathsf{C} \to \mathsf{C}$
  \begin{equation*}
    \times\colon (A, B) \to A \times B\qquad\text{and}\qquad \tilde{\times}\colon (A, B) \to B \times A.
  \end{equation*}
\end{lemma}
\begin{proof}
  We define a natural transformation $\Phi\colon \times \Rightarrow \tilde{\times}$ with components
  \begin{equation*}
    \Phi_{A, B}\colon A \times B \to B \times A
  \end{equation*}
  as follows. Denote the canonical projections for the product $A \times B$ by $\pi_{A}$ and $\pi_{B}$. Then $(\pi_{B}, \pi_{A})$ is a map $A \times B \to (B, A)$, and the universal property for products gives us a map $A \times B \to B \times A$. We can pull the same trick to go from $B \times A$ to $A \times B$, using the pair $(\pi_{A}, \pi_{B})$ and the universal property. Furthermore, these maps are inverse to each other, so $\Phi_{A, B}$ is an isomorphism.

  We need only check naturality, i.e.\ that for $f\colon A \to A'$, $g\colon B \to B'$, the following square commutes.
  \begin{equation*}
    \begin{tikzcd}
      A \times B
      \arrow[r]
      \arrow[d, swap]
      & A' \times B'
      \arrow[d]
      \\
      B \times A
      \arrow[r]
      & B' \times A'
    \end{tikzcd}
  \end{equation*}
\end{proof}

\begin{note}
  It is an important fact that the product is associative. We shall see this in \hyperref[thm:categoricalproductisassociative]{Theorem~\ref*{thm:categoricalproductisassociative}}, after we have developed the machinery to do it cleanly.
\end{note}


\section{Coproducts}

\begin{definition}
  Let $\mathsf{C}$ be a category, $X_{1}$ and $X_{2} \in \Obj(\mathsf{C})$. The \defn{coproduct} of $X_{1}$ and $X_{2}$, denoted $X_{1} \amalg X_{2}$, is the initial object in the category $((X_{1},X_{2})\downarrow \Delta)$. In everyday language, we have the following.
  \begin{quote}
    An object $X_{1} \amalg X_{2}$ is called the coproduct of $X_{1}$ and $X_{2}$ if
    \begin{enumerate}
      \item there exist morphisms $i_{1}\colon X_{1} \to X_{1} \amalg X_{2}$ and $i_{2}\colon X_{2} \to X_{1} \amalg X_{2}$ called \emph{canonical injections} such that

      \item for any object $Y$ and morphisms $f_{1}\colon X_{1} \to Y$ and $f_{2}\colon X_{2} \to Y$ there exists a unique morphism $f\colon X_{1} \amalg X_{2} \to Y$ such that $f_{1} = f \circ i_{1}$ and $f = f_{2} = f \circ i_{2}$, i.e.\ the following diagram commutes.
        \begin{equation*}
          \begin{tikzcd}
            & Y & \\
            X_{1} \arrow[ur, "f_{1}"] \arrow[r, swap, "i_{1}"] & X_{1} \arrow[u, "f"] \amalg X_{2} & X_{2} \arrow[l, "i_{2}"] \arrow[ul, swap, "f_{2}"]
          \end{tikzcd}
        \end{equation*}
    \end{enumerate}
  \end{quote}
\end{definition}

\begin{definition}[category with coproducts]
  \label{def:categorywithcoproducts}
  We say that a category $\mathsf{C}$ \defn{has coproducts} if for all $A$, $B \in \Obj(C)$, the coproduct $A \amalg B$ is in $\Obj(C)$.
\end{definition}

We have the following analog of \hyperref[thm:productisafunctor]{Theorem~\ref*{thm:productisafunctor}}.
\begin{theorem}
  \label{thm:coproductisafunctor}
  Let $C$ be a category with coproducts. Then we have a functor $\amalg\colon C \times C \rightarrow C$.
\end{theorem}
\begin{proof}
  We verify that $\amalg$ allows us to define canonically a coproduct of morphisms. Let $X_{1}$, $X_{2}$, $Y_{1}$, $Y_{2} \in \Obj(C)$, and let $f\colon X_{1} \to Y_{1}$ and $g\colon X_{2} \to Y_{2}$. We have the following diagram.
  \begin{equation*}
    \begin{tikzcd}[row sep=huge, column sep=huge]
      X_{1} \arrow[d, swap, "f"] \arrow[dr, "{i_{i,Y} \circ f}"] \arrow[r, "{i_{1, X}}"]& X_{1} \amalg X_{2} & X_{2} \arrow[dl, swap, "{i_{2, Y} \circ g}"] \arrow[ d, "g"] \arrow[l, swap, "{i_{2, X}}"] \\
      Y_{1} \arrow[r, "{i_{1, Y}}"]& Y_{1} \amalg Y_{2} & Y_{2} \arrow[l, swap, "{i_{2, Y}}"]
    \end{tikzcd}
  \end{equation*}
  But by the universal property of coproducts, the diagonal morphisms induce a map $X_{1} \amalg X_{2} \to Y_{1} \amalg Y_{2}$, which we define to be $f \amalg g$.
  \begin{equation*}
    \begin{tikzcd}[row sep=huge, column sep=huge]
      X_{1} \arrow[dr, swap, "{i_{i,Y} \circ f}"] \arrow[r, "{i_{1, X}}"]& X_{1} \arrow[d, "f \amalg g"] \amalg X_{2} & X_{2} \arrow[dl, "{i_{2, Y} \circ g}"] \arrow[l, swap, "{i_{2, X}}"] \\
      & Y_{1} \amalg Y_{2} &
    \end{tikzcd}
  \end{equation*}
  The rest of the verification that $\amalg$ really is a functor is identical to that in \hyperref[thm:productisafunctor]{Theorem~\ref*{thm:productisafunctor}}.
\end{proof}

\begin{example}
  In $\mathsf{Set}$, the coproduct is the disjoint union.
\end{example}

\begin{example}
  In $\mathsf{Vect}_{k}$, the coproduct $\oplus$ is the direct sum.
\end{example}

\begin{example}
  In $k\mhyp\mathsf{Alg}$ the tensor product is the coproduct. To see this, let $A$, $B$, and $R$ be $k$-algebras. The tensor product $A \otimes_{k} B$ has canonical injections $\iota_{1}\colon A \to A \otimes B$ and $\iota_{2}\colon B \to A \otimes B$ given by
  \begin{equation*}
    \iota_{A}\colon v \to v \otimes \id_{B}\qquad\text{and}\qquad \iota_{B}\colon v \mapsto \id_{A} \otimes v.
  \end{equation*}

  Let $f_{1}\colon A \to R$ and $f_{2}\colon B \to R$ be $k$-algebra homomorphisms. Then there is a unique homomorphism $f\colon A \otimes B \to R$ which makes the following diagram commute.
  \begin{equation*}
    \begin{tikzcd}
      A
      \arrow[r, "\iota_{A}"]
      \arrow[dr, swap, "f_{1}"]
      & A \otimes B
      \arrow[d, "f"]
      & B
      \arrow[l, swap, "\iota_{B}"]
      \arrow[dl, "f_{2}"]
      \\
      & R
    \end{tikzcd}
  \end{equation*}
\end{example}


\section{Biproducts}\label{sec:biproducts}

A lot of the stuff in this section was adapted and expanded from~\cite{annoying-precision-meditation}.

One often hears that a biproduct is an object which is both a product and a coproduct, but this is both misleading and unnecessarily vague. Objects satisfying universal properties are only defined up to isomorphism, so it doesn't make much sense to demand that products \emph{coincide} with coproducts. However, demanding only that they be isomorphic (or even naturally isomorphic) is too weak, and does not determine a biproduct uniquely. In this section we will give the correct definition which encapsulates all of the properties of the product and the coproduct we know and love.

In order to talk about biproducts, we need to be aware of the following result, which we will consider in much more detail in \hyperref[sec:preabeliancategories]{Definition~\ref*{sec:preabeliancategories}}.

\begin{definition}[zero morphism]
  Let $\mathsf{C}$ be a category with zero object $0$. For any two objects $A$, $B \in \Obj(\mathsf{C})$, the \defn{zero morphism} $0_{A,B}$ is the unique morphism $A \to B$ which factors through $0$.
  \begin{equation*}
    \begin{tikzcd}
      A
      \arrow[rr, bend left, "0_{A, B}"]
      \arrow[r]
      & 0
      \arrow[r]
      & B
    \end{tikzcd}
  \end{equation*}
\end{definition}

\begin{notation}
  It will often be clear what the source and destination of the zero morphism are; in this case we will drop the subscripts, writing $0$ instead of $0_{AB}$. With this notation, for all morphisms $f$ and $g$ we have $f \circ 0 = 0$ and $0 \circ g = 0$.
\end{notation}

\begin{definition}[category with biproducts]
  \label{def:categorywithbiproducts}
  Let $\mathsf{C}$ be a category with zero object $0$, products $\times$ (with canonical projections $\pi$), and coproducts $\amalg$ (with natural injections $\iota$). We say that $\mathsf{C}$ \defn{has biproducts} if for each two objects $A_{1}$, $A_{2} \in \Obj(\mathsf{C})$, there exists an isomorphism
  \begin{equation*}
    \Phi_{A_{1}, A_{2}}\colon A_{1} \amalg A_{2} \to A_{1} \times A_{2}
  \end{equation*}
  such that
  \begin{equation*}
    \begin{tikzcd}[column sep=large]
      A_{i}
      \arrow[r, "\iota_{i}"]
      & A_{1} \amalg A_{2}
      \arrow[r, "{\Phi_{A_{1}, A_{2}}}"]
      & A_{1} \times A_{2}
      \arrow[r, "\pi_{j}"]
      & A_{j}
    \end{tikzcd}
    =
    \begin{cases}
      \id_{A_{1}}, & i = j \\
      0, & i \neq j.
    \end{cases}
  \end{equation*}
  Here is a picture.
  \begin{equation*}
    \begin{tikzcd}
      & A_{1} \amalg A_{2}
      \arrow[dd, "\Phi_{A_{1},A_{2}}"]
      \\
      A_{1}
      \arrow[ur, "\iota_{1}"]
      & & A_{2}
      \arrow[ul, swap, "\iota_{2}"]
      \\
      & A_{1} \times A_{2}
      \arrow[ul, "\pi_{1}"]
      \arrow[ur, swap, "\pi_{2}"]
    \end{tikzcd}
  \end{equation*}
\end{definition}

\begin{lemma}
  The isomorphisms $\Phi_{A, B}$ are unique if they exist.
\end{lemma}
\begin{proof}
  We can compose $\Phi_{A_{1}, A_{2}}$ with $\pi_{1}$ and $\pi_{2}$ to get maps $A_{1} \amalg A_{2} \to A_{1}$ and $A_{1} \amalg A_{2} \to A_{2}$. The universal property for products tells us that there is a unique map $\varphi\colon A_{1} \amalg A_{2} \to A_{1} \times A_{2}$ such that $\pi_{1} \circ \varphi = \pi_{1} \circ \Phi_{A_{1}, A_{2}}$ and $\pi_{2} \circ \varphi = \pi_{2} \circ \Phi_{A_{1}, A_{2}}$; but $\varphi$ is just $\Phi_{A_{1}, A_{2}}$. Hence $\Phi_{A_{1}, A_{2}}$ is unique.
\end{proof}

\begin{lemma}
  \label{lemma:mapcoprodtoproddeterminedbycomponents}
  In any category $\mathsf{C}$ with biproducts $\Phi_{A, B}\colon A \amalg B \to A \times B$, any map $A \amalg B \to A' \times B'$ is completely determined by its components $A \to A'$, $A \to B'$, $B \to A'$, and $B \to B'$. That is, given any map $f\colon A \amalg B \to A' \times B'$, we can create four maps
  \begin{equation*}
    \pi_{A'} \circ f \circ \iota_{A}\colon A \to A', \qquad \pi_{B'} \circ f \circ \iota_{A}\colon A \to B',\qquad\text{etc},
  \end{equation*}
  and given four maps
  \begin{equation*}
    \psi_{A, A'}\colon A \to A';\qquad \psi_{A, B'}\colon A \to B';\qquad \psi_{B,A'}\colon B \to A',\qquad\text{and } \psi_{B,B'}\colon B \to B',
  \end{equation*}
  we can construct a unique map $\psi\colon A \amalg B \to A' \times B'$ such that
  \begin{equation*}
    \psi_{A, A'} = \pi_{A'} \circ \psi \circ \iota_{A},\qquad \psi_{A,B'} = \pi_{B'} \circ \psi \circ \iota_{A},\qquad\text{etc}.
  \end{equation*}
\end{lemma}
\begin{proof}
  The universal property for coproducts give allows us to turn the morphisms $\psi_{A,A'}$ and $\psi_{A, B'}$ into a map $\psi_{A}\colon A \to B \times B'$, the unique map such that
  \begin{equation*}
    \pi_{A'} \circ \psi_{A} = \psi_{A,A'} \qquad\text{and}\qquad\pi_{B'} \circ \psi_{A} = \psi_{A,B'}.
  \end{equation*}

  The morphisms $\psi_{B,A'}$ and $\psi_{B,B'}$ give us a map $\psi_{B}$ which is unique in the same way.

  Now the universal property for coproducts gives us a map $\psi\colon A \amalg B \to A' \times B'$, the unique map such that $\psi \circ \iota_{A} = \psi_{A}$ and $\psi \circ \iota_{B} = \psi_{B}$.
\end{proof}

\begin{theorem}
  The $\Phi_{A, B}$ defined above form the components of a natural isomorphism.
\end{theorem}
\begin{proof}
  Let $A$, $B$, $A'$, and $B' \in \Obj(\mathsf{C})$, and let $f\colon A \to A'$ and $g\colon B \to B'$. To check that $\Phi_{A, B}$ are the components of a natural isomorphism, we have to check that the following naturality square commutes.
  \begin{equation*}
    \begin{tikzcd}[column sep=large, row sep=large]
      A \amalg B
      \arrow[r, "f \amalg g"]
      \arrow[d, swap, "{\Phi_{A, B}}"]
      & A' \amalg B'
      \arrow[d, "{\Phi_{A', B'}}"]
      \\
      A \times B
      \arrow[r, swap, "f \times g"]
      & A' \times B'
    \end{tikzcd}
  \end{equation*}
  That is, we need to check that $(f \times g) \circ \Phi_{A, B} = \Phi_{A', B'} \circ (f \amalg g)$.

  Both of these are morphisms $A \amalg B \to A' \times B'$, and by \hyperref[lemma:mapcoprodtoproddeterminedbycomponents]{Lemma~\ref*{lemma:mapcoprodtoproddeterminedbycomponents}}, it suffices to check that their components agree, i.e.\ that
\end{proof}


\begin{example}
  In the category $\mathsf{Vect}_{k}$ of vector spaces over a field $k$, the direct sum $\oplus$ is a biproduct.
\end{example}

\begin{example}
  In the category $\mathsf{Ab}$ of abelian groups, the direct sum $\oplus$ is both a product and a coproduct. Hence $\mathsf{Ab}$ has biproducts.
\end{example}


\section{Exponentials}

Astonishingly, we can talk about functional evaluation purely in terms of products and universal properties.

\begin{definition}[exponential]
  \label{def:exponential}
  Let $\mathsf{C}$ be a category with products, $A$, $B \in \Obj(\mathsf{C})$. The \defn{exponential} of $A$ and $B$ is an object $B^{A} \in \Obj(\mathsf{C})$ together with a morphism $\varepsilon\colon B^{A} \times A \to B$, called the \emph{evaluation morphism}, which satisfies the following universal property.

  \begin{quote}
    For any object $X \in \Obj(\mathsf{C})$ and morphism $f\colon X \times A \to B$, there exists a unique morphism $\bar{f}\colon X \to B^{A}$ which makes the following diagram commute.
    \begin{equation*}
      \begin{tikzcd}
        B^{A}
        & & B^{A} \times A
        \arrow[r, "\varepsilon"]
        & B
        \\
        X
        \arrow[u, "\exists!\bar{f}"]
        & & X \times A
        \arrow[ur, swap, "f"]
        \arrow[u, "\bar{f} \times \id_{A}"]
      \end{tikzcd}
    \end{equation*}
  \end{quote}
\end{definition}

\begin{example}
  In $\mathsf{Set}$, the exponential object $B^{A}$ is the set of functions $A \to B$, and the evaluation morphism $\varepsilon$ is the map which assigns $(f, a) \mapsto f(a)$. Let us check that these objects indeed satisfy the universal property given.

  Suppose we are given a function $f\colon X \times A \to B$. We want to construct from this a function $\bar{f}\colon X \to B^{A}$, i.e.\ a function which takes an element $x \in X$ and returns a function $\bar{f}(x)\colon A \to B$. There is a natural way to do this: fill the first slot of $f$ with $x$! That is to say, define $\bar{f}(x) = f(x, -)$. It is easy to see that this makes the diagram commute:
  \begin{equation*}
    \begin{tikzcd}
      {(f(x,-), a)}
      \arrow[r, mapsto, "\varepsilon"]
      &{f(x, a)}
      \\
      {(x, a)}
      \arrow[u, mapsto, "\bar{f} \times \id_{A}"]
      \arrow[ur, mapsto, swap, "f"]
    \end{tikzcd}
  \end{equation*}
  Furthermore, $\bar{f}$ is unique since if it sent $x$ to any function other than $f$ the diagram would not commute.
\end{example}

\begin{example}
  One might suspect that the above generalizes to all sorts of categories. For example, one might hope that in $\mathsf{Grp}$, the exponential $H^{G}$ of two groups $G$ and $H$ would somehow capture all homomorphisms $G \to H$. This turns out to be impossible; we will see why in REF\@.
\end{example}


\section{Cartesian closed categories}

\begin{definition}[cartesian closed category]
  \label{def:cartesianclosedcategory}
  A category $\mathsf{C}$ is \defn{cartesian closed} if it has
  \begin{enumerate}
    \item products: for all $A$, $B \in \Obj(\mathsf{C})$, $A \times B \in \Obj(\mathsf{C})$;
    \item exponentials: for all $A$, $B \in \Obj(\mathsf{C})$, $B^{A} \in \Obj(\mathsf{C})$;
    \item a terminal element (\hyperref[def:initialfinalzeroobject]{Definition~\ref*{def:initialfinalzeroobject}}) $1 \in \mathsf{C}$.
  \end{enumerate}
\end{definition}

Cartesian closed categories have many inveresting properties. For example, they replicate many properties of the integers: we have the following familiar formulae for any objects $A$, $B$, and $C$ in a Cartesian closed category with coproducts $+$.
\begin{enumerate}
  \item $A \times (B + C) \simeq (A \times B) + (A \times C)$
  \item $C^{A + B} \simeq C^{A} \times C^{B}$.
  \item ${(C^{A})}^{B} \simeq C^{A \times B}$
  \item ${(A \times B)}^{C} \simeq A^{C} \times B^{C}$
  \item $C^{1} \simeq C$
\end{enumerate}

We could prove these immediately, by writing down diagrams and appealing to the universal properties. However, we will soon have a powerful tool, the Yoneda lemma, that makes their proof completely routine.


\end{document}
