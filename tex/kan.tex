\documentclass[notes.tex]{subfiles}

\begin{document}

\chapter{Kan extensions}
\label{sec:kan_extensions}

\section{Motivation}
\label{sec:motivation}

Let $\mathsf{C}$ be a category, and $\mathcal{I}\colon \mathsf{S} \to \mathsf{C}$ a subcategory inclusion. Given a functor $\mathcal{F}$ from $\mathsf{C}$ to any category $\mathsf{D}$, we can restrict $\mathcal{F}$ to $\mathsf{S}$ by pulling it back with $\mathcal{I}$. We denote this restriction by
\begin{equation*}
  \mathcal{I}^{*}(\mathcal{F}) = \mathcal{F} \circ \mathcal{I}.
\end{equation*}

In fact, $\mathcal{I}$ induces a functor from the category of functors $\mathsf{C} \to \mathsf{D}$ (\hyperref[def:functorcategory]{Definition~\ref*{def:functorcategory}}) to the category of functors $\mathsf{S} \to \mathsf{D}$. Recall that we denoted these categories by $[\mathsf{C}, \mathsf{D}]$ and $[\mathsf{S}, \mathsf{D}]$ respectively.

Admittedly, restricting the domain of a functor is not a very good party trick, even if done in a functorial way. The more impressive thing would be to \emph{increase} the domain of a functor.

We have come up against this kind of problem before. For example, it is easy to turn a group into a set by forgetting the group structure. We saw that there was a canonical way to turn a set into a group by forming the free group over it. We mentioned that this was a reflection of a more general phenomenon known as \emph{free-forgetful adjunction} (\hyperref[note:freeforgetfuladjunctions]{Note~\ref*{note:freeforgetfuladjunctions}}). This suggests that the best possible solution to extending the domain of a functor is to search for a functor which is a left or right adjoint to the restriction functor $\mathcal{I}^{*}$.

Of course, it is also interesting to consider the case where $\mathcal{I}$ is not an inclusion. In fact, we will see that by choosing $\mathcal{I}$ in different ways leads to a generalization of many different categorical concepts, such as limits. Left and right adjoints will be known as \emph{left right Kan extensions.}

\section{Global Kan extensions}
\label{sec:global_kan_extensions}

The most high-level notion (although, as we will see, not the most useful) of a Kan extension is called a \emph{global Kan extension.}

\begin{definition}[pullback]
  \label{def:pullback}
  Let $\mathcal{F}\colon \mathsf{C} \to \mathsf{C}'$ be a functor. For any other category $\mathsf{D}$, $\mathcal{F}$ induces a functor, called the \defn{pullback by $\mathcal{F}$},
  \begin{equation*}
    \mathcal{F}^{*}\colon [\mathsf{C}', \mathsf{D}] \to [\mathsf{C}, \mathsf{D}]
  \end{equation*}
  which sends a functor $\mathcal{G}\colon \mathsf{C}' \to \mathsf{D}$ to its precomposition $\mathcal{G} \circ \mathcal{F}$
  \begin{equation*}
    \begin{tikzcd}
      \mathsf{C}'
      \arrow[r, "\mathcal{G}"]
      & \mathsf{D}
    \end{tikzcd}
    \qquad
    \mapsto
    \qquad
    \begin{tikzcd}
      \mathsf{C}
      \arrow[dr, swap, "\mathcal{F}"]
      \arrow[rr, "\mathcal{F}^{*}\mathcal{G}"]
      && \mathsf{D}
      \\
      & \mathsf{C'}
      \arrow[ur, swap, "\mathcal{G}"]
    \end{tikzcd}
  \end{equation*}
  and a natural transformation $\eta\colon \mathcal{G} \to \mathcal{G}'$ to its left whiskering (\hyperref[eg:whiskering]{Example~\ref*{eg:whiskering}}) $\eta\mathcal{F}$.
  \begin{equation*}
    \begin{tikzcd}[column sep=huge]
      \mathsf{C}
      \arrow[r, "\mathcal{F}"]
      &\mathsf{C}'
      \arrow[r, bend left, "\mathcal{G}"{name=U}]
      \arrow[r, bend right, swap, "\mathcal{G}'"{name=D}]
      & \mathsf{D}
      \arrow[from=U, to=D, Rightarrow, "\eta"]
    \end{tikzcd}
    \qquad
    \mapsto
    \qquad
    \begin{tikzcd}[column sep=huge]
      \mathsf{C}
      \arrow[r, bend left, "\mathcal{G} \circ \mathcal{F}"{name=U}]
      \arrow[r, bend right, swap, "\mathcal{G} \circ \mathcal{F}'"{name=D}]
      & \mathsf{D}
      \arrow[from=U, to=D, Rightarrow, "\mathcal{F}^{*}(\eta)"]
    \end{tikzcd}
  \end{equation*}
\end{definition}

\begin{definition}[Kan extension]
  \label{def:kan_extension}
  Suppose the pullback functor $\mathcal{F}^{*}$ defined above has a left adjoint
  \begin{equation*}
    \mathcal{F}_{!}\colon [\mathsf{C}, \mathsf{D}] \to [\mathsf{C}', \mathsf{D}].
  \end{equation*}
  and right adjoint
  \begin{equation*}
    \mathcal{F}_{*}\colon [\mathsf{C}, \mathsf{D}] \to [\mathsf{C}', \mathsf{D}].
  \end{equation*}
  \begin{itemize}
    \item We call $\mathcal{F}_{!}$ the \defn{left Kan extension} along $\mathcal{F}$. For $\mathcal{\mathcal{G}} \in [\mathsf{C}, \mathsf{D}]$, we call $\mathcal{F}_{!}(\mathcal{G})$ the \defn{left Kan extension of $\mathcal{G}$ along $\mathcal{F}$}.

    \item We call $\mathcal{F}_{*}$ the \defn{right Kan extension} along $\mathcal{F}$, and for any $\mathcal{G} \in [\mathsf{C}, \mathsf{D}]$, we call $\mathcal{F}_{*}(\mathcal{G})$ the \defn{right Kan extension of $\mathcal{G}$ along $\mathcal{F}$}.
  \end{itemize}
\end{definition}

\begin{example}[Kan extensions capture limits and colimits]
  Consider the case in which $\mathsf{C}'$ is the terminal category $\mathsf{1}$ (\hyperref[eg:categorywithoneobject]{Example~\ref*{eg:categorywithoneobject}}). There is a unique functor from any category $\mathsf{C} \to \mathsf{1}$, which sends every object to $\mathsf{1}$ and every morphism to the identity morphism. Furthermore, any functor $\mathcal{F}\colon 1 \to \mathsf{D}$ is completely determined by where it sends the unique object $*$, as $\mathcal{F}(\id_{*}) = \id_{\mathcal{F}(\id(*))}$.

  Now fix categories $\mathsf{C}$, and consider functors $\mathcal{F}\colon \mathsf{C} \to \mathsf{1}$ and $\mathcal{F}_{X}\colon \mathsf{1} \to \mathsf{D}$.

  Suppose we are given an adjunction $\mathcal{F}^{*} \dashv \mathcal{F}_{*}$. This can be realized as a hom-set bijection
  \begin{equation*}
    \Hom_{[\mathsf{C}, \mathsf{D}]}(\mathcal{F}^{*}\mathcal{F}_{X}, \mathcal{G}) \simeq \Hom_{[\mathsf{1}, \mathsf{D}]}(\mathcal{F}_{X}, \mathcal{F}_{*}\mathcal{G}).
  \end{equation*}

  Now let us think about what precisely is contained in each of the hom-sets. First note that the functor $\mathcal{F}^{*}\mathcal{F}_{X} = \mathcal{F}_{X} \circ \mathcal{F}$ is simply the constant functor which assigns to every object in $\mathcal{D}$ the object $X \in \Obj(\mathsf{D})$, and to every morphism the identity morphism $\id_{X}$. A natural transformation from this to any other functor
\end{example}


\section{Local Kan extensions}
\label{sec:local_kan_extensions}

By \hyperref[sec:unit_counit_adjunctions_are_hom_set_adjunctions]{Section~\ref*{sec:unit_counit_adjunctions_are_hom_set_adjunctions}}, we can also look at $\mathcal{F}$'s right-adjointness as guaranteeing that for every $\mathcal{D} \in \Obj([\mathsf{C}', \mathsf{D}])$, the category $(\mathcal{F}^{*} \downarrow \mathcal{D})$ has a terminal object. Similarly, we can look at $\mathcal{F}_{!}$'s left-adjointness as demonstrating that the category $(\mathcal{D} \downarrow \mathcal{F}^{*})$ has an initial object. This mirrors the definition of limits and colimits as being terminal (resp. initial) objects in the category $(\Delta \downarrow \mathcal{D})$ (resp. $(\mathcal{D} \downarrow \Delta)$).

For now, we will specialize to limits, so as not to have to keep writing down 'resp.'

We have seen that there is a functor which assigns limits, and that this is the right adjoint to the diagonal functor. Of course, we could have defined limits as being the images of the limit functor. However, this would have been a wasteful definition, since in order to define limits for any diagram $\mathcal{D}\colon \mathsf{J} \to \mathsf{C}$ we would have had to demand that limits exist for \emph{every} diagram $\mathsf{J} \to \mathsf{C}$. Clearly this is not what we want; for example, we want to be able to take products of two objects in a category without worrying about whether the product of every two objects exists in that category.

But this is not what we did: we first defined the categories $(\Delta \downarrow \mathcal{D})$, then defined limits as their terminal objects. Only much later did we show that under certain conditions, the limit was a functor whose left adjoint was the diagonal functor.

By comparison, we first defined right Kan extensions as right adjoints, then showed that their existence was equivalent to the existence of terminal morphisms in $(\mathcal{F}^{*} \downarrow \mathcal{D})$. The analogy with limits suggests that we can define partial Kan extensions along a functor $\mathcal{F}$ even when the extension functor $\mathcal{F}_{*}$ does not exist.

This definition will turn out to be much more useful for us. It will turn out even to allow us to compute Kan extensions in certain situations.

\subsection{Extensions}
\label{ssc:extensions}

An extension is an approximate solution to the problem that the Kan extension solves.

\begin{definition}[extension]
  \label{def:extension}
  Consider categories and functors as below.
  \begin{equation*}
    \begin{tikzcd}
        \mathsf{C}
        \arrow[dr, swap, "\mathcal{F}"]
        \arrow[rr, "\mathcal{D}"]
        && \mathsf{D}
        \\
        & \mathsf{C'}
      \end{tikzcd}
  \end{equation*}
    A \defn{right extension of $\mathcal{G}$ along $\mathcal{F}$} is an object in the comma category $(\mathcal{F}^{*} \downarrow \mathcal{D})$. That is, it is a pair $(\mathcal{G}, \eta)$, where $\mathcal{G}\colon \mathsf{C}' \to \mathsf{D}$ and
    \begin{equation*}
      \eta\colon \mathcal{D} \Rightarrow \mathcal{F}^{*}\mathcal{G}
    \end{equation*}
\end{definition}

\subsection{Right extensions induce cones}
\label{ssc:right_extensions_induce_cones}

Let $\mathcal{D}\colon \mathsf{C} \to \mathsf{D}$ and $\mathcal{F}\colon \mathsf{C} \to \mathsf{C'}$ be functors. Let $(G, \eta)$ be a right extension of $\mathcal{D}$ along $\mathcal{F}$.
\begin{equation*}
  \begin{tikzcd}
    \mathsf{C}
    \arrow[dr, swap, "\mathcal{F}"]
    \arrow[rr, "\mathcal{D}", ""'{name=U}]
    && \mathsf{D}
    \\
    & \mathsf{C'}
    \arrow[ur, swap, "\mathcal{G}"]
    \arrow[d, to=U, Rightarrow, swap, "\eta"] 
  \end{tikzcd}
\end{equation*}
For any $c \in \Obj(\mathsf{C}')$, denote by $(c \downarrow \mathcal{F})$ the category of morphisms from $c$ down to $\mathcal{F}$ (see \hyperref[def:categoryofmorphismsfromanobjecttoafunctor]{Definition~\ref*{def:categoryofmorphismsfromanobjecttoafunctor}}), and denote by $\mathcal{U}$ the forgetful functor $(c \downarrow \mathcal{F}) \to \mathsf{C}$. We can define a functor $\evl{\mathcal{D}}_{(c \downarrow \mathcal{F})}\colon (c \downarrow \mathcal{F}) \to \mathsf{D}$ as follows.
\begin{equation*}
  \begin{tikzcd}
    (c \downarrow \mathcal{F})
    \arrow[r, "\mathcal{U}"]
    & \mathsf{C}
    \arrow[r, "\mathcal{D}"]
    & \mathsf{D}
  \end{tikzcd}
\end{equation*}

To each such $c \in \Obj(\mathsf{C}')$, there is an associated cone $\xi$ over $\evl{\mathcal{D}}_{(c \downarrow \mathcal{F})}$, constructed as follows.

\begin{itemize}
  \item The tip of the cone is $\mathcal{D}(c)$.

  \item For each $(e, f) \in \Obj(c \downarrow \mathcal{F})$, the morphism $\xi_{(e, f)}\colon\mathcal{D}(c) \to \evl{\mathcal{D}}_{(c \downarrow \mathcal{F})}(e, f) = \mathcal{D}(e)$ is defined by the composition
    \begin{equation*}
      \begin{tikzcd}
        \mathcal{G}(c)
        \arrow[r, "\mathcal{G}(f)"]
        & \mathcal{G}(\mathcal{F}(e))
        \arrow[r, "\eta_{e}"]
        & \mathcal{D}(e)
      \end{tikzcd}.
    \end{equation*}
\end{itemize}

\begin{lemma}
  \label{lemma:right_extension_induces_cone}
  The cone described above really is a cone. That is, for each morphism $g\colon (e, f) \to (e', f')$,
  \begin{equation*}
    \begin{tikzcd}[column sep=tiny]
      & c
      \arrow[ld, swap, "f"]
      \arrow[rd, "f'"]
      \\
      \mathcal{F}(e)
      \arrow[rr, swap, "\mathcal{F}(g)"]
      && \mathcal{F}(e')
    \end{tikzcd}
  \end{equation*}
  the diagram
  \begin{equation*}
    \begin{tikzcd}[column sep=tiny]
      & \mathcal{D}(c)
      \arrow[dl, swap, "\xi_{(e, f)}"]
      \arrow[dr, "\xi_{(e', f')}"]
      \\
      \mathcal{D}(e)
      \arrow[rr, swap, "\mathcal{D}(g)"]
      && \mathcal{D}(e')
    \end{tikzcd}
  \end{equation*}
  commutes.
\end{lemma}
\begin{proof}
  Consider the following diagram.
  \begin{equation*}
    \begin{tikzcd}[column sep=tiny]
      & \mathcal{D}(c)
      \arrow[ld, swap, "\mathcal{G}(f)"]
      \arrow[rd, "\mathcal{G}(f')"]
      \\
      (\mathcal{G} \circ \mathcal{F})(e)
      \arrow[rr, swap, "\mathcal{G} \circ \mathcal{F}(e)"]
      \arrow[d, swap, "\eta_{e}"]
      && \mathcal{G} \circ \mathcal{F}(e')
      \arrow[d, "\eta_{e'}"]
      \\
      \mathcal{D}(e)
      \arrow[rr, swap, "\mathcal{D}(f)"]
      &&\mathcal{D}(e')
    \end{tikzcd}
  \end{equation*}
  The upper triangle is the functor $\mathcal{D}$ applied to the definition of commutativity in a category of morphisms, and the lower square is the naturality square for $\eta$.
\end{proof}

\begin{lemma}
  \label{lemma:morphism_induces_functor_between_comma_categories}
  Let $\mathcal{F}\colon \mathsf{I} \to \mathsf{J}$ be a functor. Let $f\colon i \to i'$ be a morphism in $\mathsf{I}$. Then $\mathcal{F}$ induces the following functors.
  \begin{itemize}
    \item $(\mathcal{F} \downarrow i) \to (\mathcal{F} \downarrow i')$.

    \item $(i' \downarrow \mathcal{F}) \to (i \downarrow \mathcal{F})$.
  \end{itemize}
\end{lemma}

We saw above that given any extension $(\mathcal{G}, \eta)$ and any object $c \in \Obj(\mathsf{C}')$, we could construct a cone. This cone turns out to encode a lot of interesting properties about the extension $(\mathcal{G}, \eta)$. In fact, we have the following rather astonishing result.

\begin{theorem}
  Let $(\mathcal{G}, \eta)$ be a right extension as below.
  \begin{equation*}
    \begin{tikzcd}
      \mathsf{C}
      \arrow[dr, swap, "\mathcal{F}"]
      \arrow[rr, "\mathcal{D}", ""'{name=U}]
      && \mathsf{D}
      \\
      & \mathsf{C'}
      \arrow[ur, swap, "\mathcal{G}"]
      \arrow[d, to=U, Rightarrow, swap, "\eta"] 
    \end{tikzcd}
  \end{equation*}
  If for every $c \in \Obj(\mathsf{C'})$ the cone 
  
  is a limit cone, then $(\mathcal{G}, \eta)$ is a right Kan extension.
\end{theorem}
\begin{proof}

\end{proof}

\end{document}
