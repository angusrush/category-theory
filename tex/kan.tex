\documentclass[notes.tex]{subfiles}

\begin{document}

\chapter{Kan extensions}\label{sec:kan_extensions}

\section{Motivation}
\label{sec:motivation}

Let $\mathsf{C}$ be a category, and $\mathcal{I}\colon \mathsf{S} \to \mathsf{C}$ a subcategory inclusion. Given a functor $\mathcal{F}$ from $\mathcal{C}$ to any category $\mathcal{D}$, we can restrict $\mathcal{F}$ to $\mathcal{S}$ by pulling it back with $\mathcal{I}$. We denote this restriction by
\begin{equation*}
  \mathcal{I}^{*}(\mathcal{F}) = \mathcal{F} \circ \mathcal{I}.
\end{equation*}

In fact, $\mathcal{I}$ induces a functor on the category of functors $\mathsf{C} \to \mathsf{D}$ (\hyperref[def:functorcategory]{Definition~\ref*{def:functorcategory}}). Recall that we denoted this category by $[\mathsf{C}, \mathsf{D}]$. 

Admittedly, restricting the domain of a functor is not a very good party trick, even if done in a functorial way. The more impressive thing would be to \emph{increase} the domain of a functor. 

We have come up against this kind of problem before. For example, it is easy to turn a group into a set by forgetting the group structure. We saw that there was a way to turn a set into a group in a canonical way by forming the free group over it. We mentioned that this was a reflection of a more general phenomenon known as \emph{free-forgetful adjunction} (\hyperref[note:freeforgetfuladjunctions]{Note~\ref*{note:freeforgetfuladjunctions}}). This suggests that the best possible solution to extending the domain of a functor is to search for a functor which is adjoint to the restriction functor $\mathcal{I}^{*}$.

Of course, it is also interesting to consider the case where $\mathcal{I}$ is not an inclusion. In fact, we will see that by choosing $\mathcal{I}$ in different ways leads to a generalization of many different categorical concepts, such as limits. Left and right adjoints will be known as \emph{left right Kan extensions.}

\section{Global Kan extensions}
\label{sec:global_kan_extensions}

The most general, high-level notion of a Kan extension is called a \emph{global Kan extension.}

Let $\mathcal{F}\colon \mathsf{C} \to \mathsf{C}'$ be a functor. For any other category $\mathsf{D}$, $\mathcal{F}$ induces a functor
\begin{equation*}
  \mathcal{F}^{*}\colon [\mathsf{C}', \mathsf{D}] \to [\mathsf{C}, \mathsf{D}]
\end{equation*}
which sends a functor $\mathcal{G}\colon \mathsf{C}' \to \mathsf{D}$ to its precomposition $\mathcal{G} \circ \mathcal{F}$
\begin{equation*}
  \begin{tikzcd}
    \mathsf{C}'
    \arrow[r, "\mathcal{G}"]
    & \mathsf{D}
  \end{tikzcd}
  \qquad
  \mapsto
  \qquad
  \begin{tikzcd}
    \mathsf{C}
    \arrow[r, swap, "\mathcal{F}"]
    \arrow[rr, bend left, "\mathcal{F}^{*}(\mathcal{G})"]
    & \mathsf{C'}
    \arrow[r, swap, "\mathcal{G}"]
    & \mathsf{D}
  \end{tikzcd}
\end{equation*}
and a natural transformation $\eta\colon \mathcal{G} \to \mathcal{G}'$ to its left whiskering (\hyperref[eg:whiskering]{Example~\ref*{eg:whiskering}}) $\eta\mathcal{F}$.
\begin{equation*}
  \begin{tikzcd}[column sep=huge]
    \mathsf{C}
    \arrow[r, "\mathcal{F}"]
    &\mathsf{C}'
    \arrow[r, bend left, "\mathcal{G}"{name=U}]
    \arrow[r, bend right, swap, "\mathcal{G}'"{name=D}]
    & \mathsf{D}
    \arrow[from=U, to=D, Rightarrow, "\eta"]
  \end{tikzcd} 
  \qquad
  \mapsto
  \qquad
  \begin{tikzcd}[column sep=huge]
    \mathsf{C}
    \arrow[r, bend left, "\mathcal{G} \circ \mathcal{F}"{name=U}]
    \arrow[r, bend right, swap, "\mathcal{G} \circ \mathcal{F}'"{name=D}]
    & \mathsf{D}
    \arrow[from=U, to=D, Rightarrow, "\mathcal{F}^{*}(\eta)"]
  \end{tikzcd}
\end{equation*}

\begin{definition}[Kan extension]
  \label{def:kan_extension}
  Suppose the functor $\mathcal{F}^{*}$ defined above has a left adjoint
  \begin{equation*}
    \mathcal{F}_{!}\colon [\mathsf{C}, \mathsf{D}] \to [\mathsf{C}', \mathsf{D}].
  \end{equation*}
  and right adjoint
  \begin{equation*}
    \mathcal{F}_{*}\colon [\mathsf{C}, \mathsf{D}] \to [\mathsf{C}', \mathsf{D}].
  \end{equation*}
  \begin{itemize}
    \item We call $\mathcal{F}_{!}$ the \defn{left Kan extension} along $\mathcal{F}$. For $\mathcal{\mathcal{G}} \in [\mathsf{C}, \mathsf{D}]$, we call $\mathcal{F}_{!}(\mathcal{G})$ the \defn{left Kan extension of $\mathcal{G}$ along $\mathcal{F}$}. 
          
    \item We call $\mathcal{F}_{*}$ the \defn{right Kan extension} along $\mathcal{F}$, and for any $\mathcal{G} \in [\mathsf{C}, \mathsf{D}]$, we call $\mathcal{F}_{*}(\mathcal{G})$ the \defn{right Kan extension of $\mathcal{G}$ along $\mathcal{F}$}.
  \end{itemize}
\end{definition}

\begin{example}
  Consider the case in which $\mathsf{C}'$ is the terminal category $\mathsf{1}$ (\hyperref[eg:categorywithoneobject]{Example~\ref*{eg:categorywithoneobject}}). There is a unique functor from any category $\mathsf{C} \to \mathsf{1}$, which sends every object to $\mathsf{1}$ and every morphism to the identity morphism. Furthermore, any functor $\mathcal{F}\colon 1 \to \mathsf{D}$ is completely determined by where it sends the unique object $*$, as $\mathcal{F}(\id_{*}) = \id_{\mathcal{F}(\id(*))}$.

  Now fix categories $\mathsf{C}$, and consider functors $\mathcal{F}\colon \mathsf{C} \to \mathsf{1}$ and $\mathcal{F}_{X}\colon \mathsf{1} \to \mathsf{D}$.

  Suppose we are given an adjunction $\mathcal{F}^{*} \dashv \mathcal{F}_{*}$. This can be realized as a hom-set bijection
  \begin{equation*}
    \Hom_{[\mathsf{C}, \mathsf{D}]}(\mathcal{F}^{*}\mathcal{F}_{X}, \mathcal{G}) \simeq \Hom_{[\mathsf{1}, \mathsf{D}]}(\mathcal{F}_{X}, \mathcal{F}_{*}\mathcal{G}).
  \end{equation*}

  Now let us think about what precisely is contained in each of the hom-sets. First note that the functor $\mathcal{F}^{\infty}\mathcal{F}_{X} = \mathcal{F}_{X} \circ \mathcal{F}$ is simply the constant functor which assigns to every object in $\mathcal{D}$ the object $X \in \Obj(\mathsf{D})$, and to every morphism the identity morphism $\id_{X}$. A natural transformation from this to any other functor
\end{example}

\end{document}
