\documentclass[notes.tex]{subfiles}

\begin{document}

\chapter{Adjunctions}\label{sec:adjunctions}

\section{Motivating example}
\label{sec:motivating_example}

\begin{note}
  This section is under heavy construction. It's not very good at the moment.
\end{note}

Consider the following functors:
\begin{itemize}
  \item $\mathcal{U}\colon \mathsf{Grp} \rightarrow \mathsf{Set}$, which sends a group to its underlying set, and

  \item $\mathcal{F}\colon \mathsf{Set} \rightarrow \mathsf{Grp}$, which sends a set to the free group on it.
\end{itemize}

The functors $\mathcal{U}$ and $\mathcal{F}$ are dual in the following sense: $\mathcal{U}$ is the most efficient way of moving from $\mathsf{Grp}$ to $\mathsf{Set}$ since all groups are in particular sets; $\mathcal{F}$ might be thought of as providing the most efficient way of moving from $\mathsf{Set}$ to $\mathsf{Grp}$. But how would one go about formalizing this?

Free-forgetful adjunctions are an example of an extremely powerful structure called an \emph{adjuction}. We have encountered it before---we saw in \hyperref[eg:naturaltransformationsforcccs]{Example~\ref*{eg:naturaltransformationsforcccs}} that there was a natural bijection between
\begin{equation*}
  \Hom_{\mathsf{C}}(X \times (-), B)\qquad\text{and}\qquad \Hom_{\mathsf{C}}(X, B^{(-)}).
\end{equation*}
this is a so-called \emph{hom-set adjunction}.

\subsection{Hom-set adjunction}
\label{ssc:hom_set_adjunction}

The functors $\mathcal{U}$ and $\mathcal{F}$ have the following property. Let $S$ be a set, $G$ be a group, and let $f\colon S \to \mathcal{U}(G)$, $s \mapsto f(s)$ be a set-function. Then there is an associated group homomorphism $\tilde{f}\colon \mathcal{F}(S) \to G$, which sends $s_{1}s_{2}\dots s_{n} \mapsto f(s_{1}s_{2}\dots s_{n}) = f(s_{1})\cdots f(s_{n})$. In fact, $\tilde{f}$ is the unique homomorphism $\mathcal{F}(S) \to G$ such that $f(s) = \tilde{f}(s)$ for all $s \in S$.

Similarly, for every group homomorphism $g\colon \mathcal{F}(S) \to G$, there is an associated function $S \to \mathcal{U}(G)$ given by restricting $g$ to $S$. In fact, this is the unique function $\mathcal{F}(S) \to G$ such that $f(s) = \tilde{f}(s)$ for all $s \in S$.

Thus for each $f \in \Hom_{\mathsf{Grp}}(S, \mathcal{U}(G))$ we can construct an $\tilde{f} \in \Hom_{\mathsf{Set}}(\mathcal{F}(S), G)$, and vice versa.

Let us add some mathematical scaffolding to the ideas explored above. We build two functors $\mathsf{Set}^{\mathrm{op}} \times \mathsf{Grp} \rightarrow \mathsf{Set}$ as follows.
\begin{enumerate}
  \item Our first functor maps the object $(S, G) \in \Obj(\mathsf{Set}^{\mathrm{op}} \times \mathsf{Grp})$ to the hom-set $\Hom_{\mathsf{Grp}}(\mathcal{F}(S), G)$, and a morphism $(\alpha, \beta)\colon (S,G) \to (S', G')$ to a function
    \begin{equation*}
      \Hom_{\mathsf{Grp}}(\mathcal{F}(S), G) \to \Hom_{\mathsf{Grp}}(\mathcal{F}(S'), G');\qquad m \mapsto \mathcal{F}(\alpha) \circ m \circ \beta
    \end{equation*}

  \item Our second functor maps $(S, G)$ to $\Hom_{\mathsf{Set}}(S, \mathcal{U}(G))$, and $(\alpha, \beta)$ to
    \begin{equation*}
      m \mapsto \alpha \circ m \circ \mathcal{U}(\beta).
    \end{equation*}
\end{enumerate}
We can define a natural isomorphism $\Phi$ between these functors with components
\begin{equation*}
  \Phi_{S, G}\colon \Hom_{\mathsf{Grp}}(\mathcal{F}(S), G) \to \Hom_{\mathsf{Set}}(S, \mathcal{U}(G));\qquad f \to \tilde{f}.
\end{equation*}

\subsection{Unit-counit adjunction}
\label{ssc:unit_counit_adjunction}

Suppose we take the natural isomorphism from the last section
\begin{equation*}
  \Phi_{S, G}\colon \Hom_{\mathsf{Grp}}(\mathcal{F}(S), G) \to \Hom_{\mathsf{Set}}(S, \mathcal{U}(G))
\end{equation*}
and evaluate it at $S = \mathcal{U}(G)$. This gives us an isomorphsim
\begin{equation*}
  \Phi_{\mathcal{U}(G), G}\colon \Hom_{\mathsf{Grp}}(\mathcal{F}(\mathcal{U}(S)), G) \to \Hom_{\mathsf{Set}}(\mathcal{U}(G), \mathcal{U}(G)).
\end{equation*}
In particular, the element $\id_{\mathcal{U}(G)}$ is the image of some group homomorphism
\begin{equation*}
  \epsilon_{G}\colon \mathcal{F}(\mathcal{U}(G)) \to G.
\end{equation*}
Taken together for all $G$, the $\epsilon_{G}$ form a natural transformation
\begin{equation*}
  \epsilon\colon \mathcal{F} \circ \mathcal{U} \to \id_{\mathsf{Grp}}.
\end{equation*}

Evaluating it at $G = \mathcal{F}(S)$. This gives us an isomorphism
\begin{equation*}
  \Hom_{\mathsf{Grp}}(\mathcal{F}(S), \mathcal{F}(S)) \to \Hom_{\mathsf{Set}}(S, \mathcal{U}(\mathcal{F}(G))).
\end{equation*}
In particular, the identity $\id_{\mathcal{F}(G)}$ is mapped to some map
\begin{equation*}
  \eta_{S}\colon S \to \mathcal{U}(\mathcal{G}(S)).
\end{equation*}
In fact, taken together, the $\eta_{S}$ form a natural transformation
\begin{equation*}
  \eta\colon \id_{\mathsf{Set}} \Rightarrow \mathcal{U} \circ \mathcal{G}.
\end{equation*}

\subsection{Adjunction from universal property}
\label{ssc:adjunction_from_universal_property}

The previous discussion may be slightly odd to those who have already encountered free groups, because free groups are almost always defined in terms of a universal property rather than some sort of abstract nonsense involving functors. The universal property of the free group is as follows.

Let $S$ be a set, and let $f\colon S \to G$ be any function from $S$ to a group $G$. Then there is a unique group homomorphism $\tilde{f}\colon \mathcal{F}(S) \to G$ such that the following diagram commutes.
\begin{equation*}
  \begin{tikzcd}
    S
    \arrow[r, "\eta_{S}"]
    \arrow[rd, swap, "f"]
    & (\mathcal{U} \circ \mathcal{F})(S)
    \arrow[d, "\mathcal{U}(\tilde{f})"]
    & \mathcal{F}(S)
    \arrow[d, "\exists!\tilde{f}"]
    \\
    & \mathcal{U}(G)
    & G
  \end{tikzcd}
\end{equation*}

As we saw in \hyperref[sec:universalproperties]{Section~\ref*{sec:universalproperties}}, this simply says that the free group over $S$ is initial in the comma category $(S \downarrow \mathcal{U})$.

We have written the injection $S \hookrightarrow (\mathcal{U} \circ \mathcal{F})(S)$ in the evocative notation $\eta_{S}$. As the reader has probably guessed, this is because taken together the $\eta_{S}$ form a natural transformation which functions as the unit in a unit-counit adjunction.

\section{Hom set adjunction}
\label{sec:hom_set_adjunction}

\begin{definition}[hom-set adjunction]
  \label{def:homsetadjunction}
  Let $\mathsf{C}$, $\mathsf{D}$ be categories and $\mathcal{F}$, $\mathcal{G}$ functors as follows.
  \begin{equation*}
    \begin{tikzcd}
      \mathsf{C}
      \arrow[r, rightarrow, bend left, "\mathcal{F}"]
      & \mathsf{D}
      \arrow[l, rightarrow, bend left, "\mathcal{G}"]
    \end{tikzcd}
  \end{equation*}
  We say that \defn{$\mathcal{F}$ is left-adjoint to $\mathcal{G}$} (or equivalently $\mathcal{G}$ is right-adjoint to $\mathcal{F}$) and write $\mathcal{F} \dashv \mathcal{G}$ if there is a natural isomorphism
  \begin{equation*}
    \Phi\colon \Hom_{\mathsf{D}}(\mathcal{F}(-), -) \Rightarrow \Hom_{\mathsf{C}}(-, \mathcal{G}(-)),
  \end{equation*}
  which fits between $\mathcal{F}$ and $\mathcal{G}$ like this.
  \begin{equation*}
    \begin{tikzcd}[column sep=huge]
      \mathsf{C}^{\mathrm{op}} \times \mathsf{D}
      \arrow[r, bend left, rightarrow, anchor=s, "{\Hom_{\mathsf{D}}(\mathcal{F}(-), -)}", ""{name=U, below}]
      \arrow[r, bend right, rightarrow, swap, "{\Hom_{\mathsf{D}}(-, \mathcal{G}(-))}", ""{name=D, above}]
      \arrow[from=U, to=D, Rightarrow, "\Phi"]
      & \mathsf{Set}\qquad
    \end{tikzcd}
  \end{equation*}
  The natural isomorphsim amounts to a family of bijections
  \begin{equation*}
    \Phi_{A, B}\colon \Hom_{\mathsf{D}}(\mathcal{F}(A), B) \to \Hom_{\mathsf{C}}(A, \mathcal{G}(B))
  \end{equation*}
  which satisfies the coherence conditions for a natural transformation.

\end{definition}

\section{Unit-counit adjunction}
\label{sec:unit_counit_adjunction}

\begin{definition}[unit-counit adjunction]
  \label{def:unitcounitadjunction}
  We say that two functors $\mathcal{F}\colon \mathsf{C} \rightarrow \mathsf{D}$ and $\mathcal{G}\colon \mathsf{D} \rightarrow \mathsf{C}$ form a \defn{unit-counit adjunction} if there are two natural transformations
  \begin{equation*}
    \eta\colon \id_{\mathsf{C}} \Rightarrow \mathcal{G} \circ \mathcal{F},\qquad\text{and}\qquad \varepsilon\colon \mathcal{F} \circ \mathcal{G} \Rightarrow \id_{\mathsf{D}},
  \end{equation*}
  called the \emph{unit} and \emph{counit} respectively, which make the following so-called \emph{triangle diagrams}
  \begin{equation*}
    \begin{tikzcd}
      \mathcal{F}
      \arrow[r, Rightarrow, "\mathcal{F}\eta"]
      \arrow[rd, Rightarrow, swap, "\id_{\mathcal{F}}"]
      & \mathcal{FGF}
      \arrow[d, Rightarrow, "\varepsilon\mathcal{F}"]
      \\
      & \mathcal{F}
    \end{tikzcd},
    \qquad
    \begin{tikzcd}
      \mathcal{G}
      \arrow[r, Rightarrow, "\eta\mathcal{G}"]
      \arrow[rd, Rightarrow, swap, "\id_{\mathcal{G}}"]
      & \mathcal{GFG}
      \arrow[d, Rightarrow, "\mathcal{G}\varepsilon"]
      \\
      & \mathcal{G}
    \end{tikzcd}
  \end{equation*}
  commute.
\end{definition}

\begin{note}
  Using $\eta$ for the unit and $\epsilon$ for the counit is very common, but not universal! Both Wikipedia and the nLab have them this way, but the Higher Categories course uses them the other way around.
\end{note}

The triangle diagrams take quite some explanation. The unit $\eta$ is a natural transformation $\id_{\mathsf{C}} \Rightarrow \mathcal{G} \circ \mathcal{F}$. We can draw it like this.
\begin{equation*}
  \begin{tikzcd}[column sep=huge]
    \mathsf{C}
    \arrow[r, rightarrow, bend left, "\id_{\mathsf{C}}"{name=U}]
    \arrow[r, rightarrow, swap, bend right, "\mathcal{G} \circ \mathcal{F}"{name=D}]
    \arrow[from=U, to=D, Rightarrow, "\eta"]
    & \mathsf{C}
  \end{tikzcd}.
\end{equation*}
Analogously, we can draw $\varepsilon$ like this.
\begin{equation*}
  \begin{tikzcd}[column sep=huge]
    \mathsf{D}
    \arrow[r, rightarrow, bend left, "\mathcal{F} \circ \mathcal{G}"{name=U}]
    \arrow[r, rightarrow, swap, bend right, "\id_{\mathsf{D}}"{name=D}]
    \arrow[from=U, to=D, Rightarrow, "\varepsilon"]
    & \mathsf{D}
  \end{tikzcd}.
\end{equation*}

We can arrange these artfully like so.
\begin{equation*}
  \begin{tikzcd}
    \mathsf{C}
    \arrow[r, rightarrow, "\mathcal{F}" description]
    \arrow[rr, rightarrow, bend left=75, "\id_{\mathsf{C}}"{name=UU}]
    \arrow[rr, swap, rightarrow, bend left=30, "\mathcal{G} \circ \mathcal{F}"{name=U}]
    & \mathsf{D}
    \arrow[r, rightarrow, "\mathcal{G}" description]
    \arrow[rr, rightarrow, bend right=75, swap, "\id_{\mathsf{D}}"{name=DD}]
    \arrow[rr, rightarrow, bend right=30, "\mathcal{F} \circ \mathcal{G}"{name=D}]
    \arrow[from=D, to=DD, Rightarrow, "\varepsilon"]
    & \mathsf{C}
    \arrow[from=UU, to=U, Rightarrow, swap, "\eta"]
    \arrow[r, rightarrow, "\mathcal{F}" description]
    & \mathsf{D}
  \end{tikzcd}
\end{equation*}
Notice that we haven't actually \emph{done} anything; this diagram is just the diagrams for the unit and counit, plus some extraneous information.

We can whisker the $\eta$ on top from the right, and the $\varepsilon$ below from the left, to get the following diagram,
\begin{equation*}
  \begin{tikzcd}
    \mathsf{C}
    \arrow[r, rightarrow, "\mathcal{F}" description]
    \arrow[rrr, rightarrow, bend right=60, swap, "\mathcal{F}"{name=D}]
    \arrow[rrr, rightarrow, bend left=60, "\mathcal{F}"{name=U}]
    \arrow[rrr, rightarrow, bend left=30, swap, "\mathcal{F} \circ \mathcal{G} \circ \mathcal{F}"{name=MU}]
    \arrow[rrr, rightarrow, bend right=30, "\mathcal{F} \circ \mathcal{G} \circ \mathcal{F}"{name=MD}]
    & \mathsf{D}
    \arrow[from=U, to=MU, Rightarrow, swap, "\mathcal{F}\eta"]
    \arrow[r, rightarrow, "\mathcal{G}" description]
    \arrow[from=MD, to=D, Rightarrow, "\varepsilon\mathcal{F}"]
    & \mathsf{C}
    \arrow[r, rightarrow, "\mathcal{F}" description]
    & \mathsf{D}
  \end{tikzcd}
\end{equation*}
then consolidate to get this:
\begin{equation*}
  \begin{tikzcd}[row sep=huge, column sep=huge]
    \mathsf{C}
    \arrow[r, rightarrow, bend left=60, "\mathcal{F}"{name=U}]
    \arrow[r, rightarrow, "\mathcal{F} \circ \mathcal{G} \circ \mathcal{F}"{name=M} description]
    \arrow[r, rightarrow, bend right=60, swap, "\mathcal{F}"{name=D}]
    & \mathsf{D}
    \arrow[from=U, to=M, Rightarrow, "\mathcal{F}\eta"]
    \arrow[from=M, to=D, Rightarrow, "\varepsilon\mathcal{F}"]
  \end{tikzcd}
\end{equation*}
We can then take the composition $\varepsilon \mathcal{F} \circ \mathcal{F}\eta$ to get a natural transformation $\mathcal{F} \Rightarrow \mathcal{F}$
\begin{equation*}
  \begin{tikzcd}[row sep=huge, column sep=huge]
    \mathsf{C}
    \arrow[r, bend left, rightarrow, "\mathcal{F}"{name=U}]
    \arrow[r, bend right, swap, rightarrow, "\mathcal{F}"{name=D}]
    \arrow[from=U, to=D, Rightarrow, "\varepsilon\mathcal{F} \circ \mathcal{F}\eta"]
    &\mathsf{D};
  \end{tikzcd}
\end{equation*}
the first triangle diagram says that this must be the same as the identity natural transformation $\id_{\mathcal{F}}$.

What this means in practice is that if we start with the functor $\mathcal{F}$, use the unit to get $\mathcal{F} \circ \mathcal{G} \mathcal{F}$, then use the counit to map back to $\mathcal{F}$, this should be the same as not doing anything at all.

The second triangle diagram is analogous; it tells us that starting with $\mathcal{G}$, using the counit to send this to $\mathcal{G} \circ \mathcal{F} \circ \mathcal{G}$, then going back to $\mathcal{G}$ with the unit should be the same as not doing anything.

\begin{proposition}
  Consider a unit-counit adjunction between functors $\mathcal{F}$ and $\mathcal{G}$ as above.
  \begin{enumerate}
    \item The functor $\mathcal{F}$ is fully faithful if and only if the unit
      \begin{equation*}
        \epsilon\colon \mathcal{F} \circ \mathcal{G} \Rightarrow \id_{\mathsf{D}}
      \end{equation*}
      is a natural isomorphism.

    \item The functor $\mathcal{G}$ is fully faithful if and only if the counit
      \begin{equation*}
        \eta\colon \mathcal{G} \circ \mathcal{F} \Rightarrow \id_{\mathsf{C}}
      \end{equation*}
      is a natural isomorphism.
  \end{enumerate}
\end{proposition}
\begin{proof}
  We prove the second part. The second part is analogous. 
  
  Suppose $\eta$ is a natural isomorphism. Consider the following diagram.
  \begin{equation*}
    \begin{tikzcd}
      \mathsf{C}(a, b) 
      \arrow[d, swap, "\eta_{b} \circ (-)"]
      \arrow[rd]
      \\
      \mathsf{C}\left(a, (G \circ F)(b)\right)
      \arrow[r, swap, "\Phi_{a,b}"]
      & \mathsf{D}(F(a), F(b))
    \end{tikzcd}
  \end{equation*}
  The downward arrow is obviously a bijection, and the rightward arrow is given to us since every unit-counit adjunction is equivalently a hom-set isomorphism. Thus, their composition is also a bijection.

  Now suppose that $\mathcal{G}$ is fully faithful.
\end{proof}

\section{Adjunction from universal property}
\label{sec:adjunction_from_universal_property}

\begin{definition}[universal morphism adjunction]
  \label{def:universal_morphism_adjunction}
  Let $\mathcal{F}\colon \mathsf{C} \to \mathsf{D}$ be a functor. We say that $\mathcal{F}$ is a \defn{left universal morphism adjunction} if for each $X \in \Obj(\mathsf{D})$, the category of morphisms $(\mathcal{F} \downarrow X)$ (see \hyperref[def:categoryofmorphismsfromafunctortoanobject]{Definition~\ref*{def:categoryofmorphismsfromafunctortoanobject}}) has a terminal object, which we denote by $\mathcal{G}(X)$. That is, $\mathcal{F}$ is a left adjoing universal morphism functor if for each $X \in \Obj(\mathsf{C})$, $Y \in \Obj(\mathsf{D})$, and morphism $f\colon \mathcal{F}(Y) \to X$, we have the following diagram.
  \begin{equation*}
    \begin{tikzcd}
      \mathcal{F}(Y)
      \arrow[rd, "f"]
      \arrow[d, swap, "\mathcal{F}(g)"]
      & & Y
      \arrow[d, dashed, "\exists!g"]
      \\
      \mathcal{F}(\mathcal{G}(X))
      \arrow[r, swap, "\epsilon_{X}"]
      & X
      & \mathcal{G}(X)
    \end{tikzcd}
  \end{equation*}

  Similarly, we say that $\mathcal{F}$ is a \defn{right universal morphism adjunction} if for each $Y \in \Obj(\mathsf{C})$, the category of morphisms $(Y \downarrow \mathcal{G})$ (\hyperref[def:categoryofmorphismsfromanobjecttoafunctor]{Definition~\ref*{def:categoryofmorphismsfromanobjecttoafunctor}}) has a initial object.
\end{definition}

The above terminology is horrendous, and at some point I'll figure out a nicer way of doing this. probably be changed. The rationale behind introducing it is that it will not be around for long, since we will see that all of these definitions of left- and right adjunctions are equivalent.

\section{All of these are equivalent}
\label{sec:unit_counit_adjunctions_are_hom_set_adjunctions}

\begin{lemma}
  \label{lemma:universal_morphism_adjunction_implies_hom_set_adjunction}
  If the functors $\mathcal{F}\colon \mathsf{C} \to \mathsf{D}$ is part of a left universal morphism adjunction, there is a functor $\mathcal{G}\colon \mathsf{D} \to \mathsf{C}$ such that $\mathcal{F} \dashv \mathcal{G}$ is a hom-set adjunction.
\end{lemma}
\begin{proof}
  Suppose $\mathcal{F}$ is part of a universal adjunction (\hyperref[def:universal_morphism_adjunction]{Definition~\ref*{def:universal_morphism_adjunction}}), i.e.\ for every object $X \in \Obj(\mathsf{D})$ there exists a terminal object in the category $(\mathcal{F} \downarrow X)$, which we will denote $(\mathcal{G}(X), \epsilon_{X})$.\footnote{Of course, the objects $\mathcal{G}(X)$ are only specified up to unique isomorphism, and we have to pick representatives using the axiom of choice.} The first claim is that the assignment $X \mapsto \mathcal{G}(X)$ extends to a functor $\mathcal{G}\colon \mathsf{D} \to \mathsf{C}$.  To this end, let $f\colon X \to X'$.
  \begin{equation*}
    \begin{tikzcd}
      (\mathcal{F} \circ \mathcal{G})(X)
      \arrow[r, "\epsilon_{X}"]
      & X
      \arrow[d, "f"]
      \\
      (\mathcal{F} \circ \mathcal{G})(X')
      \arrow[r, swap, "\epsilon_{X'}"]
      & X'
    \end{tikzcd}
  \end{equation*}
  Composing $f \circ \epsilon_{X}$, we find the following diagram.
  \begin{equation*}
    \begin{tikzcd}
      (\mathcal{F} \circ \mathcal{G})(X)
      \arrow[r, "\epsilon_{X}"]
      \arrow[rd, "f \circ \epsilon_{X}"]
      & X
      \arrow[d, "f"]
      \\
      (\mathcal{F} \circ \mathcal{G})(X')
      \arrow[r, swap, "\epsilon_{X'}"]
      & X'
    \end{tikzcd}
  \end{equation*}
  The universal property now gives us a unique morphism $\mathcal{G}(X) \to \mathcal{G}(X')$, which we'll call $\mathcal{G}(f)$.
  \begin{equation*}
    \begin{tikzcd}
      (\mathcal{F} \circ \mathcal{G})(X)
      \arrow[r, "\epsilon_{X}"]
      \arrow[rd, "f \circ \epsilon_{X}"]
      \arrow[d, swap, "(\mathcal{G} \circ \mathcal{F})(f)"]
      & X
      \arrow[d, "f"]
      & \mathcal{G}(X)
      \arrow[d, "\exists!\mathcal{G}(f)"]
      \\
      (\mathcal{F} \circ \mathcal{G})(X')
      \arrow[r, swap, "\epsilon_{X'}"]
      & X'
      & \mathcal{G}(X')
    \end{tikzcd}
  \end{equation*}
  The usual manipulations (formally identical to those in \hyperref[thm:productisafunctor]{Theorem~\ref*{thm:productisafunctor}}) show that $\mathcal{G}$ respects compositions, and therefore defines a bona fide functor.

  Let's take a closer look at the commuting square above.
  \begin{equation*}
    \begin{tikzcd}
      (\mathcal{F} \circ \mathcal{G})(X)
      \arrow[r, "\epsilon_{X}"]
      \arrow[d, swap, "(\mathcal{G} \circ \mathcal{F})(f)"]
      & X
      \arrow[d, "f"]
      \\
      (\mathcal{F} \circ \mathcal{G})(X')
      \arrow[r, swap, "\epsilon_{X'}"]
      & X'
    \end{tikzcd}
  \end{equation*}
\end{proof}

\begin{lemma}
  \label{lemma:hom-set_adjunction_implies_unit-counit_adjunction}
  If the functors $\mathcal{F}\colon \mathsf{C} \to \mathsf{D}$ and $\mathcal{G}\colon \mathsf{D} \to \mathsf{C}$ form a hom-set adjunction, they form a unit-counit adjunction.
\end{lemma}
\begin{proof}[Sketch of proof]
  Our plan of attack is the following.
  \begin{enumerate}
    \item We show that from a hom-set adjunction $\mathcal{F} \dashv \mathcal{G}$, we can form, for each $A \in \Obj(\mathsf{C})$, a map
      \begin{equation*}
        \eta_{A}\colon A \to (\mathcal{G} \circ \mathcal{F})(A)
      \end{equation*}
      and for each $B \in \Obj(\mathsf{D})$, a map
      \begin{equation*}
        \epsilon_{B}\colon (\mathcal{F} \circ \mathcal{G})(B) \to B.
      \end{equation*}

    \item We show that, component-wise, these satisfy the triangle identities.
      \begin{equation*}
        {(\varepsilon\mathcal{F})}_{A} \circ {(\mathcal{F}\eta)}_{A} = {(\id_{\mathcal{F}})}_{A}\qquad\text{and}\qquad {(\mathcal{G}_{\varepsilon})}_{B} \circ {(\eta\mathcal{G})}_{B} = {(\id_{\mathcal{G}})}_{B}.
      \end{equation*}

    \item We show that $\eta_{A}$ and $\epsilon_{B}$ are components of two natural transformations
      \begin{equation*}
        \eta\colon \id_{\mathsf{C}} \to \mathcal{G} \circ \mathcal{F} \qquad \text{and}\qquad \epsilon\colon \mathcal{F} \circ \mathcal{G} \to \id_{\mathsf{D}}.
      \end{equation*}
  \end{enumerate}
  The reason for doing things in this order is that we will use the component-wise triangle identities in showing naturality.
\end{proof}
\begin{proof}
  \leavevmode
  \begin{enumerate}
    \item Suppose $\mathcal{F}$ and $\mathcal{G}$ form a hom-set adjunction with natural isomorphism $\Phi$. Then for any $A \in \Obj(\mathsf{C})$, we have $\mathcal{F}(A) \in \Obj(\mathsf{D})$, so $\Phi$ give us a bijection
      \begin{equation*}
        \Phi_{A, \mathcal{F}(A)}\colon \Hom_{\mathsf{D}}(\mathcal{F}(A), \mathcal{F}(A)) \to \Hom_{\mathsf{C}}(A, (\mathcal{G} \circ \mathcal{F})(A)).
      \end{equation*}
      We don't know much in general about $\Hom_{\mathsf{D}}(\mathcal{F}(A), \mathcal{F}(A))$, but the category axioms tell us that it always contains $\id_{\mathcal{F}(A)}$. We can use $\Phi_{A, \mathcal{F}(A)}$ to map this to
      \begin{equation*}
        \Phi_{A, \mathcal{F}(A)}(\id_{\mathcal{F}(A)}) \in \Hom_{\mathsf{C}}(A, (\mathcal{G} \circ \mathcal{F})(A)).
      \end{equation*}
      Let's call $\Phi_{A, \mathcal{F}(A)}(\id_{\mathcal{F}(A)}) = \eta_{A}$.

      Similarly, if $B \in \Obj(\mathsf{D})$, then $\mathcal{G}(B) \in \Obj(\mathsf{C})$, so $\Phi$ gives us a bijection
      \begin{equation*}
        \Phi_{\mathcal{G}(B), B}\colon \Hom_{\mathsf{D}}((\mathcal{F} \circ \mathcal{G})(B), B) \to \Hom_{\mathsf{C}}(\mathcal{G}(B), \mathcal{G}(B)).
      \end{equation*}

      Since $\Phi_{\mathcal{G}(B), B}$ is a bijection, it is invertible, and we can evaluate the inverse on $\id_{\mathcal{G}(B)}$. Let's call
      \begin{equation*}
        \Phi^{-1}_{\mathcal{G}(B), B}(\id_{\mathcal{G}(B)}) = \varepsilon_{B}.
      \end{equation*}

    \item Clearly, $\eta_{A}$ and $\varepsilon_{B}$ are completely determined by $\Phi$ and $\Phi^{-1}$ respectively. It turns out that the converse is also true; in a manner reminiscent of the proof of the Yoneda lemma, we can express $\Phi$ in terms of $\eta$, and $\Phi^{-1}$ in terms of $\varepsilon$, for \emph{any} $A$ and $B$. We do this using the naturality of $\Phi$.

      The naturality of $\Phi$ tells us for any $A \in \Obj(\mathsf{C})$, $B \in \Obj(\mathsf{D})$, and $g\colon \mathcal{F}(A) \to B$, the following diagram has to commute.
      \begin{equation*}
        \begin{tikzcd}[row sep=huge, column sep=huge]
          \Hom_{\mathsf{D}}(\mathcal{F}(A), \mathcal{F}(A))
          \arrow[r, "g \circ (-)"]
          \arrow[d, swap, "\Phi_{A, \mathcal{F}(A)}"]
          & \Hom_{\mathsf{D}}(\mathcal{F}(A), B)
          \arrow[d, "\Phi_{A, B}"]
          \\
          \Hom_{\mathsf{C}}(A, (\mathcal{G} \circ \mathcal{F})(A))
          \arrow[r, "\mathcal{G}(g) \circ (-)"]
          & \Hom_{\mathsf{C}}(A, \mathcal{G}(B)).
        \end{tikzcd}
      \end{equation*}
      Let's start at the top left with $\id_{\mathcal{F}(A)}$ and see what happens. Taking the top road to the bottom right, we have $\Phi_{A, B}(g)$, and from the bottom road we have $\mathcal{G}(g) \circ \eta_{A}$. The diagram commutes, so we have
      \begin{equation}
        \label{eq:eta_triangle}
        \Phi_{A, B}(g) = \mathcal{G}(g) \circ \eta_{A}.
      \end{equation}
      Similarly, the commutativity of the diagram
      \begin{equation*}
        \begin{tikzcd}[row sep=huge, column sep=huge]
          \Hom_{\mathsf{D}}((\mathcal{F} \circ \mathcal{G})(B), B)
          \arrow[r, "(-) \circ \mathcal{F}(f)"]
          & \Hom_{\mathsf{D}}(\mathcal{F}(A), B)
          \\
          \Hom_{\mathsf{C}}(\mathcal{G}(B), \mathcal{G}(B))
          \arrow[r, "(-) \circ f"]
          \arrow[u, "{\Phi^{-1}_{\mathcal{G}(B), B}}"]
          & \Hom_{\mathsf{C}}(A, \mathcal{G}(B))
          \arrow[u, swap, "\Phi^{-1}_{A, B}"]
        \end{tikzcd}
      \end{equation*}
      means that, for any $f\colon A \to \mathcal{G}(B)$,
      \begin{equation}
        \label{eq:epsilon_triangle}
        \Phi^{-1}_{A, B}(f) = \varepsilon_{B} \circ \mathcal{F}(f)
      \end{equation}

      To show that $\eta$ and $\varepsilon$ as defined here satisfy the triangle identities, we need to show that for all $A \in \Obj(\mathsf{C})$ and all $B \in \Obj(\mathsf{D})$,
      \begin{equation*}
        {(\varepsilon\mathcal{F})}_{A} \circ {(\mathcal{F}\eta)}_{A} = {(\id_{\mathcal{F}})}_{A}\qquad\text{and}\quad {(\mathcal{G}\varepsilon)}_{B} \circ {(\eta\mathcal{G})}_{B} = {(\id_{\mathcal{G}})}_{B}.
      \end{equation*}
      We have
      \begin{equation*}
        {(\varepsilon\mathcal{F})}_{A} \circ {(\mathcal{F}\eta)}_{A} = \varepsilon_{\mathcal{F}(A)} \circ \mathcal{F}(\eta_{A}) = \Phi^{-1}_{A, \mathcal{F}(A)}(\eta_{A}) = \id_{A} = {(\id_{\mathcal{F}})}_{A}
      \end{equation*}
      and
      \begin{equation*}
        {(\mathcal{G}_{\varepsilon})}_{B} \circ {(\eta\mathcal{G})}_{B} = \mathcal{G}(\varepsilon_{B}) \circ \eta_{\mathcal{G}(B)} = \Phi_{\mathcal{G}(B), B}(\varepsilon_{B}) = \id_{B} = {(\id_{\mathcal{G}})}_{B}.
      \end{equation*}

    \item Let $g\colon A' \to A$. By the naturality of $\Phi$, the following diagram commutes.
      \begin{equation*}
        \begin{tikzcd}[column sep=tiny]
          \Hom_{\mathsf{D}}(\mathcal{F}(A), \mathcal{F}(A))
          \arrow[rrr, "(-) \circ \mathcal{F}(g)"]
          \arrow[ddd, swap, "\Phi_{A, \mathcal{F}(A)}"]
          &&& \Hom_{\mathsf{D}}(\mathcal{F}(A'), \mathcal{F}(A))
          \arrow[ddd, "\Phi_{A', \mathcal{F}(A)}"]
          \\
          & \id_{\mathcal{F}(A)}
          \arrow[r, mapsto]
          \arrow[d, mapsto]
          & \mathcal{F}(g)
          \arrow[d, mapsto]
          \\
          & \eta_{A}
          \arrow[r, mapsto]
          & \eta_{A} \circ g \overset{!}{=} \Phi_{A', \mathcal{F}_{A}}(\mathcal{F}(g))
          \\
          \Hom_{\mathsf{C}}(A, (\mathcal{G} \circ \mathcal{F})(A))
          \arrow[rrr, "(-) \circ g"]
          &&& \Hom_{\mathsf{C}}(A', (\mathcal{G} \circ \mathcal{F})(A))
        \end{tikzcd}
      \end{equation*}

      Following $\id_{\mathcal{F}(A)}$ around, we find the condition $\eta_{A} \circ g \overset{!}{=} \Phi_{A', \mathcal{F}_{A}}(\mathcal{F}(g))$. But by \hyperref[eq:eta_triangle]{Equation~\ref*{eq:eta_triangle}}, we have
      \begin{equation*}
        \Phi_{A', \mathcal{F}(A)}(\mathcal{F}(g)) = (\mathcal{G} \circ \mathcal{F})(g) \circ \eta_{A'},
      \end{equation*}
      so our condition reads
      \begin{equation*}
        \eta_{A} \circ g = (\mathcal{G} \circ \mathcal{F})(g) \circ \eta_{A'}.
      \end{equation*}
      But this says precisely that the naturality square
      \begin{equation*}
        \begin{tikzcd}
          (\mathcal{G} \circ \mathcal{F})(A)
          \arrow[r, "\eta_{A}"]
          \arrow[d, swap, "(\mathcal{G} \circ \mathcal{F})(f)"]
          & A
          \arrow[d, "f"]
          \\
          (\mathcal{G} \circ \mathcal{F})(A')
          \arrow[r, "\eta_{A'}"]
          & A'
        \end{tikzcd}
      \end{equation*}
      commutes.

  \end{enumerate}
\end{proof}

\begin{lemma}
  \label{lemma:unit-counit_adjunction_implies_hom-set_adjunction}
  If the functors $\mathcal{F}\colon \mathsf{C} \to \mathsf{D}$ and $\mathcal{G}\colon \mathsf{D} \to \mathsf{C}$ form a unit-counit adjunction, they form a hom-set  adjunction.
\end{lemma}
\begin{proof}
  We have seen in the proof of the previous lemma that we can write
  \begin{equation*}
    \Phi_{A, B}(g) = \mathcal{G}(g) \circ \eta_{A}\qquad\text{and}\qquad \Phi^{-1}_{A,B}(g) = \epsilon_{B} \circ \mathcal{F}(f).
  \end{equation*}
\end{proof}

\begin{definition}[adjunct]
  \label{def:adjunct}
  Let $\mathcal{F} \dashv \mathcal{G}$ be an adjunction as follows.
  \begin{equation*}
    \begin{tikzcd}
      \mathsf{C}
      \arrow[r, bend left, rightarrow, "\mathcal{F}"]
      & \mathsf{D}
      \arrow[l, bend left, rightarrow, "\mathcal{G}"]
    \end{tikzcd}
  \end{equation*}
  Then for each $A \in \Obj(\mathsf{C})$ $B \in \Obj(\mathsf{D})$, we have a natural isomorphism (i.e.\ a bijection)
  \begin{equation*}
    \Phi_{A, B}: \Hom_{\mathsf{D}}(\mathcal{F}(A), B) \to \Hom_{\mathsf{C}}(A, \mathcal{G}(B)).
  \end{equation*}
  Thus, for each $f \in \Hom_{\mathsf{D}}(\mathcal{F}(A), B)$ there is a corresponding element $\tilde{f} \in \Hom_{\mathsf{C}}(A, \mathcal{G}(B))$, and vice versa. The morphism $\tilde{f}$ is called the \defn{adjunct} of $f$, and $f$ is called the adjunct of $\tilde{f}$.
\end{definition}

\begin{lemma}
  Let $\mathsf{C}$ and $\mathsf{D}$ be categories, $\mathcal{F}\colon \mathsf{C} \rightarrow \mathsf{D}$ and $\mathcal{G}$, $\mathcal{G}'\colon \mathsf{D} \rightarrow \mathsf{C}$ functors,
  \begin{equation*}
    \begin{tikzcd}
      \mathsf{C}
      \arrow[r, rightarrow, "\mathcal{F}"]
      & \mathsf{D}
      \arrow[l, rightarrow, bend right=45, swap, "\mathcal{G}"]
      \arrow[l, rightarrow, bend left=45, "\mathcal{G}'"]
    \end{tikzcd}
  \end{equation*}
  and suppose that $\mathcal{G}$ and $\mathcal{G}'$ are both right-adjoint to $\mathcal{F}$. Then there is a natural isomorphism $\mathcal{G} \Rightarrow \mathcal{G}'$.
\end{lemma}
\begin{proof}
  Since any adjunction is a hom-set adjunction, we have two isomorphisms
  \begin{equation*}
    \Phi_{C, D}\colon \Hom_{\mathsf{D}}(\mathcal{F}(C), D) \Rightarrow \Hom_{\mathsf{C}}(C, \mathcal{G}(D))\qquad\text{and}\qquad \Psi_{C, D}\colon \Hom_{\mathsf{D}}(\mathcal{F}(C), D) \Rightarrow \Hom_{\mathsf{C}}(C, \mathcal{G}'(D))
  \end{equation*}
  which are natural in both $C$ and $D$. By \hyperref[lemma:naturalisomorphismshaveinverses]{Lemma~\ref*{lemma:naturalisomorphismshaveinverses}}, we can construct the inverse natural isomorphism
  \begin{equation*}
    \Phi^{-1}_{C, D}\colon \Hom_{\mathsf{C}}(C, \mathcal{G}(D)) \Rightarrow \Hom_{\mathsf{D}}(\mathcal{F}(C), D),
  \end{equation*}
  and compose it with $\Psi$ to get a natural isomorphsim
  \begin{equation*}
    {(\Psi \circ \Phi^{-1})}_{C, D}\colon \Hom_{\mathsf{C}}(C, \mathcal{G}(D)) \Rightarrow \Hom_{\mathsf{C}}(C, \mathcal{G}'(D)).
  \end{equation*}

  Thus for any morphism $f\colon D \to E$, the following diagram commutes.
  \begin{equation*}
    \begin{tikzcd}[row sep=huge, column sep=huge]
      \Hom_{\mathsf{C}}(C, \mathcal{G}(D))
      \arrow[r, "{\Hom_{\mathsf{C}}(C, \mathcal{G}(f))}"]
      \arrow[d, swap, "{{(\Psi \circ \Phi^{-1})}_{C, D}}"]
      & \Hom_{\mathsf{C}}(C, \mathcal{G}(E))
      \arrow[d, "{{(\Psi \circ \Phi^{-1})}_{C, E}}"]
      \\
      \Hom_{\mathsf{C}}(C, \mathcal{G}'(D))
      \arrow[r, "{\Hom_{\mathsf{C}}(C, \mathcal{G}'(f))}"]
      & \Hom_{\mathsf{C}}(C, \mathcal{G}'(E))
    \end{tikzcd}
  \end{equation*}
  But the full- and faithfulness of the Yoneda embedding tells us that there is a natural isomorphism with components $\mu_{D}$ and $\mu_{E}$ making the following diagram commute.
  \begin{equation*}
    \begin{tikzcd}
      \mathcal{G}(D)
      \arrow[r, "\mathcal{G}(f)"]
      \arrow[d, swap, "\mu_{D}"]
      & \mathcal{G}(E)
      \arrow[d, "\mu_{E}"]
      \\
      \mathcal{G}'(D)
      \arrow[r, "\mathcal{G}'(f)"]
      & \mathcal{G}'(E)
    \end{tikzcd}
  \end{equation*}
\end{proof}

\begin{note}
  \label{note:freeforgetfuladjunctions}
  We do not give very many examples of adjunctions now because of the frequency with which category theory graces us with them. Howver, it is worth mentioning a specific class of adjunctions: the so-called \emph{free-forgetful adjunctions}. There are many \emph{free} objects in mathematics: free groups, free modules, free vector spaces, free categories, etc. These are all unified by the following property: the functors defining them are all left adjoints.

  Let us take a specific example: the free vector space over a set. This takes a set $S$ and constructs a vector space which has as a basis the elements of $S$.

  There is a forgetful functor $\mathcal{U}\colon \mathsf{Vect}_{k} \rightarrow \mathsf{Set}$ which takes any set and returns the set underlying it. There is a functor $\mathcal{F}\colon \mathsf{Set} \rightarrow \mathsf{Vect}_{k}$, which takes a set and returns the free vector space on it. It turns out that there is an adjunction $\mathcal{F} \dashv \mathcal{U}$.

  And this is true of any free object! (In fact by definition.) In each case, the functor giving the free object is left adjoint to a forgetful functor.
\end{note}

\begin{theorem}
  \label{thm:rightadjointspreservelimits}
  Let $\mathsf{C}$ and $\mathsf{D}$ be categories and $\mathcal{F}$ and $\mathcal{G}$ functors as follows.
  \begin{equation*}
    \begin{tikzcd}
      \mathsf{C}
      \arrow[r, rightarrow, bend left, "\mathcal{F}"]
      & \mathsf{D}
      \arrow[l, rightarrow, bend left, "\mathcal{G}"]
    \end{tikzcd}
  \end{equation*}
  Let $\mathcal{F} \dashv \mathcal{G}$ be an adjunction. Then $\mathcal{G}$ preserves limits, i.e.\ if $\mathcal{D}\colon \mathsf{J} \to \mathsf{C}$ is a diagram and $\lim_{\leftarrow i}\mathcal{D}_{i}$ exists in $\mathsf{C}$, then
  \begin{equation*}
    \mathcal{G}(\lim_{\leftarrow i}\mathcal{D}_{i}) \simeq \lim_{\leftarrow i} (\mathcal{G} \circ \mathcal{D}_{i}).
  \end{equation*}
\end{theorem}
\begin{proof}
  We have the following chain of isomorphisms, natural in $Y \in \Obj(\mathsf{D})$.
  \begin{align*}
    \Hom_{\mathsf{D}}(Y, \mathcal{G}(\lim_{\leftarrow i}\mathcal{D}_{i})) &\simeq \Hom_{\mathsf{C}}(\mathcal{F}(Y), \lim_{\leftarrow i}\mathcal{D}_{i}) \\
    &\simeq \lim_{\leftarrow i} \Hom_{\mathsf{C}}(\mathcal{G}(Y), \mathcal{D}_{i}) &\left(\substack{\text{Hom functor commutes with} \\ \text{limits: \hyperref[thm:homfunctorpreserveslimits]{Theorem~\ref*{thm:homfunctorpreserveslimits}}}}\right) \\
    &\simeq \lim_{\leftarrow i} \Hom_{\mathsf{D}}(Y, \mathcal{G}\circ \mathcal{D}_{i}) \\
    &\simeq \Hom_{\mathsf{C}}(Y, \lim_{\leftarrow i}(\mathcal{G}\circ \mathcal{D}_{i})).
  \end{align*}

  By the Yoneda lemma, specifically \hyperref[cor:yonedaembeddingrespectsisomorphisms]{Corollary~\ref*{cor:yonedaembeddingrespectsisomorphisms}}, we have a natural isomorphism
  \begin{equation*}
    \mathcal{G}(\lim_{\leftarrow i}\mathcal{D}_{i}) \simeq \lim_{\leftarrow i}(\mathcal{G} \circ \mathcal{D}_{i}).
  \end{equation*}
\end{proof}

\begin{corollary}
  \label{cor:leftadjointspreservecolimits}
  Any functor $\mathcal{F}$ which is a left-ajoint preserves colimits.
\end{corollary}
\begin{proof}
  Dual to the proof of \hyperref[thm:rightadjointspreservelimits]{Theorem~\ref*{thm:rightadjointspreservelimits}}.
\end{proof}

\section{Adjunctions generaize categorical equivalences}
\label{sec:adjunctions_generaize_categorical_equivalences}

\begin{theorem}
  Let $\mathcal{F}\colon \mathsf{C} \to \mathsf{D}$ be a functor. The following are equivalent.
  \begin{enumerate}
    \item $\mathcal{F}$ is part of an equivalence of categories (\hyperref[def:categoricalequivalence]{Definition~\ref*{def:categoricalequivalence}}) $(\mathcal{F}, \mathcal{G}, \eta, \epsilon)$.

    \item $\mathcal{F}$ is fully faithful (\hyperref[def:essentiallysurjective]{Definition~\ref*{def:essentiallysurjective}}) and essentially surjective (\hyperref[def:fullfaithfulfunctor]{Definition~\ref*{def:fullfaithfulfunctor}}).

    \item $\mathcal{F}$ is part of a unit-counit adjunction (\hyperref[def:unitcounitadjunction]{Definition~\ref*{def:unitcounitadjunction}}) $(\mathcal{F}, \mathcal{G}, \epsilon, \eta)$, where $\epsilon$ and $\eta$ are isomorphisms.
  \end{enumerate}
\end{theorem}
\begin{proof}
  \leavevmode
  \begin{enumerate}
    \item[3. $\Rightarrow$ 1.] Obvious.

    \item[1. $\Rightarrow$ 2.] Assume $(\mathcal{F}, \mathcal{G}, \eta\colon \mathcal{F} \circ \mathcal{G} \to \id_{G}, \epsilon\colon \id_{C} \to \mathcal{G} \circ \mathcal{F})$ is an equivalence of categories.

      First we show that $\mathcal{F}$ is essentially surjective. Evaluating $\eta$ at any $Y \in \Obj(\mathsf{D})$, we get an isomorphism $(\mathcal{F} \circ \mathcal{G})(Y) \simeq Y$. Thus, for any $Y \in \Obj(\mathsf{C})$, there is an isomorphism between $Y$ and some element $\mathcal{F}(\mathcal{G}(Y))$ in the image of $\mathcal{F}$.

      Next, we show that $\mathcal{F}$ is fully faithful.

    \item[2. $\Rightarrow$ 3.]
  \end{enumerate}
\end{proof}

\section{Important examples}
\label{sec:important_examples}

\begin{example}[limit-constant adjunction]
  Let $\mathsf{J}$ and $\mathsf{C}$ be small categories. Denote by $\Delta\colon \mathsf{C} \to [\mathsf{I}, \mathsf{C}]$ the functor such that for all $X \in \Obj(\mathsf{C})$, $\Delta(X)$ is the constant functor which sends every object of $\mathsf{I}$ to $X$ and every morphism of $\mathsf{J}$ to $\id_{X}$.

  Assume that every diagram $\mathcal{F}\colon \mathsf{J} \to \mathsf{C}$ has a limit. By TODO, this means that for every $\mathcal{D} \in [\mathsf{J}, \mathsf{C}]$, the comma category $(\Delta \downarrow \mathcal{D})$ has a terminal object. But this means that there is an adjoint functor $[\mathsf{J}, \mathsf{C}] \to \mathsf{C}$ which acts on objects by sending $\mathcal{D} \mapsto \lim_{\leftarrow} \mathcal{D}$.
\end{example}

Keep this example in mind during the chapter on Kan extensions.

\end{document}
