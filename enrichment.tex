\documentclass[notes.tex]{subfiles}

\begin{document}

\chapter{Enrichment}
\label{ch:enrichment}

\begin{definition}[enriched category]
  \label{def:enriched_category}
  Let $(\mathsf{M}, \otimes, 1_{\mathsf{M}}, \alpha, \lambda, \rho)$ be a monoidal category. A \defn{$\mathsf{M}$-enriched category} $\mathsf{C}$ consists of the following data.
  \begin{itemize}
    \item A set $\Obj(\mathsf{C})$ of objects.

    \item For each pair $x$, $y$ of objects, an object $\mathsf{C}(x, y) \in \mathsf{M}$ of morphisms.

    \item For each object $x \in \mathsf{C}$, a morphism
      \begin{equation*}
        1_{\mathsf{M}} \to \mathsf{C}(x, x)
      \end{equation*}
      called the \emph{unit}

    \item For objects $x$, $y$, and $z$, a morphism
      \begin{equation*}
        \mathsf{C}(x, y) \otimes \mathsf{C}(y, z) \to \mathsf{C}(x, z)
      \end{equation*}
      called the \emph{composition}.
  \end{itemize}
  This composition must be unital and associative, i.e.\ the diagrams
  \begin{equation*}
    \begin{tikzcd}
      \mathsf{C}(x, y)
      \arrow[r]
      \arrow[d]
      \arrow[rd, "\id"]
      & \mathsf{C}(y, y) \otimes \mathsf{C}(x, y)
      \arrow[d]
      \\
      \mathsf{C}(x, y) \otimes \mathsf{C}(x, x)
      \arrow[r]
      & \mathsf{C}(x, y)
    \end{tikzcd}
  \end{equation*}
  and
  \begin{equation*}
    \begin{tikzcd}
      \mathsf{C}(x, y) \otimes \mathsf{C}(y, z) \otimes \mathsf{C}(z, w)
      \arrow[r]
      \arrow[d]
      & \mathsf{C}(x, y) \otimes \mathsf{C}(y, w)
      \arrow[d]
      \\
      \mathsf{C}(x, z) \otimes \mathsf{C}(z, w)
      \arrow[r]
      & \mathsf{C}(x, w)
    \end{tikzcd}
  \end{equation*}
  must commute (where unitors and associators are notationally supressed).
\end{definition}

\begin{lemma}
  \label{lemma:monoidal_functor_switches_enrichment}
  Let $\mathcal{C}$ be an $\mathcal{M}$-enriched category, and let $F\colon \mathcal{M} \to \mathcal{N}$ be a monoidal functor with data
  \begin{equation*}
    \Phi_{x, y}\colon F(x) \otimes_{\mathcal{N}} F(y) \to F(x \otimes_{\mathcal{M}} y),\qquad \phi\colon 1_{\mathcal{N}} \to F(1_{\mathcal{M}}).
  \end{equation*}
  This data allows us to build from $\mathcal{C}$ a category enriched over $\mathcal{N}$.
\end{lemma}
\begin{proof}
  \leavevmode
  \begin{itemize}
    \item We apply $F$ to each hom-object in $\mathcal{M}$ to get a new hom-object in $\mathcal{N}$;
      \begin{equation*}
        \mathcal{C}(x, y)_{\mathcal{N}} = F(\mathcal{C}(x, y)_{\mathcal{M}}).
      \end{equation*}

    \item The unit is the composition
      \begin{equation*}
        \begin{tikzcd}
          1_{\mathcal{N}}
          \arrow[r]
          & F(1_{\mathcal{M}})
          \arrow[r]
          & \mathcal{F}(\mathcal{C}(x, y)_{\mathcal{M}})
          \arrow[r, equals]
          & \mathcal{C}(x, y)_{\mathcal{N}}
        \end{tikzcd}
      \end{equation*}

    \item The composition is given by the obvious.
  \end{itemize}
\end{proof}
\end{document}
