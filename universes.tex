\documentclass[notes.tex]{subfiles}

\begin{document}

\chapter{Foundational issues}
\label{ch:_foundational_issues}

\section{Grothendieck universes}
\label{sec:grothendieck_universes}

\begin{note}
  I want to edit this section pretty seriously as I don't really understand universes at the moment. Furthermore, since most of these notes was written in the language of proper classes, there are probably still a few latent references to them. I'll remove these references as I find them.
\end{note}

Set theory as formulated in ZFC has foundational annoyances. One particularly pernicious one says that not every definable collection of objects is small enough to be a set. For example, there is no set of all sets, nor set of all groups, rings, etc.

Category theory is to a large extent built upon ZFC, and needs to talk about the collection of all sets, groups, rings, and many other structures. When doing category theory, there are three common ways to deal with collections which are too large to be sets.
\begin{enumerate}
  \item One defines the notion of a \emph{proper class,} which is simply another word for a definable collection too large to be a set, and works with proper classes in addition to sets.

  \item Following Grothendieck, one can add an additional axiom, called the \emph{universe axiom,} to ZFC which ensures the existence a set of all sets one is allowed to use. This set of all sets one is allowed to use is called a \emph{Grothendieck universe.}

  \item Introduce categories, rather than sets, as foundational objects.
\end{enumerate}

The first approach is the most common, and the third the most radical. We will use the second here.

\begin{definition}[Grothendieck universe]
  \label{def:universe}
  A \defn{Grothendieck universe} (or simply \emph{universe}) is a set $\mathcal{U}$ with the following properties:
  \begin{enumerate}[label=\emph{U\arabic*.}]
    \item If $x \in \mathcal{U}$ and $y \in x$, then $y \in \mathcal{U}$. That is, if if $x \in \mathcal{U}$, then $x$ is a subset of $\mathcal{U}$.

    \item If $x$, $y \in \mathcal{U}$, then $\{x, y\} \in \mathcal{U}$

    \item If $x \in \mathcal{U}$, then $2^{X}$ is also in $\mathcal{U}$.

    \item If $I \in \mathcal{U}$ and $\{x_{i}\}_{i \in U}$ with $x_{i} \in \mathcal{U}$, then
      \begin{equation*}
        \bigcup_{i \in I} x_{i} \in \mathcal{U}.
      \end{equation*}
  \end{enumerate}
\end{definition}

\begin{definition}[universe axiom]
  \label{def:universe_axiom}
  The \defn{universe axiom} says that for every set $X$, there is a universe which has $X$ as an element.
\end{definition}

These axioms imply that universes are closed under many of the properties one would hope.\footnote{See the below lemma.} The upshot is that all familiar constructions from everyday life can be performed within any universe $\mathcal{U}$. From now on, we fix a universe $\mathcal{U}$ such that $\N \in \mathcal{U}$ (which is possible by the universe axiom).

\begin{lemma}
  \label{lemma:properties_of_universes}
  Grothendieck universes have the following properties.
  \begin{enumerate}
    \item $x \in \mathcal{U} \implies \{x\} \in \mathcal{U}$

    \item $y \in \mathcal{U}$, $x \subseteq y \implies x \in \mathcal{U}$

    \item $x$, $y \in \mathcal{U} \implies (x, y) \equiv \{x, \{x, y\}\} \in \mathcal{U}$

    \item $x$, $y \in \mathcal{U} \implies x \cup y$, $x \times y \in \mathcal{U}$.

    \item $x$, $y \in \mathcal{U} \implies \mathrm{Maps}(x, y) \in \mathcal{U}$

    \item $I \in \mathcal{U}$, $\{x_{i}\}_{i \in J}$ with $x_{i} \in \mathcal{U} \implies \prod_{i \in J} x_{i}$, $\coprod_{i \in J} x_{i}$, $\bigcap_{i \in J} x_{i} \in \mathcal{U}$.

    \item $x \in \mathcal{U} \implies x \cup \{x\} \in \mathcal{U}$, hence $\N \subset \mathcal{U}$.
  \end{enumerate}
\end{lemma}
\begin{proof}
  \leavevmode
  \begin{enumerate}
    \item If $x$ in $\mathcal{U}$, then by \emph{U2} $\{x, x\} \in \mathcal{U}$. But $\{x, x\} = \{x\}$.

    \item If $y \in \mathcal{U}$, then $2^{y} \in \mathcal{U}$ by \emph{U3.} Any subset of $y$ is an element of $2^{y}$, so \emph{U1} implies that any subset of $y$ belongs to $\mathcal{U}$.

    \item If $x$ and $y \in \mathcal{U}$, then $\{x, y\} \in \mathcal{U}$ by \emph{U1.} Then again by \emph{U1,} $\{x, \{x, y\}\} \in \mathcal{U}$.

    \item For any $a$, $b \in \mathcal{U}$, $a \neq b$, we have by \emph{U2} that $\{a, b\} \in \mathcal{U}$. Thus, binary unions follow from \emph{U4,} which demands the existence of unions indexed by any set in $\mathcal{U}$. 
      
      From \emph{U2.}, if $x$, $y \in \mathcal{U}$ then every element of $x$ and every element of $y$ is in $U$, so ordered pairs of an element of $x$ and an element of $y$ are in $\mathcal{U}$. Thus, the singletons each containing an ordered pair are in $\mathcal{U}$, so the union
      \begin{equation*}
        \bigcup_{\alpha \in x} \bigcup_{\beta \in y} \{(\alpha, \beta)\}
      \end{equation*}
      is in $\mathcal{U}$.

    \item The set of all maps $x \to y$ is a subset of the set $2^{x \times y}$.

    \item As a set, 
      \begin{equation*}
        \prod_{i \in I} x_{i} = \left\{f\colon I \to \bigcup_{i \in I} x_{i}\ \Big|\ f(i) \in x_{i} \text{ for all }i \in I\right\} \subset \mathrm{Maps}\left(I, \bigcup_{i \in I} x_{i}\right)
      \end{equation*}

      As a set,
      \begin{equation*}
        \coprod_{i \in I} x_{i} = \bigcup_{i \in I} x_{i} \times \{i\}.
      \end{equation*}

      Intersection is subset of union.

    \item If $x \in \mathcal{U}$ then $\{x\} \in \mathcal{U}$ by 1., so by 4., $x \cup \{x\} \in \mathcal{U}$.
  \end{enumerate}
\end{proof}

\begin{definition}[small set, class, large set]
  \label{def:small_set_class_large_set}
  We make the following definitions.
  \begin{itemize}
    \item We call the elements of $\mathcal{U}$ \emph{small sets}. 
    \item We call subsets of $\mathcal{U}$ \emph{classes}.
    \item We call everything else \emph{Large sets}.
  \end{itemize}

  In general, small sets agree with the usual (i.e.\ ZFC) notion of sets, and classes agree with the usual notion of proper classes. In what follows, we will write \emph{set} when we mean \emph{small set,} unless confusion is likely to arise. Unfortunately, especially when talking about foundational issues, confusion is likely to arise.
\end{definition}

By the universe axiom, for any set (as defined in ZFC), there is a universe which has that set as an element. From now on, we will work in some universe $\mathcal{U}$ which has $\N$ as an element.

\end{document}
