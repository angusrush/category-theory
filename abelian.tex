\documentclass[main.tex]{subfiles}

\begin{document}

\chapter{Abelian categories}\label{sec:abeliancategories}

This section draws heavily from~\cite{EGNO-tensor-categories}.

\section{Additive categories}

Recall that $\mathsf{Ab}$ is the category of abelian groups.
\begin{definition}[$\mathsf{Ab}$-enriched category]
  \label{def:abenrichedcategory}
  A category $\mathsf{C}$ is \defn{$\mathsf{Ab}$-enriched} if
  \begin{enumerate}
    \item for all objects $A$, $B \in \Obj(\mathsf{C})$, the hom-set $\text{Hom}_{\mathsf{C}}(A, B)$ has the structure of an abelian group (i.e.\ one can add morphisms), such that

    \item the composition
      \begin{equation*}
        \circ\colon \Hom_{\mathsf{C}}(B, C) \times \Hom_{\mathsf{C}}(A, B) \to \Hom_{C}(A, C)
      \end{equation*}
      is additive in each slot: for any $f_{1}$, $f_{2} \in \Hom_{\mathsf{C}}(B, C)$ and $g \in \Hom_{\mathsf{C}}(A, B)$, we must have
      \begin{equation*}
        (f_{1} + f_{2}) \circ g = f_{1} \circ g + f_{2} \circ g,
      \end{equation*}
      and similarly in the second slot.
  \end{enumerate}

  That is to say, $\mathsf{Ab}$-enriched categories are categories enriched over $(\mathsf{Ab}, \otimes_{\Z}, 0)$.
\end{definition}

\begin{note}
  \label{note:inabenrichedcategorynoemptyhomsets}
  In any $\mathsf{Ab}$-enriched category, every hom-set has at least one element---the identity element of the hom-set taken as an abelian group.
\end{note}

\begin{lemma}
  \label{lemma:abeliancategorycoproductsareproducts}
  Let $\mathsf{C}$ be an $\mathsf{Ab}$-enriched category with finite products. Then the product $X_{1} \times \cdots \times X_{n}$ satisfies the universal property for coproducts. In particular, initial objects and terminal objects coincide.
\end{lemma}
\begin{proof}
  First, we show that initial and terminal objects coincide. Let $\emptyset$ be initial and $*$ terminal in $\mathsf{C}$.

  Since $\Hom_{\mathsf{C}}(\emptyset, \emptyset)$ has only one element, it must be both the identity morphism on $\emptyset$ \emph{and} the identity element of $\Hom_{\mathsf{C}}(\emptyset, \emptyset)$ as an abelian group. Similarly, $\id_{*} = 0$ in $\Hom_{\mathsf{C}}(*, *)$.

  Since $\emptyset$ is initial and $*$ is final, $\Hom_{\mathsf{C}}(\emptyset, *)$ has exactly one element $f$. By \hyperref[note:inabenrichedcategorynoemptyhomsets]{Note \ref*{note:inabenrichedcategorynoemptyhomsets}}, we are guaranteed that $\Hom_{\mathsf{C}}(*, \emptyset)$ has at least one element, $g$. Then $g \circ f = 0 = \id$ and $f \circ g = 0 = \id$, so $f\colon \emptyset \to *$ is an isomorphism. Thus, initial and terminal objects coincide, so $\mathsf{C}$ has a zero object $0$.

  We prove the lemma for binary products. The general case is preciesly the same.

  For each $X$, $Y \in \mathcal{C}$, we can define $i_{X}\colon X \to X \times Y$ using the universal property for products as follows.
  \begin{equation*}
    \begin{tikzcd}
      & X
      \arrow[dl, swap, "\id"]
      \arrow[d, dashed, "i_{X}"]
      \arrow[dr, "0"]
      \\
      X
      & X \times Y
      \arrow[l]
      \arrow[r]
      & Y
    \end{tikzcd}
  \end{equation*}
  Similarly, get $i_{Y}\colon Y \to X \times Y$.

  The claim is that $X \times Y$, together with $i_{X}$ and $i_{Y}$, satisfies the universal property for coproducts; that is, that we can find a unique map $\phi$ making the below diagram commute.
  \begin{equation*}
    \begin{tikzcd}
      X
      \arrow[r, "i_{X}"]
      \arrow[dr, swap, "f"]
      & X \times Y
      \arrow[d, dashed, "\phi"]
      & Y
      \arrow[l, swap, "i_{Y}"]
      \arrow[dl, "g"]
      \\
      & Z
    \end{tikzcd}
  \end{equation*}
  In fact, $\phi = f \circ \pi_{X} + g \circ \pi_{Y}$ satisfies the universal property. It is clear that this makes the diagram commute since $\pi_{X} \circ i_{X} = \id$, $\pi_{X} \circ i_{Y} = 0$, and similarly for $Y$.

  Note that, by the universal property for products, there is a unique downward map. But both $\id_{X \times Y}$ and $i_{X} \circ \pi_{X} + i_{Y} \circ \pi_{Y}$ make it commute, so they must be equal.
  \begin{equation*}
    \begin{tikzcd}
      & X \times Y
      \arrow[dr, "\pi_{X}"]
      \arrow[dl, swap, "\pi_{Y}"]
      \arrow[d]
      \\
      X
      & X \times Y
      \arrow[r, swap, "\pi_{Y}"]
      \arrow[l, "\pi_{X}"]
      & Y
    \end{tikzcd}
  \end{equation*}

  Now we are ready to show that $\phi$ is unique. Suppose we have another candidate $\psi$ which also makes the above diagram commute. Then
  \begin{equation*}
    \psi = \psi \circ \id_{X \times Y} = f \circ \pi_{X} + g \circ \pi_{Y} = \phi.
  \end{equation*}
\end{proof}

\begin{definition}[additive category]
  \label{def:additivecategory}
  A category $\mathsf{C}$ is \defn{additive} if it has products and is $\mathsf{Ab}$-enriched.
\end{definition}

\begin{example}
  The category $\mathsf{Ab}$ of Abelian groups is an additive category. We have already seen that it has the direct sum $\oplus$ as biproduct. Given any two abelian groups $A$ and $B$ and morphisms $f$, $g\colon A \to B$, we can define the sum $f + g$ via
  \begin{equation*}
    (f + g)(a) = f(a) + g(a)\qquad\text{for all }a \in A.
  \end{equation*}
  Then for another abelian group $C$ and a morphism $h\colon B \to C$, we have
  \begin{equation*}
    \left[ h \circ (f+g) \right](a) = h(f(a) + g(a)) = h(f(a)) + h(g(a)) = \left[ h \circ f + h \circ g \right](a),
  \end{equation*}
  so
  \begin{equation*}
    h \circ (f+g) =h \circ f + h \circ g,
  \end{equation*}
  and similarly in the other slot.
\end{example}

\begin{example}
  The category $\mathsf{Vect}_{k}$ is additive. Since vector spaces are in particular abelian groups under addition, it is naturally $\mathsf{Ab}$-enriched,
\end{example}

\begin{definition}[additive functor]
  \label{def:additivefunctor}
  Let $\mathcal{F}\colon \mathsf{C} \rightarrow \mathsf{D}$ be a functor between additive categories. We say that $\mathcal{F}$ is \defn{additive} if for each $X$, $Y \in \Obj(\mathsf{C})$ the map
  \begin{equation*}
    \Hom_{\mathsf{C}}(X, Y) \to \Hom_{\mathsf{D}}(\mathcal{F}(X), \mathcal{F}(Y))
  \end{equation*}
  is a homomorphism of abelian groups.
\end{definition}

\begin{lemma}
  For any additive functor $\mathcal{F}\colon \mathsf{C} \rightarrow \mathsf{D}$, there exists a natural isomorphism
  \begin{equation*}
    \Phi\colon \mathcal{F}(-) \oplus \mathcal{F}(-) \Rightarrow \mathcal{F}(- \oplus -).
  \end{equation*}
\end{lemma}
\begin{proof}
  The commutativity of the following diagram is immediate.
  \begin{equation*}
    \begin{tikzcd}[column sep=huge, row sep=huge]
      \mathcal{F}(X \oplus Y)
      \arrow[r, "\mathcal{F}(f \oplus g)"]
      \arrow[d, swap, "\Phi_{X, Y}"]
      & \mathcal{F}(X' \oplus Y')
      \arrow[d, "\Phi_{X', Y'}"]
      \\
      \mathcal{F}(X) \oplus \mathcal{F}(Y)
      \arrow[r, "\mathcal{F}(f) \oplus \mathcal{F}(g)"]
      & \mathcal{F}(X') \oplus \mathcal{F}(Y')
    \end{tikzcd}
  \end{equation*}

  The $(X, Y)$-component $\Phi_{X, Y}$ is an isomorphism because
\end{proof}


\section{Pre-abelian categories}~\label{sec:preabeliancategories}

\begin{definition}[pre-abelian category]
  \label{def:preabeliancategory}
  A category  $\mathsf{C}$ is \defn{pre-abelian} if it is additive and every morphism has a kernel (\hyperref[def:kernelofmorphism]{Definition~\ref*{def:kernelofmorphism}}) and a cokernel (\hyperref[def:cokernalofmorphism]{Definition~\ref*{def:cokernalofmorphism}}).
\end{definition}

\begin{lemma}
  \label{lemma:preabeliancategorieshaveequalizers}
  Pre-abelian categories have equalizers (\hyperref[def:equalizer]{Definition~\ref*{def:equalizer}}).
\end{lemma}
\begin{proof}
  We show that in an pre-abelian category, the equalizer of $f$ and $g$ coincides with the kernel of $f - g$. It suffices to show that the kernel of $f - g$ satisfies the universal property for the equalizer of $f$ and $g$.

  Here is the diagram for the universal property of the kernel of $f - g$.
  \begin{equation*}
    \begin{tikzcd}
      Z
      \arrow[rrd, bend left]
      \arrow[ddr, bend right, swap, "i"]
      \arrow[dr, dashed, "\exists!\bar{\imath}"]
      \\
      & \ker(f - g)
      \arrow[d, swap, "\iota_{f - g}"]
      \arrow[r]
      & 0
      \arrow[d]
      \\
      & A
      \arrow[r, "f - g"]
      & B
    \end{tikzcd}
  \end{equation*}

  The universal property tells us that for any object $Z \in \Obj(\mathsf{C})$ and any morphism $i\colon Z \to A$ with $i \circ (f - g) = 0$ (i.e. $i \circ f = i \circ g$), there exists a unique morphism $\bar{\imath}\colon Z \to \ker(f - g)$ such that $i = \bar{\imath} \circ \iota_{f - g}$.
\end{proof}

\begin{corollary}
  Every pre-abelian category has all finite limits and colimits.
\end{corollary}
\begin{proof}
  By \hyperref[thm:criterionforfinitelimits]{Theorem~\ref*{thm:criterionforfinitelimits}}, a category has finite limits if and only if it has finite products and equalizers. Pre-abelian categories have finite products by definition, and equalizers by \hyperref[lemma:preabeliancategorieshaveequalizers]{Lemma~\ref*{lemma:preabeliancategorieshaveequalizers}}. The colimit case is dual.
\end{proof}

Recall from \hyperref[sec:biproducts]{Section~\ref*{sec:biproducts}} the following definition.

\begin{definition}[zero morphism]
  \label{def:zeromorphism}
  Let $\mathsf{C}$ be a category with zero object $0$. For any two objects $A$, $B \in \Obj(\mathsf{C})$, the \defn{zero morphism} $0_{A,B}$ is the unique morphism $A \to B$ which factors through $0$.
  \begin{equation*}
    \begin{tikzcd}
      A
      \arrow[rr, bend left, "0_{A, B}"]
      \arrow[r]
      & 0
      \arrow[r]
      & B
    \end{tikzcd}
  \end{equation*}
\end{definition}

\begin{notation}
  It will often be clear what the source and destination of the zero morphism are; in this case we will drop the subscripts, writing $0$ instead of $0_{AB}$.
\end{notation}

It is easy to see that the left- or right-composition of the zero morphism with any other morphism results in the zero morphism: $f \circ 0 = 0$ and $0 \circ g = 0$.

\begin{lemma}
  \label{lemma:everymorphisminapreabeliancategoryfactors}
  Every morphism $f\colon A \to B$ in a pre-abelian catgory has a canonical decomposition
  \begin{equation*}
    \begin{tikzcd}
      A
      \arrow[r, "p"]
      & \coker(\ker(f))
      \arrow[r, "\bar{f}"]
      & \ker(\coker(f))
      \arrow[r, "i"]
      & B
    \end{tikzcd},
  \end{equation*}
  where $p$ is an epimorphism (\hyperref[def:epimorphism]{Definition~\ref*{def:epimorphism}}) and $i$ is a monomorphism (\hyperref[def:monomorphism]{Definition~\ref*{def:monomorphism}}).
\end{lemma}
\begin{proof}
  We start with a map
  \begin{equation*}
    \begin{tikzcd}
      A
      \arrow[r, "f"]
      & B
    \end{tikzcd}.
  \end{equation*}

  Since we are in a pre-abelian category, we are guaranteed that $f$ has a kernel $(\ker(f), \iota)$ and a cokernel $(\coker(f), \pi)$. From the universality squares it is immediate that $f \circ \iota = 0$ and $\pi \circ f = 0$ This tells us that the composition $\pi \circ f \circ \iota = 0$, so the following commutes.
  \begin{equation*}
    \begin{tikzcd}
      & A
      \arrow[r, "f"]
      & B
      \arrow[dr, "\pi"]
      \\
      \ker(f)
      \arrow[ur, "\iota"]
      \arrow[dr]
      & & & \coker(f)
      \\
      & 0
      \arrow[r]
      & 0
      \arrow[ur]
    \end{tikzcd}
  \end{equation*}
  We know that $\pi$ has a kernel $(\ker(\pi), i)$ and $\iota$ has a cokernel $(\coker(\iota), p)$, so we can add their commutativity squares as well.
  \begin{equation*}
    \begin{tikzcd}
      & A
      \arrow[r, "f"]
      \arrow[d, "p"]
      & B
      \arrow[dr, "\pi"]
      \\
      \ker(f)
      \arrow[ur, "\iota"]
      \arrow[dr]
      & \coker(\iota)
      & \ker(\pi)
      \arrow[u, "i"]
      \arrow[d]
      & \coker(f)
      \\
      & 0
      \arrow[u]
      \arrow[r]
      & 0
      \arrow[ur]
    \end{tikzcd}
  \end{equation*}

  If we squint hard enough, we can see the following diagram.
  \begin{equation*}
    \begin{tikzcd}
      A
      \arrow[rrd, bend left]
      \arrow[rdd, bend right, swap, "f"]
      \\
      & \ker(\pi)
      \arrow[r]
      \arrow[d, "i"]
      & 0
      \arrow[d]
      \\
      & B
      \arrow[r, "\pi"]
      & \coker(f)
    \end{tikzcd}
  \end{equation*}
  The outer square commutes because $\pi \circ f = 0$, so the universal property for $\ker(\pi)$ gives us a unique morphism $A \to \ker(\pi)$. Let's add this to our diagram, along with a morphism $0 \to \ker(\pi)$ which trivially keeps everything commutative.
  \begin{equation*}
    \begin{tikzcd}
      & A
      \arrow[r, "f"]
      \arrow[d, "p"]
      \arrow[dr]
      & B
      \arrow[dr, "\pi"]
      \\
      \ker(f)
      \arrow[ur, "\iota"]
      \arrow[dr]
      & \coker(\iota)
      & \ker(\pi)
      \arrow[u, "i"]
      \arrow[d]
      & \coker(f)
      \\
      & 0
      \arrow[u]
      \arrow[r]
      \arrow[ur]
      & 0
      \arrow[ur]
    \end{tikzcd}
  \end{equation*}

  Again, buried in the bowels of our new diagram, we find the following.
  \begin{equation*}
    \begin{tikzcd}
      \ker(f)
      \arrow[r, "\iota"]
      \arrow[d]
      & A
      \arrow[d, "\pi"]
      \arrow[ddr, bend left]
      \\
      0
      \arrow[r]
      \arrow[drr, bend right]
      & \coker(\iota)
      \\
      & & \ker(\pi)
    \end{tikzcd}
  \end{equation*}
  And again, the universal property of cokernels gives us a unique morphism $\bar{f}\colon \coker(\iota) \to \ker(\pi)$.
  \begin{equation*}
    \begin{tikzcd}
      & A
      \arrow[r, "f"]
      \arrow[d, "p"]
      \arrow[dr]
      & B
      \arrow[dr, "\pi"]
      \\
      \ker(f)
      \arrow[ur, "\iota"]
      \arrow[dr]
      & \coker(\iota)
      \arrow[r, "\bar{f}"]
      & \ker(\pi)
      \arrow[u, "i"]
      \arrow[d]
      & \coker(f)
      \\
      & 0
      \arrow[u]
      \arrow[r]
      \arrow[ur]
      & 0
      \arrow[ur]
    \end{tikzcd}
  \end{equation*}

  The fruit of our laborious construction is the following commuting square.
  \begin{equation*}
    \begin{tikzcd}
      A
      \arrow[r, "f"]
      \arrow[d, swap, "p"]
      & B
      \arrow[d, "i"]
      \\
      \coker(\iota)
      \arrow[r, "\bar{f}"]
      & \ker(\pi)
    \end{tikzcd}
  \end{equation*}
  We have seen (\hyperref[lemma:canonicalinjectionismono]{Lemma~\ref*{lemma:canonicalinjectionismono}}) that $i$ is mono, and (\hyperref[lemma:canonicalsurjectionisepi]{Lemma~\ref*{lemma:canonicalsurjectionisepi}}) that $p$ is epi.

  Now we abuse terminology by calling $\iota = \ker(f)$ and $\pi = \coker(f)$. Then we have the required decomposition.
\end{proof}

\begin{note}
  The abuse of notation above is ubiquitous in the literature.
\end{note}


\section{Abelian categories}

This section is under very heavy construction. Don't trust anything you read here.
\begin{definition}[abelian category]
  \label{def:abeliancategory}
  A pre-abelian category $\mathsf{C}$ is \defn{abelian} if every monic is the kernel of its cokernel, and every epic is the cokernel of its kernel.
\end{definition}

\begin{note}
  The above piecemeal definition is equivalent to the following.

  A category $\mathsf{C}$ is \emph{abelian} if
  \begin{enumerate}
    \item it is $\mathsf{Ab}$-enriched \hyperref[def:additivecategory]{Definition~\ref*{def:additivecategory}}, i.e.\ each hom-set has the structure of an abelian group and composition is bilinear;
    \item it admits finite coproducts, hence (by \hyperref[lemma:abeliancategorycoproductsareproducts]{Lemma~\ref*{lemma:abeliancategorycoproductsareproducts}}) biproducts and zero objects;
    \item every morphism has a kernel and a cokernel;
    \item every monomorphism is the kernel of its cokernel, and every epimorphism is the cokernel of its kernel.
  \end{enumerate}
\end{note}

\begin{example}
  \leavevmode
  \begin{itemize}
    \item Let $R$ be a ring. Then the categories $R\mhyp\mathrm{Mod}$ and $\mathrm{Mod}\mhyp R$ of left- and right $R$-modules respectively, are abelian.

    \item If $\mathsf{A}$ is abelian, then $\mathrm{Ch}(\mathsf{A})$, the category of chain complexes in $\mathsf{A}$, is abelian.

    \item If $I$ is a small category, then $\mathrm{Fun}(I, \mathsf{A})$ is abelian.
      \begin{itemize}
        \item With $I = BG$ for some group $G$, this gives the category of $\mathsf{A}$-representations of $G$

        \item With $X$ a topological space and $\mathrm{Open}(X)$ its category of opens, taking $I = \mathrm{Open}^{\op})$ tells us that the category of presheaves on $X$, denoted $\mathrm{Psh}(X)$, is abelian.
      \end{itemize}
  \end{itemize}
\end{example}

\begin{example}
  Here are two more tricky cases.
  \begin{itemize}
    \item Consider $R = \C[x_{1}, x_{2}, \ldots]$ the ring of polynomials in countably many variables. Consider the full subcategory
      \begin{equation*}
        (R\mhyp\mathrm{Mod})^{\mathrm{fin\ gen}} \hookrightarrow R\mhyp\mathrm{Mod}.
      \end{equation*}

      This is additive (as a full subcategory), but does not have kernels in general. For example, consider the quotient
      \begin{equation*}
        f\colon R \twoheadrightarrow R/(x_{1}, \ldots) \overset{\text{as a vector space}}{\simeq} \C.
      \end{equation*}
      Both $R$ and $R/(x_{1}, \ldots)$ are finitely generated $R$-modules. It may at first seem that there cannot be a kernel since the kernel of $f$ is $(x_{1}, \ldots)$, which is not finitely generated. But this is not the kernel we are looking for---it is taken in the category of $R\mhyp\mathrm{Mod}$, which satisfies a much stronger universal property than the hypothetical kernel in $(R\mhyp\mathrm{Mod})$.

      However, it turns out that there is no saving this construction. Suppose there was a kernel. It would have to be finitely generated, i.e.\ we could write
      \begin{equation*}
        \ker f = R[g_{1}, \ldots, g_{n}]/(\cdots).
      \end{equation*}
      Each of the $g_{i}$ are polynomials in (finitely many) variables $x_{i}$; in particular, there is a highest $i$ such that $x_{i}$ appears in one of the $g_{i}$. But then the map
  \end{itemize}
\end{example}

For the remainder of the section, let $\mathsf{C}$ be an abelian category.

\begin{lemma}
  \label{lemma:kernel_of_monic_cokernel_of_epic_are_zero}
  Let $f\colon A \to B$ be a morphism in an abelian category.
  \begin{itemize}
    \item If $f$ is mono, then $\ker f = 0$.

    \item If $f$ is epi, then $\coker f = 0$.
  \end{itemize}
\end{lemma}
\begin{proof}
  Suppose $f$ is mono. The kernel of $f$ has the property that for any morphism $g\colon C \to A$ such that $f \circ g = 0$, there exists a unique morphism $h\colon C \to \ker f$ such that $\iota \circ h  = g$. However, if $f$ is mono, then $f \circ g = 0 \implies g = 0$, so the zero morphism also has this property. Thus, $0$ is a kernel of $f$.
  \begin{equation*}
    \begin{tikzcd}
      & C
      \arrow[dr, "0"]
      \arrow[d, swap, "g = 0"]
      \arrow[dl, swap, "\exists!0"]
      \\
      0
      \arrow[r]
      & X
      \arrow[r, swap, "f"]
      & Y
    \end{tikzcd}
  \end{equation*}

  The cokernel case is dual.
\end{proof}

\begin{lemma}
  \label{lemma:mono_plus_epi_equals_iso_in_abelian_cat}
  Let $\mathcal{A}$ be an abelian category, and let $f\colon A \to B$ be a morphism in $\mathcal{A}$. Then $f$ is an isomorphism if and only if $f$ is monic and epic.
\end{lemma}
\begin{proof}
  The direction iso$\implies$monic and epic is obvious. To see the other direction, suppose that $f$ is monic and epic. By monicness, \hyperref[lemma:kernel_of_monic_cokernel_of_epic_are_zero]{Lemma~\ref*{lemma:kernel_of_monic_cokernel_of_epic_are_zero}} implies that $\ker f = 0$. Because $f$ is epic, $f$ is the cokernel of the map $0 \to X$.
\end{proof}

\begin{lemma}
  In an abelian category, every morphism decomposes into the composition of an epimorphism and a monomorphism.
\end{lemma}
\begin{proof}
  %For any morphism $f$, bracketing the decomposition $f = i \circ \bar{f} \circ p$ as
  %\begin{equation*}
  %  i \circ (\bar{f} \circ p).
  %\end{equation*}
  %gives such a composition.
\end{proof}

\begin{note}
  The above decomposition is unique up to unique isomorphism.
\end{note}

\begin{definition}[image of a morphism]
  \label{def:imageofamorphism}
  Let $f\colon A \to B$ be a morphism. The object $\ker(\coker(f))$ is called the \defn{image} of $f$, and is denoted $\im(f)$.
\end{definition}

\begin{lemma}
  \label{lemma:morphismwhichkillsontherightismono}
  $\,$
  \begin{enumerate}
    \item A morphism $f\colon A \to B$ is mono iff for all $Z \in \Obj(\mathsf{C})$ and for all $g\colon Z \to A$, $f \circ g = 0$ implies $g = 0$.
    \item A morphism $f\colon A \to B$ is epi iff for all $Z \in \Obj(\mathsf{C})$ and for all $g\colon B \to Z$, $g \circ f = 0$ implies $g = 0$.
  \end{enumerate}
\end{lemma}
\begin{proof}
  $\,$
  \begin{enumerate}
    \item First, suppose $f$ is mono. Consider the following diagram.
      \begin{equation*}
        \begin{tikzcd}
          Z
          \arrow[r, shift left, "g"]
          \arrow[r, swap, shift right, "0"]
          & A
          \arrow[r, "f"]
          & B
        \end{tikzcd}
      \end{equation*}

      If the above diagram commutes, i.e.\ if $f \circ g = 0$, then $g = 0$, so $1 \implies 2$.

      Now suppose that for all $Z \in \Obj(\mathsf{C})$ and all $g\colon Z \to A$, $f \circ g = 0$ implies $g = 0$.

      Let $g$, $g'\colon Z \to A$, and suppose that $f \circ g = f \circ g'$. Then $f \circ (g - g') = 0$. But that means that $g - g' = 0$, i.e. $g = g'$. Thus, $2 \implies 1$.

    \item Dual to the proof above.
  \end{enumerate}
\end{proof}

\begin{lemma}
  \label{lemma:monoifkernelzeroepiifcokerzero}
  Let $f\colon A \to B$. We have the following.
  \begin{enumerate}
    \item The morphism $f$ is mono iff $\ker(f) = 0$\label{part:monoifkernelzeroepiifcokerzero1}
    \item The morphism $f$ is epi iff $\coker(f) = 0$.\label{part:monoifkernelzeroepiifcokerzero2}
  \end{enumerate}
\end{lemma}
\begin{proof}
  $\,$
  \begin{enumerate}
    \item We first show that if $\ker(f) = 0$, then $f$ is mono. Suppose $\ker(f) = 0$. By the universal property of kernels, we know that for any $Z \in \Obj(\mathsf{C})$ and any $g\colon Z \to A$ with $f \circ g = 0$ there exists a unique map $\bar{g}\colon Z \to \ker(f)$ such that $g = \iota \circ \bar{g}$.
      \begin{equation*}
        \begin{tikzcd}
          Z
          \arrow[rdd, bend right, swap, "g"]
          \arrow[rd, dashed, "\exists!g"]
          \\
          & \ker(f) = 0
          \arrow[r]
          \arrow[d, swap, "\iota"]
          & 0
          \arrow[d]
          \\
          & A
          \arrow[r, "f"]
          & B
        \end{tikzcd}
      \end{equation*}
      But then $g$ factors through the zero object, so we must have $g = 0$. This shows that $f \circ g = 0 \implies g = 0$, and by \hyperref[lemma:morphismwhichkillsontherightismono]{Lemma~\ref*{lemma:morphismwhichkillsontherightismono}} $f$ must be mono.

      Next, we show that if $f$ is mono, then $\ker(f) = 0$. To do this, it suffices to show that $\ker(f)$ is final, i.e.\ that there exists a unique morphism from every object to $\ker(f)$.

      Since $\Hom_{\mathsf{C}}(Z, \ker(f))$ has the structure of an abelian group, it must contain at least one element. Suppose it contains two morphisms $h_{1}$ and $h_{2}$.
      \begin{equation*}
        \begin{tikzcd}
          Z
          \arrow[rd, shift left, "h_{1}"]
          \arrow[rd, shift right, swap, "h_{2}"]
          \\
          & \ker(f)
          \arrow[r]
          \arrow[d, swap, "\iota"]
          & 0
          \arrow[d]
          \\
          & A
          \arrow[r, "f"]
          & B
        \end{tikzcd}
      \end{equation*}
      Our aim is to show that $h_{1} = h_{2}$. To this end, compose each with $\iota$.
      \begin{equation*}
        \begin{tikzcd}
          Z
          \arrow[rd, shift left, "h_{1}"]
          \arrow[rd, shift right, swap, "h_{2}"]
          \arrow[rdd, bend right, shift left, "\iota \circ h_{1}"]
          \arrow[rdd, bend right, shift right, swap, "\iota \circ h_{2}"]
          \\
          & \ker(f)
          \arrow[r]
          \arrow[d, swap, "\iota"]
          & 0
          \arrow[d]
          \\
          & A
          \arrow[r, "f"]
          & B
        \end{tikzcd}
      \end{equation*}
      Since $f \circ \iota = 0$, we have $f \circ (\iota \circ h_{1}) = 0$ and $f \circ (\iota \circ h_{2}) = 0$. But since $f$ is mono, by \hyperref[lemma:morphismwhichkillsontherightismono]{Lemma~\ref*{lemma:morphismwhichkillsontherightismono}}, we must have $\iota \circ h_{1} = 0 = \iota \circ h_{2}$. But by \hyperref[lemma:canonicalinjectionismono]{Lemma~\ref*{lemma:canonicalinjectionismono}}, $\iota$ is mono, so again we have
      \begin{equation*}
        h_{1} = h_{2} = 0,
      \end{equation*}
      and we are done.

    \item Dual to the proof above.
  \end{enumerate}
\end{proof}

\begin{lemma}
  We have the following.
  \begin{enumerate}
    \item The kernel of the zero morphism $0: A \to B$ is the pair $(A, \id_{A})$.
    \item The cokernel of the zero morphism $0: A \to B$ is the pair $(B, \id_{B})$.
  \end{enumerate}
\end{lemma}
\begin{proof}
  $\,$
  \begin{enumerate}
    \item We need only verify that the universal property is satisfied. That is, for any object $Z \in \Obj(\mathsf{C})$ and any morphism $h\colon Z \to A$ such that $0 \circ g = 0$, there exists a unique morphism $\bar{g}\colon Z \to \ker(0)$ such that the following diagram commutes.
      \begin{equation*}
        \begin{tikzcd}
          Z
          \arrow[rdd, swap, bend right, "g"]
          \arrow[rd, dashed, "\bar{g}=g"]
          \\
          & \ker(0) = A
          \arrow[r, "0"]
          \arrow[d, swap, "\iota = \id_{A}"]
          & 0
          \arrow[d]
          \\
          & A
          \arrow[r, "0"]
          & B
        \end{tikzcd}
      \end{equation*}
      But this is pretty trivial: $\bar{g} = g$.

    \item Dual to above.
  \end{enumerate}
\end{proof}

\begin{theorem}
  All abelian categories are binormal (\hyperref[def:binormalcategory]{Definition~\ref*{def:binormalcategory}}). That is to say:
  \begin{enumerate}
    \item all monomorphisms are kernels
    \item all epimorphisms are cokernels.
  \end{enumerate}
\end{theorem}
\begin{proof}
  $\,$
  \begin{enumerate}
    \item Consider the following diagram taken \hyperref[lemma:everymorphisminapreabeliancategoryfactors]{Lemma~\ref*{lemma:everymorphisminapreabeliancategoryfactors}}, which shows the canonical factorization of any morphism $f$.
      \begin{equation*}
        \begin{tikzcd}
          & A
          \arrow[r, "f"]
          \arrow[d, "p"]
          \arrow[dr]
          & B
          \arrow[dr, "\pi"]
          \\
          \ker(f)
          \arrow[ur, "\iota"]
          \arrow[dr]
          & \coker(\iota)
          \arrow[r, "\bar{f}"]
          & \ker(\pi)
          \arrow[u, "i"]
          \arrow[d]
          & \coker(f)
          \\
          & 0
          \arrow[u]
          \arrow[r]
          \arrow[ur]
          & 0
          \arrow[ur]
        \end{tikzcd}
      \end{equation*}

      By definition of a pre-abelian category, we know that $\bar{f}$ is an isomorphism.
  \end{enumerate}
\end{proof}

\begin{note}
  The above theorem is actually an equivalent definition of an abelian category, but the proof of equivalence is far from trivial. See e.g.~\cite{freyd-abelian-categories} for details.
\end{note}

\begin{definition}[subobject, quotient object, subquotient object]
  \label{def:subobjectquotientobject}
  Let $Y \in \Obj(\mathsf{C})$.
  \begin{enumerate}
    \item A \defn{subobject} of $Y$ is an object $X \in \Obj(\mathsf{C})$ together with a monomorphism $i\colon X \hookrightarrow Y$. If $X$ is a subobject of $Y$ we will write $X \subseteq Y$.
    \item A \defn{quotient object} of $Y$ is an object $Z$ together with an epimorphism $p\colon Y \twoheadrightarrow Z$.
    \item A \defn{subquotient object} of $Y$ is a quotient object of a subobject of $Y$.
  \end{enumerate}
\end{definition}

%\begin{lemma}
%  Let $O$ be an object, $Z$ be a subquotient of $O$, and $Z'$ a subquotient of $Z$. Then $Z'$ is a subquotient of $O$.
%\end{lemma}
%\begin{proof}
%  A subquotient $Z$ of $O$ is a quotient $Z$ of a subobject $X$ of $O$.
%  \begin{equation*}
%    \begin{tikzcd}
%      X
%      \arrow[r, hookrightarrow]
%      \arrow[d, twoheadrightarrow]
%      & O
%      \\
%      Z
%    \end{tikzcd}
%  \end{equation*}
%  A subquotient $Z'$ of $Z$ looks like this.
%  \begin{equation*}
%    \begin{tikzcd}
%      & X
%      \arrow[r, hookrightarrow]
%      \arrow[d, twoheadrightarrow]
%      & O
%      \\
%      X'
%      \arrow[r, hookrightarrow]
%      \arrow[d, twoheadrightarrow]
%      & Z
%      \\
%      Z'
%    \end{tikzcd}
%  \end{equation*}
%
%  Take pullback yadda yadda. Will finish later.
%\end{proof}

\begin{definition}[quotient]
  \label{def:quotient}
  Let $X \subseteq Y$, i.e.\ let there exist a monomorphism $f\colon X \hookrightarrow Y$. The \defn{quotient} $Y/X$ is the cokernel $(\coker(f), \pi_{f})$.
\end{definition}

\begin{example}
  Let $V$ be a vector space, $W \subseteq V$ a subspace. Then we have the canonical inclusion map $\iota\colon W \hookrightarrow V$, so $W$ is a subobject of $V$ in the sense of \hyperref[def:subobjectquotientobject]{Definition~\ref*{def:subobjectquotientobject}}.

  According to \hyperref[def:quotient]{Definition~\ref*{def:quotient}}, the quotient $V / W$ is the cokernel $(\coker(\iota), \pi_{\iota})$ of $\iota$. We saw in \hyperref[eg:invectcokernelsarequotientsbyimage]{Example~\ref*{eg:invectcokernelsarequotientsbyimage}} that the cokernel of $\iota$ was $V / \im(\iota)$. However, $\im(\iota)$ is exactly $W$! So the categorical notion of the quotient $V / W$ agrees with the linear algebra notion.
\end{example}

\begin{definition}[$k$-linear category]
  \label{def:linearcategory}
  An abelian category \hyperref[def:abeliancategory]{Definition~\ref*{def:abeliancategory}} $\mathsf{C}$ is \defn{$k$-linear} if for all $A$, $B \in \Obj(\mathsf{C})$ the hom-set $\Hom_{\mathsf{C}}(A, B)$ has the structure of a $k$-vector space whose additive structure is the abelian structure, and for which the composition of morphisms is $k$-linear.
\end{definition}

\begin{example}
  The category $\mathsf{Vect}_{k}$ is $k$-linear.
\end{example}

\begin{definition}[$k$-linear functor]
  \label{def:linearfunctor}
  let $\mathsf{C}$ and $\mathsf{D}$ be two $k$-linear categories, and $\mathcal{F}\colon \mathsf{C} \rightarrow \mathsf{D}$ a functor. Suppose that for all objects $C$, $D \in \Obj(\mathsf{C})$ all morphisms $f$, $g\colon C \to D$, and all $\alpha$, $\beta \in k$, we have
  \begin{equation*}
    \mathcal{F}(\alpha f + \beta g) = \alpha \mathcal{F}(f) + \beta \mathcal{G}(g).
  \end{equation*}

  Then we say that $\mathcal{F}$ is \defn{$k$-linear}.
\end{definition}


\section{Exact sequences}

\begin{definition}[exact sequence]
  \label{def:exactsequence}
  A sequence of morphisms
  \begin{equation*}
    \begin{tikzcd}
      \cdots
      \arrow[r]
      & X_{i-1}
      \arrow[r, "f_{i-1}"]
      & X_{i}
      \arrow[r, "f_{i}"]
      & X_{i+1}
      \arrow[r]
      & \cdots
    \end{tikzcd}
  \end{equation*}
  is called \defn{exact in degree $i$} if the image (\hyperref[def:imageofamorphism]{Definition \ref*{def:imageofamorphism}}) of $f_{i-1}$ is equal to the kernel (\hyperref[def:kernelofmorphism]{Definition \ref*{def:kernelofmorphism}}) of $f_{i}$. A sequence is \defn{exact} if it is exact in every degree.
\end{definition}

\begin{lemma}
  If a sequence
  \begin{equation*}
    \begin{tikzcd}
      \cdots
      \arrow[r]
      & X_{i-1}
      \arrow[r, "f_{i-1}"]
      & X_{i}
      \arrow[r, "f_{i}"]
      & X_{i+1}
      \arrow[r]
      & \cdots
    \end{tikzcd}
  \end{equation*}
  is exact in degree $i$, then $f_{i} \circ f_{i-1} = 0$.
\end{lemma}

\begin{definition}[short exact sequence]
  \label{def:shortexactsequence}
  A \defn{short exact sequence} is an exact sequence of the following form.
  \begin{equation*}
    \begin{tikzcd}
      0
      \arrow[r]
      & X
      \arrow[r]
      & Y
      \arrow[r]
      & Z
      \arrow[r]
      & 0
    \end{tikzcd}
  \end{equation*}
\end{definition}

%This definition has some immediate trivial consequences.
%\begin{lemma}
%  Let
%  \begin{equation*}
%    \begin{tikzcd}
%      0
%      \arrow[r]
%      & X
%      \arrow[r, "f"]
%      & Y
%      \arrow[r, "g"]
%      & Z
%      \arrow[r]
%      & 0
%    \end{tikzcd}
%  \end{equation*}
%  be a short exact sequence. Then
%  \begin{enumerate}
%    \item $f$ is mono
%    \item $g$ is epi
%    \item $Z \simeq Y/X$.
%  \end{enumerate}
%\end{lemma}
%\begin{proof}
%  $\,$
%  \begin{enumerate}
%    \item An easy consequence of \hyperref[lemma:monoifkernelzeroepiifcokerzero]{Lemma \ref*{lemma:monoifkernelzeroepiifcokerzero}, part \ref*{part:monoifkernelzeroepiifcokerzero1}}.
%    \item An easy consequence of \hyperref[lemma:monoifkernelzeroepiifcokerzero]{Lemma \ref*{lemma:monoifkernelzeroepiifcokerzero}, part \ref*{part:monoifkernelzeroepiifcokerzero2}}.
%    \item We need to show that
%  \end{enumerate}
%\end{proof}

\begin{definition}[exact functor]
  \label{def:exactfunctor}
  Let $\mathsf{C}$, $\mathsf{D}$ be abelian categories, $\mathcal{F}\colon \mathsf{C} \rightarrow \mathsf{D}$ an additive functor (\hyperref[def:additivefunctor]{Definition~\ref*{def:additivefunctor}}). We say that $\mathcal{F}$ is
  \begin{itemize}
    \item \defn{left exact} if it preserves biproducts and kernels
    \item \defn{right exact} if it preserves biproducts and cokernels
    \item \defn{exact} if it is both left exact and right exact.
  \end{itemize}
\end{definition}

%\begin{theorem}
%  Let $\mathcal{F}\colon \mathsf{C} \rightarrow \mathsf{D}$ be a functor between abelian categories. Let
%  \begin{equation*}
%    \begin{tikzcd}
%      0
%      \arrow[r]
%      & A
%      \arrow[r]
%      & B
%      \arrow[r]
%      & C
%      \arrow[r]
%      & 0
%    \end{tikzcd}
%  \end{equation*}
%  be a short exact sequence in $\mathsf{C}$. Then
%  \begin{itemize}
%    \item if $\mathcal{F}$ is left exact, then
%      \begin{equation*}
%        \begin{tikzcd}
%          0
%          \arrow[r]
%          & \mathcal{F}(A)
%          \arrow[r]
%          & \mathcal{F}(B)
%          \arrow[r]
%          & \mathcal{F}(C)
%        \end{tikzcd}
%      \end{equation*}
%      is an exact sequence in $\mathsf{D}$.
%
%    \item if $\mathcal{F}$ is right exact, then
%      \begin{equation*}
%        \begin{tikzcd}
%          \mathcal{F}(A)
%          \arrow[r]
%          & \mathcal{F}(B)
%          \arrow[r]
%          & \mathcal{F}(C)
%          \arrow[r]
%          & 0
%        \end{tikzcd}
%      \end{equation*}
%      is an exact sequence in $\mathsf{D}$.
%    \item if $\mathcal{F}$ is exact, then
%      \begin{equation*}
%        \begin{tikzcd}
%          0
%          \arrow[r]
%          & \mathcal{F}(A)
%          \arrow[r]
%          & \mathcal{F}(B)
%          \arrow[r]
%          & \mathcal{F}(C)
%          \arrow[r]
%          & 0
%        \end{tikzcd}
%      \end{equation*}
%      is an exact sequence in $\mathsf{D}$.
%  \end{itemize}
%\end{theorem}
%\begin{proof}
%
%\end{proof}


\section{Length of objects}

\begin{definition}[simple object]
  \label{def:simpleobject}A nonzero object $X \in \Obj(\mathsf{C})$ is called \defn{simple} if $0$ and $X$ are its only subobjects.
\end{definition}

\begin{example}
  In $\mathsf{Vect}_{k}$, the only simple object (up to isomorphism) is $k$, taken as a one-dimensional vector space over itself.
\end{example}

\begin{definition}[semisimple object]
  \label{def:semisimpleobject}
  An object $Y \in \Obj(\mathsf{C})$ is \defn{semisimple} if it is isomorphic to a dirct sum of simple objects.
\end{definition}

\begin{example}
  In $\mathsf{Vect}_{k}$, all finite-dimensional vector spaces are semisimple.
\end{example}

\begin{definition}[semisimple category]
  \label{def:semisimple category}
  An abelian category $\mathsf{C}$ is \defn{semisimple} if every object of $\mathsf{C}$ is semisimple.
\end{definition}

\begin{example}
  The category $\mathsf{FinVect}_{k}$ is semisimple.
\end{example}

\begin{definition}[Jordan-H{\"o}lder series]
  \label{def:jordanholderseries}
  Let $X \in \Obj(\mathsf{C})$. A filtration
  \begin{equation*}
    0 = X_{0} \subset X_{1} \subset \cdots \subset X_{n-1} \subset X_{n} = X
  \end{equation*}
  of $X$ such that $X_{i} / X_{i-1}$ is simple for all $i$ is called a \defn{Jordan H{\"o}lder series} for $X$. The integer $n$ is called the \defn{length} of the series $X_{i}$.
\end{definition}

The importance of Jordan-H{\"o}lder series is the following.

\begin{theorem}[Jordan-H{\"o}lder]
  \label{thm:jordanholdertheorem}
  Let $X_{i}$ and $Y_{i}$ be two Jordan-H{\"o}lder series for some object $X \in \Obj(\mathsf{C})$. Then the length of $X_{i}$ is equal to the length of $Y_{i}$, and the objects $Y_{i}/Y_{i-1}$ are a reordering of $X_{i}/X_{i-1}$.
\end{theorem}
\begin{proof}
  See~\cite{EGNO-tensor-categories}, pg. 5, Theorem 1.5.4.
\end{proof}

\begin{definition}[length]
  \label{def:length}
  The \defn{length} of an object $X$ is defined to be the length of any of its Jordan-H{\"o}lder series. This is well-defined by \hyperref[thm:jordanholdertheorem]{Theorem~\ref*{thm:jordanholdertheorem}}.
\end{definition}

\end{document}
